\chapter{Application and results}

\minitoc

Given previously trained detection and classification systems, this section describes how they can be put to production. They can serve several purposes such as conservation of endangered species via ship collision mitigation, presence monitoring (in relation to anthropic pressure for instance), and the analysis of song and communication patterns.


\section{Building an alert system for collision risk mitigation}
\label{chap:GIAS}

\subsection{Context and objective}

\begin{figure}[!htb]
   \begin{minipage}{0.8\textwidth}
     \centering
     \includegraphics[width=\linewidth]{fig/bombyx4.png}
   \end{minipage}\hfill
   \begin{minipage}{0.15\textwidth}
     \centering
     \includegraphics[width=\linewidth]{fig/bombyx1.png}
   \end{minipage}\hfill
   \caption{Technical plans of the Bombyx 2 system, taken from OSEAN SAS manufacturing report. (left) Mooring system. (right) Acoustic recorder and floatability variation system (total height of 3 meters).}
\end{figure}

As part of the GIAS project aiming at reducing navigation risks in the Mediterranean sea, the Bombyx 2 buoy was designed, in a collaboration between Toulon University and OSEAN SAS. This buoy is equipped with 5 hydrophones, a floatability variation system, and embedded algorithms for the detection of sperm whale clicks and 20Hz fin whale pulses. To mitigate surface noise and exposure to strong weather conditions, the buoy parks at 25 meters depth to record and acoustically detect its target species (sperm whales and fin whales). In the event of a detection, the buoy reaches the surface to transmit the alert with supporting data via the mobile network. The alerts then enable nearby ferries to make decisions to mitigate their risk of collision (reducing the speed or changing route for instance).


\subsection{\ac{CNN} deployment to an embedded uController}

Section \ref{chap:lightweight} introduced low complexity \acp{CNN}, especially designed to answer the needs of this alert system. These models, after being trained on \acp{GPU} using the Pytorch package \cite{NEURIPS2019_9015}, had to be implemented on the embedded system, namely the Microchip PIC32 micro-controller (integrated on the High-Blue sound card \cite{barchasz2020novel}).

This demanded to build a custom interface to export and load architectures and weights. The exports are done in Python to text files, and imported in C (required programming language for the micro-controller). Design choices were made for the C implementation, for a compromise between flexibility and low development effort :
%
\begin{itemize} \setlength{\itemsep}{1pt}
    \item The model input consists in a Mel-spectrogram
    \item Signal length, sampling frequency, window length, hop size, number of Mel bands, and Mel frequency boundaries are parametric
    \item The architecture consists of successive depth-wise separable convolution layers intertwined with batch normalization and leaky \ac{ReLU}
    \item The number of layers, and the number of features, kernel sizes and strides for each layer are parametric
    \item The last layer is pooled by maximum to yield a global prediction of the signal
\end{itemize}


\subsection{Computation times}

Specifications of the input parameters and processing time for the two target signals are given in Tab. \ref{tab:pic_implem}.

\begin{table}[ht]
    \centering
    \begin{tabular}{l|c|c}
    Target signal & Sperm whale clicks & 20Hz fin whale pulse \\ \hline
    Signal length (sec) & 10 & 60 \\
    Sampling frequency (kHz) & 64 & 4 \\
    \ac{FFT} window length & 512 & 4096 \\
    \ac{FFT} hop size & 256 & 256 \\
    Mel bands & 64 & 64 \\
    Mel start (Hz) & 2,000 & 0 \\
    Mel end (Hz) & 25,000 & 100 \\
    Spectrogram computation (sec) & & \\
    \ac{CNN} inference (sec) & & \\
    \end{tabular}
    \caption{Specifications and corresponding processing times on the PIC32 microprocessor, for the detection mechanisms of sperm whale clicks and 20Hz fin whale pulses.}
    \label{tab:pic_implem}
\end{table}

The longest step to process is by far the spectrogram computation compared to the \ac{CNN} inference. This comforts the choice of the Fourier transform which offers a fast \ac{FFT} implementation, rather than others such as the wavelet transforms.

% \subsection{Power budgeting}
% duty cycles
% \cite{rand2022effects, riera2013effects}


\subsection{Detection report}

In the event of detections triggered by the \acp{CNN}, the buoy is ordered to lift towards the surface to transmit a report supporting the alert. To lower the amount of data to be transmitted, chunks of signals are cut surrounding detection peaks and downsampled to lower sampling frequencies.

% TODO conclude : insight

% \section{Interpretation of model predictions}
% Check pred distribution
% Dolphin / humpbacks CARIMAM
% Carimap


\section{Long term presence monitoring}

In addition to its production use in the context of ship collision mitigation, the sperm whale click detection \ac{CNN} has been forwarded on the whole Bombyx dataset (3,532 recorded hours from May 2015 to December 2018) for a long term study of sperm whale presence. This work was conducted as a collaboration with Maxence Ferrari and Marion Poupard, and resulted in a journal publication \cite{bombyx}. Some of the results are reported here.


\subsection{Sperm whale acoustic presence}

A first analysis focused solely on reporting the presence of sperm whales through the recorder years. Files (1min long) with more than 40 \ac{CNN} predictions above 0.95 were manually analysed using the interface described in section \ref{chap:bombyx_annot}. Thus, automatic detections were validated and number of individuals were estimated (using simultaneous click trains and \ac{TDOA} tracks). This process yielded 57 new sperm whales passages (missed during the annotation procedure described in section \ref{chap:bombyx_annot}), and 25 false positives (including 15 triggered by sound card malfunctions). The notion of passage was used to account for sperm whale presence, considering that clicks belong to the same passage

In total, 226 sperm whale passages have been recovered, with a total of 347 individuals). Fig. \ref{fig:res_nbr_indiv_calendar} presents the number of detected individuals each day during the 4 years of recording, with white regions indicating the absence of recording. Sperm whales were found all year round, with no statistically significant seasonal pattern. The number of animals per passage varied from 1 to 9 individuals, with a mean duration of 4 hours.

To evaluate dial patterns, the probability of presence was computed for each hour of the day. Averaging probabilities into four periods (Night, Morning, Afternoon, and Evening) demonstrates a statistically significant differences among periods of the day : sperm whales are more present during morning or afternoons than in the evening (see Fig. \ref{fig:res_nbr_indiv_calendar}, Kruskal-Wallis test : p-value < 0.01).

\begin{figure}[h]
\centering
\includegraphics[width=\linewidth]{fig/bombyx_results_calendar.pdf}
\caption{ Left (a): The Number of detected sperm whales per day during the 4 years of recordings (white region: \textit{no d = no data}). Right (b): Mean probability of presence for each period of the day.}
\label{fig:res_nbr_indiv_calendar}
\end{figure}


\subsection{Presence in relation to anthropogenic noise pressure}

To assess the performance of the detection system as well as to measure the impact of noise on the presence of sperm whales, the amplitudes of different octave bands were computed and analyzed. The distribution of the background noise (octave 800\,Hz) through the day is shown in Fig. \ref{fig:db_cach}. All octaves' dial distributions have the same shape as the 12,800\,Hz octave, with the energy peaking around 4am and 9pm.

\begin{figure}[h]
    \centering
    \includegraphics[width=\linewidth]{fig/bombyx_results_noise.pdf}
    \caption{(left) Distribution of the amplitude for the octave 12,800\,Hz according to presence/absence of sperm whales. (right) Superposition of dial pattern of amplitudes for the octave 12,800\,Hz and probability of presence of sperm whales.}
\label{fig:db_cach}
\end{figure}

Ferries cross the study area daily, connecting Toulon or Marseille to Corsica, with scheduled times between 3am - 6am and from 8pm - 9pm. The closest ferry route is approximately 3km away from the antenna. For all octaves, dB amplitudes are significantly higher during ferry schedules (Mann–Whitney test, \textit{p-value} $<$ 0.05), with an average gain of approximately 3dB.

%Anthropogenic noises negatively influence marine mammals by affecting their abundance \cite{bowles1994relative}, their behavior \cite{richardson1997influences}, and numerous processes of importance for their well being \cite{wright2007marine} (orientation, reproduction, communication). This influence depends on many acoustic features including the intensity, the bandwidth, or the duration of the exposure. In this study, we compared the evolution of the sound pressure level according to the presence/absence of sperm whales.

Moreover, as Fig. \ref{fig:db_cach} illustrates, the data shows a significantly lower noise during the sperm whales' presence (Mann-Whitney U=14.44, sample size=300, \textit{p-value} < 0.01) for all octaves except 6,400\,Hz and 12,800\,Hz. We can thus conclude that when sperm whales are present, the noise level is lower. This is further demonstrated in Fig.~\ref{fig:db_cach} right, where, during 4 AM and 9 PM (noise peaks), the presence of sperm whales is lowest. This last figure also shows that the reduced sperm whale presence is not due to an increased background noise, since sperm whale probability drops before the background noise rises.

This study is a first demonstration of the versatility of the detection systems designed through this thesis. Indeed, they can be applied to a real-time alert system to mitigate collision risks, but also in long-term studies, revealing presence patterns that are crucial in the implementation of relevant conservation measures.


\section{Song structure analysis and temporal trends}

In parallel to the latter study, a similar one was conducted on fin whales of the Ligurian sea, again making use of the detection system designed for the GIAS buoy. The trained \ac{CNN} described in section \ref{chap:lightweight} was thus run over three available datasets : Boussole, Bombyx, and Antares (see section \ref{chap:data_Toulon}). This time, instead of presence monitoring, the study focused on fin whale song patterns, yet poorly documented in the Mediterranean sea. % TODO cite in review SR

%TODO more context : use intro

\subsection{Model inference}

While the model was trained to detect pulse presence in 5-second segments, the convolutional stack is designed to maintain the temporal resolution of the predictions throughout the network. Discarding the max pooling layer at the end of the CNN, pulse times were retained as the highest predictions above a given threshold within sliding 4 second windows. These timings are approximate up to the size of the receptive field of the network (0.8 seconds).

Thresholds were set at the balance point of the \ac{ROC} curves (equal sensitivity and specificity). This setting lead to sensitivities and specificities of 0.96 and 0.97 for the Bombyx and Boussole data respectively. For the Antares data, since the \ac{ROC} curve is unknown (no annotation were available), a threshold of 0.11 was chosen, so that there is approximately the same proportion of detections as Bombyx and Boussole ($\approx$ 0.5\%) .

Tab. \ref{tab:recap} summarises the resulting detections, along with a calendar Fig. \ref{fig:calendar}. Following \citet{Watkins_Tyack_Moore_Bird_1987}, pulses at a distance of less than 45 seconds were considered as being part of the same sequence, and sequences less than 2 hours apart were considered as being part of the same bout.

\begin{table}[ht]
\centering\resizebox{\linewidth}{!}{
\begin{tabular}{|l|c|c|c|c|}
\hline
 \textbf{Data source} & \textbf{Boussole} \cite{parsuivi} & \textbf{Bombyx} \cite{vamos2017} & \textbf{Antares} \cite{giacomelli2009antares} & \textbf{Total} \\ \hline
\textbf{Location} & South of Sanremo & Port-Cros Island & Cap Sicié & \\ \hline
\textbf{Recording year} & 2008-2009 & 2015-2018 & 2020-2021 & \\ \hline
\textbf{Recorded time (hours)} & 1,860 & 3,291 & 1,124 & 6,275 \\ \hline
\textbf{Detection threshold} &  0.15 & 0.68 & 0.11 & \\ \hline
\textbf{Detected pulses} & 1,418 & 2,272 & &  \\ \hline
\textbf{Detected A pulses} & 1,182 & 1,980 & &  \\ \hline
\textbf{Detected B pulses} & 292 & 236 & & \\ \hline
\textbf{Detected sequences} & 214 & 530 & & \\ \hline
\textbf{Detected bouts} & 43 & 203 & &  \\ \hline
\end{tabular}}
\caption{\label{tab:recap}Summary of recording characteristics and automatic detections for each source of data.}
\end{table}

\begin{figure}
    \centering
    \includegraphics[width=\linewidth]{fig/rorqual_calendar.pdf}
    \caption{Number of detected sequences for each day with recordings, normalized by the amount of recorded hours.}
    \label{fig:calendar}
\end{figure}


\subsection{Spectro-temporal pulse analysis}\label{chap:analysis}

Following the detection process, a signal processing analysis was conducted to precisely describe each pulse (exact time position, center frequency, bandwidth and \ac{SNR}). 

For this analysis, an 8 sec window surrounding the prediction peak is selected (\(T=[0,8]\)), band-pass filtered (Butterworth of order 3 between 10Hz and 30Hz), and resampled at 250Hz. The \ac{STFT} is then applied to the resulting signal (Hann window of 1024 including 75\% of zero padding and 97\% overlap) resulting in spectral and temporal resolutions of 0.24Hz and 0.03sec respectively.

From this spectrogram, the precise time position of the pulse \(\hat{t}\) is first estimated by selecting the column of the maximum value in the 18-22Hz frequency band (Eq.\ref{eq:time}). This value will be kept for the later \ac{IPI} measurements. 

\begin{equation}
\hat{t} = \argmax_{t \in T}\left( \max_{f \in [18,22]}\left( \mathbf{S}(f, t)\right)\right)
\label{eq:time}
\end{equation}

To measure the spectral envelope of the pulse, a 1.2 sec window around $\hat{t}$ is max-pooled time wise. Background components are withdrawn (to focus on the pulse spectra only) by subtracting an estimate of the background spectrum: the median of each frequency bin within the window $T$ (Eq. \ref{eq:spec}). Doing so, effects such as the impact of \ac{SNR} on peak frequency and bandwidth (observed by \citet{helble2020fin}) are mitigated.

\begin{equation}
E(f) = \max_{t \in [\hat{t}-0.6, \hat{t}+0.6]} \left(\mathbf{S}(f, t)\ \right) - \   \underset{t \in T}{\mathrm{median}}\ \left( \mathbf{S}(f, t)\right)
\label{eq:spec}
\end{equation}

The resulting pulse envelope is used to compute the left and right boundaries of the pulse spectrum, with $\max_{f}\frac{E(f))}{4}$ as a threshold (equivalent to -6 dB). Left and right intersection frequencies are linearly interpolated to increase the precision of the estimate. This process yields the 6 dB bandwidth (width between the boundaries), and the center frequency (mid-point between the boundaries) of the analysed pulse.

For later filtering by pulse quality, the \ac{SNR} was computed following Eq. \ref{eq:bckgnd} (pulse energy as the maximum of its envelope and background energy as the median of the spectrogram surrounding the pulse).


\begin{equation}
E_{Background} = \underset{T \setminus [\hat{t}-1,\hat{t}+3]}{\underset{\ f \in [15, 25]}{\mathrm{median}}}\ \mathbf{S}(f, t); \;
E_{Pulse} = \max_{f}E(f); \\
SNR = 10\log_{10}\left(\frac{E_{Pulse}}{E_{Background}}\right)
\label{eq:bckgnd}
\end{equation}

The pulse spectral characteristics of mysticetes are often described using the frequency of maximum energy (peak frequency) or the spectrum weighted mean (centroid frequency) \cite{weirathmueller2017spatial, malige2020inter}. Here, the center frequency was chosen, as it appeared to be better suited for the discrimination between the two pulse types. In fact, when modeling the distribution of peak frequencies using a Gaussian mixture model, the two components (emerging from the two types of pulses) overlap more than when using center frequencies (the Kullback-Leibler divergence between the Gaussian components in center frequency is significantly higher than that of peak frequencies, 113 nats and 30 nats respectively).




inter annual IPI increase add :
\cite{wood2022characterization, furumaki2021fin} 

Blue whale freq decrease \cite{rice2022update}

[optional]
ICI sperm whales bombyx

\section{Language modelling}
Application to Orcalab

