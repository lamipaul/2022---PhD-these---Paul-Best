\chapter{Application and results}

\minitoc

Given previously trained detection and classification systems, this section describes how they can be put to production. They can serve several purposes such as conservation of endangered species via ship collision mitigation, presence monitoring (in relation to anthropic pressure for instance), and the analysis of song and communication patterns.


\section{Building an alert system for collision risk mitigation}
\label{chap:GIAS}

\subsection{Context and objective}

\begin{figure}[!htb]
   \begin{minipage}{0.8\textwidth}
     \centering
     \includegraphics[width=\linewidth]{fig/bombyx4.png}
   \end{minipage}\hfill
   \begin{minipage}{0.15\textwidth}
     \centering
     \includegraphics[width=\linewidth]{fig/bombyx1.png}
   \end{minipage}\hfill
   \caption{Technical plans of the Bombyx 2 system, taken from OSEAN SAS manufacturing report. (left) Mooring system. (right) Acoustic recorder and floatability variation system (total height of 3 meters).}
\end{figure}

As part of the GIAS project aiming at reducing navigation risks in the Mediterranean sea, the Bombyx 2 buoy was designed, in a collaboration between Toulon University and OSEAN SAS. This buoy is equipped with 5 hydrophones, a floatability variation system, and embedded algorithms for the detection of sperm whale clicks and 20Hz fin whale pulses. To mitigate surface noise and exposure to strong weather conditions, the buoy parks at 25 meters depth to record and acoustically detect its target species (sperm whales and fin whales). In the event of a detection, the buoy reaches the surface to transmit the alert with supporting data via the mobile network. The alerts then enable nearby ferries to make decisions to mitigate their risk of collision (reducing the speed or changing route for instance).


\subsection{\ac{CNN} deployment to an embedded uController}

Section \ref{chap:lightweight} introduced low complexity \acp{CNN}, especially designed to answer the needs of this alert system. These models, after being trained on \acp{GPU} using the Pytorch package \cite{NEURIPS2019_9015}, had to be implemented on the embedded system, namely the Microchip PIC32 micro-controller (integrated on the High-Blue sound card \cite{barchasz2020novel}).

This demanded to build a custom interface to export and load architectures and weights. The exports are done in Python to text files, and imported in C (required programming language for the micro-controller). Design choices were made for the C implementation, for a compromise between flexibility and low development effort :
%
\begin{itemize} \setlength{\itemsep}{1pt}
    \item The model input consists in a Mel-spectrogram
    \item Signal length, sampling frequency, window length, hop size, number of Mel bands, and Mel frequency boundaries are parametric
    \item The architecture consists of successive depth-wise separable convolution layers intertwined with batch normalization and leaky \ac{ReLU}
    \item The number of layers, and the number of features, kernel sizes and strides for each layer are parametric
    \item The last layer is pooled by maximum to yield a global prediction of the signal
\end{itemize}


\subsection{Computation times}

Specifications of the input parameters and processing time for the two target signals are given in Tab. \ref{tab:pic_implem}.

\begin{table}[ht]
    \centering
    \begin{tabular}{l|c|c}
    Target signal & Sperm whale clicks & 20Hz fin whale pulse \\ \hline
    Signal length (sec) & 10 & 60 \\
    Sampling frequency (kHz) & 64 & 4 \\
    \ac{FFT} window length & 512 & 4096 \\
    \ac{FFT} hop size & 256 & 256 \\
    Mel bands & 64 & 64 \\
    Mel start (Hz) & 2,000 & 0 \\
    Mel end (Hz) & 25,000 & 100 \\
    Signal loading (sec) & 1 & 5 \\
    Spectrogram computation (sec) & 12 & 26 \\
    \ac{CNN} inference (sec) & 4 & 4 \\
    \end{tabular}
    \caption{Specifications and corresponding processing times on the PIC32 microprocessor, for the detection mechanisms of sperm whale clicks and 20Hz fin whale pulses.}
    \label{tab:pic_implem}
\end{table}

The longest step to process is by far the spectrogram computation compared to the \ac{CNN} inference. This comforts the choice of the Fourier transform which offers a fast \ac{FFT} implementation, rather than others such as the wavelet transforms.

% \subsection{Power budgeting}
% duty cycles
% \cite{rand2022effects, riera2013effects}


\subsection{Detection report}

In the event of detections triggered by the \acp{CNN}, the buoy is ordered to lift towards the surface to transmit a report supporting the alert. To lower the amount of data to be transmitted, chunks of signals are cut surrounding detection peaks and downsampled to lower sampling frequencies.

% TODO conclude : insight

% \section{Interpretation of model predictions}
% Check pred distribution
% Dolphin / humpbacks CARIMAM
% Carimap


\section{Long term presence monitoring}

In addition to its production use in the context of ship collision mitigation, the sperm whale click detection \ac{CNN} has been forwarded on the whole Bombyx dataset (3,532 recorded hours from May 2015 to December 2018) for a long term study of sperm whale presence. This work was conducted as a collaboration with Maxence Ferrari and Marion Poupard, and resulted in a journal publication \cite{bombyx}. Some of the results are reported here.


\subsection{Sperm whale acoustic presence}

A first analysis focused solely on reporting the presence of sperm whales through the recorder years. Files (1min long) with more than 40 \ac{CNN} predictions above 0.95 were manually analysed using the interface described in section \ref{chap:bombyx_annot}. Thus, automatic detections were validated and number of individuals were estimated (using simultaneous click trains and \ac{TDOA} tracks). This process yielded 57 new sperm whales passages (missed during the annotation procedure described in section \ref{chap:bombyx_annot}), and 25 false positives (including 15 triggered by sound card malfunctions). The notion of passage was used to account for sperm whale presence, considering that clicks belong to the same passage

In total, 226 sperm whale passages have been recovered, with a total of 347 individuals). Fig. \ref{fig:res_nbr_indiv_calendar} presents the number of detected individuals each day during the 4 years of recording, with white regions indicating the absence of recording. Sperm whales were found all year round, with no statistically significant seasonal pattern. The number of animals per passage varied from 1 to 9 individuals, with a mean duration of 4 hours.

To evaluate dial patterns, the probability of presence was computed for each hour of the day. Averaging probabilities into four periods (Night, Morning, Afternoon, and Evening) demonstrates a statistically significant differences among periods of the day : sperm whales are more present during morning or afternoons than in the evening (see Fig. \ref{fig:res_nbr_indiv_calendar}, Kruskal-Wallis test : p-value < 0.01).

\begin{figure}[ht]
\centering
\includegraphics[width=\linewidth]{fig/bombyx_results_calendar.pdf}
\caption{ Left (a): The Number of detected sperm whales per day during the 4 years of recordings (white region: \textit{no d = no data}). Right (b): Mean probability of presence for each period of the day.}
\label{fig:res_nbr_indiv_calendar}
\end{figure}


\subsection{Presence in relation to anthropogenic noise pressure}

To assess the performance of the detection system as well as to measure the impact of noise on the presence of sperm whales, the amplitudes of different octave bands were computed and analyzed. The distribution of the background noise (octave 800\,Hz) through the day is shown in Fig. \ref{fig:db_cach}. All octaves' dial distributions have the same shape as the 12,800\,Hz octave, with the energy peaking around 4am and 9pm.

\begin{figure}[ht]
    \centering
    \includegraphics[width=\linewidth]{fig/bombyx_results_noise.pdf}
    \caption{(left) Distribution of the amplitude for the octave 12,800\,Hz according to presence/absence of sperm whales. (right) Superposition of dial pattern of amplitudes for the octave 12,800\,Hz and probability of presence of sperm whales.}
\label{fig:db_cach}
\end{figure}

Ferries cross the study area daily, connecting Toulon or Marseille to Corsica, with scheduled times between 3am - 6am and from 8pm - 9pm. The closest ferry route is approximately 3km away from the antenna. For all octaves, dB amplitudes are significantly higher during ferry schedules (Mann–Whitney test, \textit{p-value} $<$ 0.05), with an average gain of approximately 3dB.

Moreover, as Fig. \ref{fig:db_cach} illustrates, the data shows a significantly lower noise during the sperm whales' presence (Mann-Whitney U=14.44, sample size=300, \textit{p-value} < 0.01) for all octaves except 6,400\,Hz and 12,800\,Hz. This is further demonstrated in Fig. \ref{fig:db_cach}, where, during 4 AM and 9 PM (noise peaks), the presence of sperm whales is lowest. This last figure also shows that the reduced sperm whale presence is not due to an increased background noise, since sperm whale probability drops before the background noise rises.

This study is a first demonstration of the versatility of the detection systems designed through this thesis. Indeed, they can be applied to a real-time alert system to mitigate collision risks, but also in long-term studies, revealing presence patterns that are crucial in the implementation of relevant conservation measures.


\section{Song structure analysis and temporal trends}

\subsection{Context and objective}

In parallel to the latter study, a similar one was conducted on fin whales of the Ligurian sea, again making use of the detection system designed for the GIAS buoy. The trained \ac{CNN} described in section \ref{chap:lightweight} was thus run over three available datasets : Boussole, Bombyx, and Antares (see section \ref{chap:data_Toulon}). This time, instead of presence monitoring, the study focused on fin whale song patterns, yet poorly documented in the Mediterranean sea. It is also subject to a journal publication \cite{finsong}, which results are reported here.

As in other cetacean species, fin whales show geographical acoustic differentiation \cite{weirathmueller2017spatial, helble2020fin, morano2012seasonal, castellote2012fin}, hypothesised to be cultural in some cases \cite{weirathmueller2017spatial}. The divergence of mysticetes songs in different populations is presumably a result of drifts emerging from the conformity and creativity constraints of song production \cite{payne2000progressively}. Moreover, the character displacement theory with songs serving as a discrimination marker for allopatric populations has been hypothised for fin whales of the Northern Atlantic \cite{delarue2009geographic}. As for the Mediterranean population, it has been shown to be resident and genetically dissociated from the North Atlantic population \cite{berube1998population}, and their songs (especially the \ac{IPI}) were shown to enable their identification \cite{castellote2012fin, pereira2020fin}. The Mediterranean fin whales do not follow strict migration patterns or reproduction periods unlike their oceanic conspecifics \cite{Notarbartolo}, so their song can be heard all year round.

The base unit of the songs, the 20hz pulse, is shared by all fin whales. These pulses occur in sequences that typically last several hours \cite{Watkins_Tyack_Moore_Bird_1987}, with highly regular pulse intervals between 10 and 40 seconds. The main differentiation of songs lies in the \ac{IPI} and pulse spectra \cite{Thompson, hatch2004acoustic}. Alike fin whales of the Pacific \cite{weirathmueller2017spatial, helble2020fin}, Mediterranean 20Hz pulses fall into 2 distinct types, one with a slightly higher frequency content than the other \cite{ClarkBorsani2002, sciacca2015annual} (see Fig. \ref{fig:spectro}). These two categories are sometimes labelled 20Hz pulse and back-beat, they will be referred to as type A and B for short, with A being the higher pitched pulse. Fin whales of the Pacific and Atlantic often exhibit sequences that alternate between A and B pulses. These are called doublet patterns, as opposed to singlets where only one of the pulse types occur. In doublets there is a strong relationship between \ac{IPI} and pulse type: two different \acp{IPI} are found, one from A to B, and another one from B to A \cite{Oleson, constaratas2021fin, Furumaki_Tsujii_Mitani_2021, morano2012seasonal, helble2020fin}. On the other hand, singlets also follow their own stereotypical \ac{IPI}.

Mediterranean fin whale songs present more diversity in the consecution of pulse types than simple singlets and doublets (see Figure \ref{fig:spectro}). Nonetheless, two studies present stereotypical \acp{IPI}. Based on recordings from 1999, \citet{ClarkBorsani2002} observe a link between pulse type and \ac{IPI} in the Mediterranean sea for two pulse sequences (about 100 pulses). About ten years later, \citet{castellote2012fin} observe a common \ac{IPI} around 14.9 sec for that same population, but do not mention its relationship with pulse types.

Besides geographical variations, fin whale song structures also exhibit temporal variations, such as seasonal \ac{IPI} increases \cite{Oleson, morano2012seasonal}, and inter-annual variations of \ac{IPI} and peak frequency \cite{weirathmueller2017spatial}. \ac{PAM} stations combined with automated analysis (template matching approach) have played a key role in revealing these long-term trends.

Until now, no large scale analysis has been conducted on Mediterranean fin whale songs that could reveal the long-term evolution of their vocal behaviour, thus the following study.


\subsection{Method}

\subsubsection{Model inference}

While the model was trained to detect pulse presence in 5-second segments, the convolutional stack is designed to maintain the temporal resolution of the predictions throughout the network. Discarding the max pooling layer at the end of the CNN, pulse times were retained as the highest predictions above a given threshold within sliding 4 second windows. These timings are approximate up to the size of the receptive field of the network (0.8 seconds).

Thresholds were set at the balance point of the \ac{ROC} curves (equal sensitivity and specificity). This setting lead to sensitivities and specificities of 0.96 and 0.97 for the Bombyx and Boussole data respectively. For the Antares data, since the \ac{ROC} curve is unknown (no annotation were available), a threshold of 0.12 was chosen so that there is approximately the same proportion of detections as in Bombyx and Boussole ($\approx$ 0.5\%) .

Tab. \ref{tab:rorqual_recap} summarises the resulting detections, along with a calendar Fig. \ref{fig:calendar}. Following \citet{Watkins_Tyack_Moore_Bird_1987}, pulses at a distance of less than 45 seconds were considered as being part of the same sequence, and sequences less than 2 hours apart were considered as being part of the same bout.

\begin{table}[ht]
\centering\resizebox{\linewidth}{!}{
\begin{tabular}{|l|c|c|c|c|}
\hline
 \textbf{Data source} & \textbf{Boussole} \cite{parsuivi} & \textbf{Bombyx} \cite{vamos2017} & \textbf{Antares} \cite{giacomelli2009antares} & \textbf{Total} \\ \hline
\textbf{Location} & South of Sanremo & Port-Cros Island & Cap Sicié & \\ \hline
\textbf{Recording year} & 2008-2009 & 2015-2018 & 2020-2021 & \\ \hline
\textbf{Recorded time (hours)} & 1,860 & 3,291 & 1,124 & 6,275 \\ \hline
\textbf{Detection threshold} &  0.15 & 0.68 & 0.12 & \\ \hline
\textbf{Detected pulses} & 1,647 & 2,827 & 657 & 5,131 \\ \hline
\textbf{Detected A pulses} & 1,411 & 2,554 & 322 & 4,287 \\ \hline
\textbf{Detected B pulses} & 236 & 273 & 335 & 844 \\ \hline
\textbf{Detected sequences} & 246 & 615 & 58 & 919 \\ \hline
\textbf{Detected bouts} & 51 & 214 & 11 & 276 \\ \hline
\end{tabular}}
\caption{\label{tab:rorqual_recap}Summary of recording characteristics and automatic detections for each source of data.}
\end{table}

\begin{figure}
    \centering
    \includegraphics[width=.75\linewidth]{fig/rorqual_calendar.pdf}
    \caption{Number of detected sequences for each day with recordings, normalized by the amount of recorded hours. Grey cells denote days with recordings but no detection.}
    \label{fig:calendar}
\end{figure}


\subsubsection{Spectro-temporal pulse analysis} \label{chap:analysis}

Following the detection process, a signal processing analysis was conducted to precisely describe each pulse (exact time position, center frequency, bandwidth and \ac{SNR}). This yields the necessary data to search for song patterns, as shown in Figure \ref{fig:spectro}.

\begin{figure}[ht]
 \centering
 \includegraphics[width=1\linewidth]{fig/spectrogram.pdf}
 \caption{Spectrogram of a fin whale pulse sequence recorded by the Bombyx buoy in October 2018. Spectrogram parameters are described in section \ref{chap:analysis}. Dots show the center frequencies of the detected pulses, with white dashed lines showing \acp{IPI}. The grey dashed line denotes the discrimination threshold between type A and B pulses, at 19.88 Hz.}\label{fig:spectro}
\end{figure}

For this analysis, an 8 sec window surrounding the prediction peak is selected (\(T=[0,8]\)), band-pass filtered (Butterworth of order 3 between 10Hz and 30Hz), and resampled at 250Hz. The \ac{STFT} is then applied to the resulting signal (Hann window of 1024 including 75\% of zero padding and 97\% overlap) resulting in spectral and temporal resolutions of 0.24Hz and 0.03sec respectively.

From this spectrogram, the precise time position of the pulse \(\hat{t}\) is first estimated by selecting the column of the maximum value in the 18-22Hz frequency band (Eq.\ref{eq:time}). This value will be kept for the later \ac{IPI} measurements. 

\begin{equation}
\hat{t} = \argmax_{t \in T}\left( \max_{f \in [18,22]}\left( \mathbf{S}(f, t)\right)\right)
\label{eq:time}
\end{equation}

To measure the spectral envelope of the pulse, a 1.2 sec window around $\hat{t}$ is max-pooled time wise. Background components are withdrawn (to focus on the pulse spectra only) by subtracting an estimate of the background spectrum: the median of each frequency bin within the window $T$ (Eq. \ref{eq:spec}). Doing so, effects such as the impact of \ac{SNR} on peak frequency and bandwidth (observed by \citet{helble2020fin}) are mitigated.

\begin{equation}
E(f) = \max_{t \in [\hat{t}-0.6, \hat{t}+0.6]} \left(\mathbf{S}(f, t)\ \right) - \   \underset{t \in T}{\mathrm{median}}\ \left( \mathbf{S}(f, t)\right)
\label{eq:spec}
\end{equation}

The resulting pulse envelope is used to compute the left and right boundaries of the pulse spectrum, with $\max_{f}\frac{E(f))}{4}$ as a threshold (equivalent to -6 dB). Left and right intersection frequencies are linearly interpolated to increase the precision of the estimate. This process yields the 6 dB bandwidth (width between the boundaries), and the center frequency (mid-point between the boundaries) of the analysed pulse.

\begin{figure}[ht]
     \centering
     \includegraphics[width=.75\linewidth]{fig/centroid.pdf}
     \caption{Histogram of the center frequencies of the detected pulses. Black lines denote the fitted \ac{GMM}.}\label{fig:centroid}
\end{figure}
 

For later filtering by pulse quality, the \ac{SNR} was computed following Eq. \ref{eq:bckgnd} (pulse energy as the maximum of its envelope and background energy as the median of the spectrogram surrounding the pulse).

\begin{equation}
E_{Background} = \underset{T \setminus [\hat{t}-1,\hat{t}+3]}{\underset{\ f \in [15, 25]}{\mathrm{median}}}\ \mathbf{S}(f, t); \;
E_{Pulse} = \max_{f}E(f); \\
SNR = 10\log_{10}\left(\frac{E_{Pulse}}{E_{Background}}\right)
\label{eq:bckgnd}
\end{equation}

The pulse spectral characteristics of mysticetes are often described using the frequency of maximum energy (peak frequency) or the spectrum weighted mean (centroid frequency) \cite{weirathmueller2017spatial, malige2020inter}. Here, the center frequency was chosen, as it appeared to be better suited for the discrimination between the two pulse types. In fact, when modeling the distribution of peak frequencies using a Gaussian mixture model, the two components (emerging from the two types of pulses) overlap more than when using center frequencies (the Kullback-Leibler divergence between the Gaussian components in center frequency is significantly higher than that of peak frequencies, 113 nats and 30 nats respectively).


\subsubsection{Pre-analysis filtering}

To filter out false positives, only pulses with a bandwidth below 10Hz and a center frequency within \([18.5,22.5]\) were retained. Besides, only sequences with a mean \ac{SNR} of at least 8 dB, and with at least 3 pulses were kept for the following analysis. Sequences containing \acp{IPI} below 10sec or above 45sec were discarded as well. The resulting number of registered pulses and sequences are shown in a calendar Fig. \ref{fig:calendar} and in Tab. \ref{tab:rorqual_recap}.

To classify between A and B types, a two component \ac{GMM} was fitted on the center frequency data (see Figure \ref{fig:centroid}) using the \ac{EM} algorithm. This lead to a threshold of 19.88 Hz to discriminate between the two types. Even though the center frequency is found to evolve over time, the change is sufficiently small to not interfere with categorisation (see Fig. \ref{fig:centScat}).


\subsection{Results}\label{chap:rorq_results}

\subsubsection{Stereotypical \ac{IPI}}

The time between a pair of consecutive pulses in a sequence \ac{IPI} appears to be strongly determined by their type (see Fig. \ref{fig:histIPI}). The typical interval for an `AB' bi-gram is 2sec longer than that of `AA' or `BA'. On the other hand, the 'BB' pairs (less frequent but still commonly found) are 11sec longer on average, but present larger variability than the others.

\begin{figure}[ht]
     \centering
     \includegraphics[width=.75\linewidth]{fig/IPI_scatter.pdf}
     \caption{Scatter plot of the most frequent \ac{IPI} per month for each type sequence. Fitted linear models are shown as grey dashed lines. Points extracted from \citet{ClarkBorsani2002} and \citet{castellote2012fin} appear as crosses.}\label{fig:scattIPI}
\end{figure}

Figure \ref{fig:scattIPI} shows how these intervals have changed over the course of two decades, following an approach similar to \citet{weirathmueller2017spatial}. For each month and pulse type pair, points denote the most frequent \ac{IPI} (quantized with a resolution of 0.1sec). For months containing more than 100 bi-gram occurrences, the most frequent \ac{IPI} was retained if representing at least 5\% of it. Points measured in previous studies were also added : in 1999 by \citet{ClarkBorsani2002} (the only study to our knowledge that references \ac{IPI} depending on type sequence in the Mediterranean sea), and in 2008 by \citet{castellote2012fin} (assuming it describes the most common pair `AA', as it was not specified). The `BB' sequence did not provide enough occurrences for the statistical tests to be relevant. For sequences `AA', `AB', and `BA', fitted linear models are plotted (their coefficients of determination are 0.83, 0.89, and 0.91 respectively). The p-values for the null-hypothesis that the slopes are not significantly different from 0 are all inferior to 0.01. The estimated slopes for the `AA', `AB', and `BA' bi-grams are 84, 83, and 88 respectively (in milliseconds/year). 

\begin{figure}[ht]
     \centering
     \includegraphics[width=.75\linewidth]{fig/IPI_hist.pdf}
     \caption{Histogram of the \ac{IPI} for each type sequence (bi-gram).}\label{fig:histIPI}
\end{figure}
 
 
\subsubsection{Center frequency}\label{seq:freq}

In a similar fashion, temporal trends of A pulses spectral characteristics were analysed. This revealed an intra-annual decrease in pulse center frequency between the months of August and May (Fig. \ref{fig:centScat}). On the other hand, no inter-annual shift was observed (Pearson yields a correlation coefficient of -0.06 between pulse absolute dates and their center frequency).

\begin{figure}[ht]
    \centering
    \includegraphics[width=.75\linewidth]{fig/centroid_hist2d.pdf}
     \caption{Bi-histogram of the center frequencies against months of the year. The horizontal line shows the separation between type A and type B pulses. The fitted linear model is shown as a black dashed line.}\label{fig:centScat}
\end{figure}

For this statistical analysis, center frequencies were quantized to a resolution of 0.1Hz and grouped by months. Center frequencies with the most occurrences were kept, if among groups (months) of at least 50 pulses. Fitting a linear model on the retained points yields a coefficient of determination of 0.73, with an estimated slope of -0.08 Hz/month) (for the null-hypothesis that the slope is not significantly different from 0, p-value < 0.01).

For comparison with other previous studies, the same analysis was ran using peak and centroid frequencies. The slope of the observed intra-annual trends are similar for all metrics (-0.09Hz/month, -0.08Hz/month, and -0.11Hz/month for peak, center, and centroid frequencies respectively) and p-values for the null-hypothesis that the slope is not different from 0 are all below 0.01.

Besides the intra-annual trend,


\subsubsection{Correlation between center frequency and \ac{IPI}}

With the observation of synchronous inter-annual shifts of both \ac{IPI} and center frequencies (Pacific fin whales), the hypothesis of a link between the two arises. \citet{weirathmueller2017spatial} states that the augmentation of the \ac{IPI} through the years could be explained by the simultaneous decrease in pulse peak frequencies (lower frequency pulses presumably requiring a bigger effort to produce, a bigger gap between them could be needed). The observed stereotypical \acp{IPI} of Mediterranean fin whales also support this idea (`AA' showing the shortest \ac{IPI} on average). This hypothesis was thus further tested by analysing the correlation between \ac{IPI} and center frequency (for pulses with \acp{IPI} between 14 and 20 seconds).

To dissociate this analysis from the link between pulse types and \ac{IPI}, 3 component Gaussian mixture model was fitted on the bi-dimensional representation of pulses (center frequency versus time until the next pulse). This enabled to group the different pulse bi-grams (`AA', `AB', and `BA'), and conduct a correlation analysis on each group independently. Figure \ref{fig:centroidIPI} shows the scatter plot of the pulses with their assignation to each mixture component. For each of the latter, the Pearson correlation coefficient was computed, yielding -0.37, -0.22, and -0.35 for `BA', `AB', and `AA' respectively (all p-values are below 0.01).

\begin{figure}[ht]
    \centering
     \includegraphics[width=.75\linewidth]{fig/centroid_vs_IPI.pdf}
     \caption{Scatter plot of pulses center frequency against the time until the next pulse (\ac{IPI}). Colors denote the \ac{GMM} assignations, whose means are marked with crosses.}\label{fig:centroidIPI}
\end{figure}


\subsection{Discussion}
% TODO \cite{rice2022update}

\begin{table}[ht]
\centering\resizebox{\linewidth}{!}{
    \begin{tabular}{|c|c|c|c|c|c|}
    \hline
    & & \multicolumn{2}{c|}{\textbf{Inter-annual}} & \multicolumn{2}{c|}{\textbf{Intra-annual}}  \\ \hline
     \textbf{Study} & \textbf{Location} & \textbf{Frequency} & \textbf{IPI} & \textbf{Frequency} & \textbf{IPI} \\ \hline
    Weirathmueller et al. \cite{weirathmueller2017spatial} & N.E. Pacific & -0.17 Hz/yr& 0.5-0.9 sec/yr & - & - \\ \hline
    \citet{Oleson} & N Pacific & - & - & - & +7.5 sec \\ \hline
    \citet{leroy2018long} & Indian & -0.21 Hz/yr & - & $\sim$ -0.1 Hz/mth & - \\ \hline
    \citet{helble2020fin} & N. Pacific & - & 0.6-1.3 sec/yr & - & - \\ \hline
    \citet{morano2012seasonal} & N.W. Atlantic & - & * 0.5 sec/yr & - & +5.5sec \\ \hline
    \citet{Watkins_Tyack_Moore_Bird_1987} & N.W. Atlantic & - & - & - & +6sec \\ \hline
    \citet{vsirovic2017fin} & Gulf of California & - & $\sim$ 1sec/yr & - & $\sim$ +8sec  \\ \hline
    \citet{furumaki2021fin} & Chukchi sea & - & $\sim$ 0.5 sec/yr & - & $\sim$ 1 sec/yr \\ \hline
    \citet{wood2022characterization} & W. Antarctic & -0.2 Hz/yr & 0.1 sec/yr & - & - \\ \hline
    \textbf{self} & W. Mediterranean & - & 0.1 sec/yr & -0.1 Hz/mth & - \\ \hline
    \end{tabular}}
    \caption{Summary of song pattern trend studies. For intra-annual \ac{IPI} shifts, since trends are not linear, we report the difference between low \ac{IPI} season and high \ac{IPI} season (summer vs winter). The inter-annual \ac{IPI} shift for \citet{morano2012seasonal} (see `*') is reported between two consecutive years only.}
    \label{tab:summTrends}
\end{table}


\subsubsection{Mediterranean sea stereotypical \acp{IPI}}

The present study led to the confirmation of the local stereotypical \acp{IPI} being determined by pulse type sequences. These results were previously shown on relatively a small corpora of around 100 pulses \cite{ClarkBorsani2002}, they are hereby confirmed with a corpus larger by an order of magnitude, and over a span of 10 years.


\subsubsection{\ac{IPI} trends}
% IPI inter year
Furthermore, these stereotypical \acp{IPI} are shown to evolve over the years, following a linear growth of approximately 0.08 sec/year over the past 20 years. As shown Tab. \ref{tab:summTrends} shows, such trends had already been observed with fin whales of the North-East Pacific \cite{weirathmueller2017spatial}, and with fin whales of the Central-North Pacific \cite{helble2020fin}.

Inter-annual shifts in \ac{IPI} are rather recent and poorly documented. \citet{weirathmueller2017spatial} state that the increasing \ac{IPI} might be linked to the downward frequency shift, lower frequency pulses potentially being more demanding in energy. A low correlation coefficient was measured between the two variables, and the data did not show any evidence for an inter-annual center frequency decrease. These observations thus go against this hypothesis, but more data is required to draw firm conclusions.

As for the \ac{IPI} shift slopes, it seems plausible that the differences between Pacific and Mediterranean populations arise culturally. Whether they are originally caused by the same factors or not, the singing patterns drift independently, with song conformity only taking place within a given population. If environmental or physiological factors alone were responsible for such patterns, they would have to be present both in the Pacific and in the Mediterranean sea, but operating at different rates. The hypothesis of the post-whaling population recovery (increasing density and animal sizes) explaining those trends suits the latter conditions, as recovery rates could differ between Mediterranean and Pacific waters. On the other hand, cultural features such as contact rate between individuals could explain slope differences as well, regardless of the root cause of the shift.

% IPI intra year
On the other hand, studies of Atlantic and Pacific fin whales \cite{morano2012seasonal,Watkins_Tyack_Moore_Bird_1987, Oleson, weirathmueller2017spatial} point to \ac{IPI} increases during winter, before dropping back to autumn values. These trends are hypothesised to be directly linked to the reproductive season \cite{Oleson} (due to hormonal activity or progressive dilution of the competition for instance). No such trend was observed in the present data, but the irregular data sampling through seasons might create an observational bias in that sense.


\subsubsection{Pulse frequency trends}
% inter annual
Inter-annual shifts in vocalization frequencies were already documented in blue whales \cite{mcdonald2009worldwide, malige2020inter, rice2022update}, and bowhead whales \cite{thode2017decadal}. Fin whales also showed similar trends in the Pacific \cite{weirathmueller2017spatial} (for 20Hz pulses, -0.17 Hz/year) and in the Indian Ocean \cite{leroy2018long} (for 99Hz pulses, -0.21 Hz/year). Numerous hypotheses have been formulated for the cause of this phenomenon, such as the increase in population density or body sizes (following the cessation of commercial whaling), the increase in calling depth \cite{gavrilov2012steady}, the augmentation of noise from melting icebergs \cite{leroy2018long}, or the acidification of the oceans affecting sound propagation \cite{hester2008unanticipated} (among others).

% centroid intra year
No inter-annual frequency shift was found in the analysed data. Mediterranean fin whales could thus be an exception to this widespread trend. 
Nonetheless, an intra-annual decrease in center frequencies was observed (-0.10 Hz/month). Such phenomenon was previously observed in large mysticetes of the Indian Ocean including fin whales \cite{leroy2018long}. The latter study hypothesised pulse frequencies to follow seasonal ambient noise level variations (notably due to melting ice). Such phenomenon does not apply to the Mediterranean sea.


inter annual IPI increase add :
\cite{wood2022characterization, furumaki2021fin} 

Blue whale freq decrease \cite{rice2022update}

[optional]
ICI sperm whales bombyx

\section{Language modelling}
Application to Orcalab

\begin{table}[ht]
    \centering
    \begin{tabular}{|c|c|c|c|c|c|c|c|c|c|} \hline
 & N1 & N2 & N23 & N3 & N4 & N5 & N9 & other & Total \\ \hline 
 N1 &0.21 & 0.03 & 0.06 & 0.01 & 0.24 & 0.1 & 0.12 & 0.09 & 15,551 \\ \hline 
 N2 &0.07 & 0.15 & 0.06 & 0.01 & 0.23 & 0.1 & 0.11 & 0.1 & 6,446 \\ \hline 
 N23 &0.07 & 0.03 & 0.23 & 0.01 & 0.23 & 0.05 & 0.1 & 0.1 & 15,752 \\ \hline 
 N3 &0.04 & 0.02 & 0.04 & 0.39 & 0.17 & 0.05 & 0.09 & 0.05 & 5,187 \\ \hline 
 N4 &0.05 & 0.02 & 0.04 & 0.01 & 0.54 & 0.06 & 0.11 & 0.06 & 82,625 \\ \hline 
 N5 &0.07 & 0.03 & 0.04 & 0.01 & 0.27 & 0.22 & 0.14 & 0.08 & 17,722 \\ \hline 
 N9 &0.06 & 0.02 & 0.05 & 0.01 & 0.31 & 0.08 & 0.27 & 0.06 & 29,808 \\ \hline 
 other &0.05 & 0.02 & 0.05 & 0.01 & 0.18 & 0.05 & 0.07 & 0.1 & 27,081 \\ \hline
    \end{tabular}
    \caption{Caption}
    \label{tab:my_label}
\end{table}

\cite{ford1989acoustic}