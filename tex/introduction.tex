% intro.tex:

\chapter{Introduction}

\minitoc
\section{Cetaceans and acoustics}

Despite a first appearance on land, some mammalian species are now commonly found in planet Earth's oceans, forming the marine mammals group. Families evolving from 4 distinct mammalian orders have evolved to thrive in the marine environment : Humans (\textit{Primates}), Pinnipeds (\textit{Carnivora}, e.g. seals or walruses), Sirenians (\textit{Afrotheria}, e.g. manatees), and Cetaceans (\textit{Cetartiodactyla}, e.g. whales and dolphins). The following thesis will focus on the latter. Cetaceans are classified into two suborders (see Fig. \ref{fig:evolution}), namely Odontocetes (toothed cetaceans, such as dolphins, orcas or sperm whales) and Mysticetes (baleen cetaceans, such as blue whales or humpback whales). 

\begin{figure}[!htb]
   \begin{minipage}{0.5\linewidth}
     \centering
     \includegraphics[width=\linewidth]{fig/evolution.png}
   \end{minipage}\hfill
   \begin{minipage}{0.5\linewidth}
     \centering
     \includegraphics[width=\linewidth]{fig/cetaceans.png}
   \end{minipage}\hfill
   \caption{(left) Evolution of the marine mammals. (right) Evolution of the cetacean evolutionary relationships (top : Odontocetes, bottom : Mysticetes). Both figures are taken from \citet{whitehead2015cultural}.}\label{fig:evolution}
\end{figure}

Leaving land some 50 million years ago \cite{whitehead2015cultural}, cetaceans now show a complete adaptation to their marine environment, with their powerful flukes, streamlined body, and nostrils displaced on top of their head (allowing for efficient breathing while swimming). Another important adaptation, especially relevant to this study, is the development of their acoustic capabilities, both in emitting and receiving sounds. Indeed, light typically fades out after a few dozen meters in water, which makes of vision a quite limited sense. In contrast, the higher density of water (compared to air) makes sound travel faster and further. Cetaceans make use of this property to communicate and/or echolocate up to great distances. Blue whale calls can be heard 200km away \cite{vsirovic2007blue}, and sperm whales are able to detect a 1m object at 470m \cite{ferrari2020study}.


\subsection{Echolocation}

One of the uses cetaceans make of underwater acoustics is echolocation. Alike an active sonar, emitting a sound and measuring how it bounces back to you (its echo) allows to sense distance from surrounding objects, their shape \cite{pack1996dolphins}, or even their texture \cite{grunwald2004classification}. Bats use echolocation to navigate and hunt in dark caves, odontocetes use echolocation in a similar way underwater. 

Short impulse like sounds, commonly named `clicks' (transitory waves) are mostly associated with echolocation purposes \cite{au2000echolocation}. However, there is not one single type of click used for echolocation, and it coincides with habitats and feeding behaviours \cite{ketten1994functional}. Using short duration clicks, more can be sent in a small period of time without them mixing up, thus increasing the potential temporal resolution of the echolocation. This is typically suited for hunting at high speeds, like small odontocetes do. On the other hand, clicks at lower frequencies will travel further, and thus would be more suited for hunting from long distances like sperm whales do (extremely high Kogia clicks go against this hypothesis).

Finally, despite the old consensus that only odontocetes echolocate with their high frequency clicks, new studies suggest that mysticetes might also make use of their low frequency signals as sonars \cite{frazer2000sonar}.

\begin{figure}
    \centering
    \includegraphics[width=.7\linewidth]{fig/echolocation.pdf}
    \caption{Illustration of the dolphin echolocation mechanism for hunting purposes (image credit : Uko Gorter - American Cetacean Society).}
    \label{fig:my_label}
\end{figure}


\subsection{Communication}

The second major use of sound by cetaceans is communication, a broad concept that can be divided into two main categories : song and social communication systems \cite{janik2014cetacean}.

The term song has been first used for cetacean signals by \citet{payne1971songs}, listening to humpback whales whose vocalizations met the following definition : "a series of notes, generally of more than one type, uttered in succession and so related as to form a recognizable sequence or pattern in time”. As for birds, these songs have shown a role in reproductive behaviors \cite{darling2001interactions, smith2008songs}. Mostly males are observed singing, during the reproductive season, potentially to attract females, fend off other males, or a combination of both. Songs usually come in strictly patterned sequences, shared by whole species or communities. Among cetaceans, they have yet been observed only in mysticetes, with the most renowned one probably being the humpback whale song.

On the other hand, communication is also observed in odontocetes social groups. Alike in songs, these signals are patterned vocalizations, which have been identified in discrete categories \cite{ford1987catalogue, weilgart1993coda}. However, they are not restricted to reproductive contexts, and appear in relatively less deterministic sequences. The term song therefore seems less appropriate for this phenomenon, which is rather associated with social bonding functions \cite{schulz2008overlapping, ford1989acoustic}.
% vocal learning ? dolphins learning sentences\cite{herman1984comprehension}.

In most cases, these vocal signals occur with tonal, whistled or pulsed calls. Their associated categories (`call type') are commonly defined by characteristics on their time / frequency contour. As an exception, sperm whales produce clicks in stereotyped rhythmic sequences (named codas) which were also attributed to communication purposes \cite{weilgart1993coda}. It is however not excluded that other odontocetes use clicks as means of communication, but no similar stereotyped sequences have yet been observed among them.


\subsection{Culture}

The term culture is often encountered when describing cetacean communication systems \cite{filatova2015cultural, garland2020cultural, rendell2003vocal}. It seems appropriate to describe the vocal divergences observed between cetacean communities. In a broad sense, culture is defined as `behavior or information shared within a community, that is acquired from conspecifics through some form of social learning' \cite{whitehead2015cultural}. %Culture is especially relevant as a community marker, helping to identify close relatives within the species \cite{todo}.
In cetaceans, it takes form as specialisation in diets or hunting techniques (e.g. with orcas) or as specific vocal patterns. For instance, sperm whale codas \cite{rendell2003vocal} (Fig. \ref{fig:codas}), orca stereotyped calls \cite{deecke2000dialect}, humpback whale songs \cite{garland2011dynamic} (Fig. \ref{fig:HB_culture}) or fin whale pulse sequences \cite{castellote2012fin}, are all community specific, some evolving through the years, and thus are described as cultural phenomenons. This is only possible thanks to the vocal production learning capacity that cetaceans demonstrate \cite{janik2014cetacean}, a relatively rare characteristic among mammals.

\begin{figure}[!htb]
   \begin{minipage}{0.5\linewidth}
     \centering
     \includegraphics[width=\linewidth]{fig/codas_map.jpeg}
   \end{minipage}\hfill
   \begin{minipage}{0.5\linewidth}
     \centering
     \includegraphics[width=.6\linewidth]{fig/codas.jpeg}
   \end{minipage}\hfill
   \caption{(left) Location of two sperm whale communities. (right) Cultural differences in codas patterns (dialects) between the two communities. Figures are taken from \citet{amano2014differences}}\label{fig:codas}
\end{figure}


\subsection{Human activity impacts}

In the twentieth century, over 725,000 fin whales were caught by whalers \cite{rocha2014emptying}. Seeing some whale species coming close to extinction has motivated a large majority of the international community to cease commercial whaling in the late 20th century. However, cetaceans are still heavily impacted by human marine activities in numerous ways (Fig. \ref{fig:human_impact}. We will focus here on the ones related to acoustics.

There exist a wide variety of anthropogenic acoustic disturbances in the marine environment, which has triggered the development of a new fields of research focusing solely on ambient noise levels \cite{merchant2012averaging}. Marine traffic, seismic surveys using airguns (often to search for oil patches), pile driving (for marine constructions such as offshore wind turbines), military sonars and explosive tests are the most widespread, with several potential consequences.

We hear better in a silent environment. This implies the first consequence of acoustic disturbances : acoustic masking. With increasing ambient noise levels, the hearing capacities of cetaceans decrease, thus impacting their communication, hunting, and navigation \cite{erbe2016communication}. More generally, dense marine traffic has also been shown to cause stress to some cetaceans species \cite{rolland2012evidence}. 

The second main consequence is acoustic impairment : temporary or permanent injuries of the hearing apparatus. Powerful sounds e.g. from airguns or military tests have been shown to cause deafness in some cetaceans, sometimes leading to mass strandings \cite{parsons2017impacts}.

Eventually, arising from a dense marine traffic, presumably combined with disorientation due to acoustic masking, the collision problem has also gained attention in the cetacean preservation community. Especially affecting large mysticetes (e.g. fin whales or right whales), records of death from collisions with boats show a significant impact on whale populations \cite{rockwood2017high}, motivating measures to mitigate collision risks.

\begin{figure}
    \centering
    \includegraphics[width=.8\linewidth]{fig/human_impact.pdf}
    \caption{Evolution of the worldwide sperm (top) and fin (bottom) whale populations and the main human-induced direct mortality threats. The threats are expressed in relative value. This figure is taken from \citet{sebe2020interdisciplinary}.}\label{fig:human_impact}
\end{figure}


\section{Passive Acoustic Monitoring of cetaceans}

To reveal the aforementioned complexity and diversity of cetacean's uses of acoustics, scientists have also put forward the hearing sense. \ac{PAM} is a field of bioacoustic studies that combines several scientific and technical domains, from electronics for recording hardware, to signal processing and statistical analysis. The term passive refers to the notion of listening from a distance, without interfering with the animals, as opposed to active sonar systems or attaching acoustic tags to the animals. The analysis of the acoustic activity of cetaceans can yield significant insights on their behavior, population dynamics, social structures, or even physiology.


\subsection{Comparison acoustic / visual surveys}

Besides PAM, visual surveys is the second main approach to the biological study of cetaceans. Each comes with its pros and cons. The acoustic approach enables low-cost, long term surveys : placing a fixed antenna allows to monitor biological activities for several consecutive months, requiring human intervention only for the installation and extraction of the recording system. In contrast, the visual approach demands a continuous human implication throughout the survey, in the relatively inaccessible marine environment.

In terms of detection capacities, cetaceans can be heard from great distances (up to 200km for the blue whale \cite{vsirovic2007blue}), even during deep dives, while they can be visually detected from relatively short distances (depending on weather conditions) only when surfacing, which for some species is only a small proportion of their time \cite{watwood2006deep}. However, species had first to be classified visually before we could learn on their associated acoustic behavior, and photo identification is still to this day the only reliable way to recognize individuals. Moreover, the observation of group sizes, behavior, and body conditions still mostly rely on vision. The two approaches thus really are complementary.


\subsection{Antenna types}

\ac{PAM} starts by placing hydrophone(s) (underwater microphones) to listen or record the acoustic environment. They can be fixed on the sea floor (bottom mounted), to a buoy (sonobuoy), to a cable towed by a boat (towed array), or directly to the hull of a boat or \ac{ASV}. When recording with multiple synchronized hydrophones, one can also triangulate (infer the position of) sound sources by measuring their \acp{TDOA}. The types of recording devices, their placement in the water column, and their layout between each other have crucial impacts on the yielded recordings, facilitating or not the following signal processing analysis.

Thus, when implementing recording systems, one has to make compromises. Recorders are typically limited by their available battery power and data storage capacity. On the other hand, increasing the number of channels, the sampling frequency, and/or the byte depth of the recorded signals yields a more detailed view of the acoustic scene, but also consumes the available resources at a faster rate.

%\subsection{Localisation}
%[optional]
%TDOA estimation, triangulation, antenna range ...


\subsection{PAM for biological studies}

The first step of the analysis of acoustic signals typically comes down to the detection and classification of cetacean vocalizations. The amount of detection through time in long term surveys already provides significant information on the animals' lives. From these, one can infer population density \cite{thomas2012passive} and seasonal or dial presence patterns \cite{bombyx}. When combining several antennas, these statistics can also be spacialised. 

The analysis of the detected signals can then bring further knowledge on the recorded animals, such as community membership, current behavior (hunting, socializing, courting), and individual characteristics (sexual maturity, body size for sperm whales \cite{bombyx}). These measures can themselves be put in a space-time perspective, potentially revealing patterns. In this way, \ac{PAM} becomes useful to cetacean behavioral biology and stock structure assessment.

A second field \ac{PAM} provides to is the study of animal communication systems. Cetaceans indeed represent a significant part among the vocal learning species (along with birds, bats, seals, elephants, mice and primates). Identifying patterned sequences and associating them with species, communities and/or behaviors yields exemplary data on the development of vocal interaction in the animal kingdom \cite{fitch2006biology}. Moreover, acoustic behavior studies revealing cultural differences has provided knowledge on population dynamics \cite{owen2019migratory} (Fig. \ref{fig:HB_culture}) and social structures \cite{gero2016individual}. There is therefore a great diversity of biological wonders that \ac{PAM} contributes to unveil.
%Again, \ac{PAM} has provided exemplary data on phenomenons such as cultural evolution and transmission, helping advance on these open research questions.

\begin{figure}
    \centering
    \includegraphics{fig/humpback_song.jpg}
    \caption{Example of passive acoustic monitoring findings : long term cultural transmission of humpback whale songs eastward through the Southern Pacific Ocean (each color represents identified song types). Taken from \citet{garland2011dynamic}.}
    \label{fig:HB_culture}
\end{figure}


\subsection{Cetacean conservation}

Some may question the amount of effort put into cetacean biology studies, considering that knowledge of nature is not in itself a sufficient driver. In that regard, it is to be kept in mind that cetaceans occupy the top of the ocean's food web, and therefore are significant regulators of their ecosystem as a whole. Moreover, the oceanic ecosystem is not only an important provider of food to humans, but also crucial to breathe (it is responsible for around 70\% of the atmosphere's oxygen production \cite{harris2012phytoplankton}). This field of study thus matters not only for the knowledge of planet earth's animal kingdom, but simply to our long term survival.

\begin{figure}
    \centering
    \includegraphics[width=.8\linewidth]{fig/speed_canada.png}
    \caption{Example of conservation measure in the Gulf of St. Lawrence in Canada \cite{speed_canada}. Reduced speed zones are put in place all year round (red) and seasonally (green) to protect North Atlantic Right Whales.}\label{fig:speed_canada}
\end{figure}

As stated previously, human activities heavily impact cetacean species, putting some of them close to extinction \cite{kraus2016recent}. Therefore, it seems relevant that we learn how to mitigate this impact and work on cetacean conservation policies. Some regulation measures have already been put in place, e.g. the Marine Mammal Protection Act \cite{fisheries2013marine}, speed regulations \cite{fisheries2012environmental} (Fig. \ref{fig:speed_canada}), and the definition of marine mammal sanctuaries \cite{notarbartolo2008pelagos}. Monitoring the efficiency and/or need for regulations, as well as maximising their relevance (e.g. habitats and/or seasons of importance) can only be done via the knowledge of the animals, thus justifying their study.


\section{Neural Networks and PAM}

\subsection{Automated PAM before Neural Networks}

To carry out the aforementioned long term cetacean surveys, acoustic detections are needed. This process can be done by manually inspecting signals, especially their time / frequency representation (spectrograms). However, this is very costly in human efforts, which motivated the development of automatic detection mechanisms. With such systems in hand, researchers can seamlessly process months of data to yield results such as spatio-temporal presence statistics.

The development of detection systems has long been done with handcrafted algorithms \cite{gillespie2008pamguard}. They can be sufficient for some use cases, but often come quite limited, as the variety of sounds to detect and potential noises increase. Analysing long streams of data across recording devices and antenna locations demands highly robust detection systems, for which handcrafted algorithms remain unsatisfactory.

As an analogy, let us consider our ability to recognize our kin by the sound of their voice. Formally describing how to differentiate talking individuals seems nearly impossible, especially in a computer language. However, we know that given a hearing sense and sufficient cognitive capacities, by listening to a voice several times, we acquire the capacity to recognize it. This led the scientific community to start shifting towards machine learning algorithms, which are introduced in the following section.


\subsection{Artificial neural networks}

Training \acp{ANN} is the chosen approach to tackle \ac{PAM} problematics throughout this thesis. It is one of the most popular techniques of machine learning, a field of computer sciences that approaches problem solving without programming solutions explicitly. Specifically, in machine learning, the algorithm is designed to find (or learn) the optimum solution to a problem, often formulated as a mathematical framework. An analogy could be made that genes encode a brain structure for it to learn but genes do not encode knowledge directly. Similarly, in machine learning, a learning framework is programmed, but the task's solution is to be learnt.

\acp{ANN} represent a major branch of today's machine learning, solving tasks in computer vision and speech recognition with performances and robustness highly superior to that of traditional handcrafted algorithms. This has motivated this research to apply \acp{ANN} to the field of \ac{PAM}, as stated in the following problematic.


\section{Thesis objectives}

This thesis was co-financed by the GIAS European project, intending to improve navigation security in the Mediterranean sea (western bassin). One of the thematic of this project, in which this thesis takes part, is the mitigation of whale-ship collision risk. For that purpose, a `smart bioacoustic buoy' was designed, with the intent of automatically detecting the large cetaceans of the zone acoustically (sperm whales and fin whales). Alert could thus be transmitted close to real-time, for ships to adapt their speed or route accordingly.

Being a thesis in computer sciences, the goal is to design and implement the acoustic detection algorithms embedded in the buoy (other parties take care of the hardware developments). As previously introduced, to achieve this, \acp{ANN} were chosen.

This work of training \acp{ANN} for the detection of cetacean sounds quickly expanded well beyond the initial needs of the GIAS project. Indeed, the team participates in a variety of projects, described in section \ref{chap:data_Toulon}. In each of them, the role of the team is typically to analyse large amounts of recordings to advance on biological questions. Hence the need for cetacean acoustic detection and/or classification mechanisms. Moreover, the performances demonstrated by \acp{ANN} in many scientific applications, in cetacean bioacoustics, and in the team itself motivated to use them extensively.

\begin{figure}
    \centering
    \includegraphics[width=\linewidth]{fig/thesis_flowchart.pdf}
    \caption{Flowchart of the typical process when using \acp{ANN} for bioacoustics. The main steps covered by this thesis are shown by arrows with their associated chapters.}
    \label{fig:thesis_flowchart}
\end{figure}

This is how the objective of this work dissociated from the GIAS implementation to become a general study of applying \acp{ANN} to cetacean acoustic detection and classification. After going through the \ac{SOTA} and the material at hand, this manuscript thus revolves around the three main steps needed to fulfill this task (Fig. \ref{fig:thesis_flowchart}). It starts with the construction of training databases, describing annotation procedures suited for a variety of constraints (depending on the recordings at hand and the target signals). Then, to train \acp{ANN} on these databases, architectures and frameworks are described to yield robust detection and classification mechanisms (again depending on constraints of computational power and target signals). Finally, for some of the trained models, applications are illustrated around two main axis : species conservation and communication modelling.

%This thesis demonstrates several \ac{PAM} use cases using \acp{ANN}. It lies between a tutorial on how to use \ac{ANN}s for \ac{PAM}, an empirical study of what works and what doesn't, and the demonstration of the wide potential ahead of this approach. It thus revolves around the following problematic : how to best use \acp{ANN} for cetacean vocalization detection ? from data annotation, architecture design, and training regularization, to detection exploitation for biological insights.