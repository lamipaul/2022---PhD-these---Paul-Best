\chapter{Material}
\minitoc

In this chapter will be introduced the material used for the experiments conducted throughout this thesis. It takes form as underwater acoustic data, containing several types of signals, and recorded at different places and times. This chapter will thus revolve around two axis : the studied signals, and the recording setups. 

The diverse set of target signals described here are those for which detection and classification systems were built. Their characteristics are summarised in Table \ref{tab:recap_data}, and each subsection then underpins the current scientific knowledge about them, especially regarding their context of emission.


\section{Target species and signals}

\begin{table}[ht]
\centering\resizebox{\linewidth}{!}{
\begin{tabular}{|l|c|c|c|c|c|}
        \hline
        \textbf{Species} & \textbf{Sperm whale} & \textbf{Fin whale} & \textbf{Orca} & \textbf{Dolphin} & \textbf{Humpback whale} \\ \hline
        \textbf{Sub-order} & Odontoceti & Mysticeti & Odontoceti & Odontoceti & Mysticeti \\ \hline
        \textbf{Signal} & clicks & 20Hz pulses & pulsed calls & whistles & calls \\ \hline
        \textbf{Signal type} & Transitory & Transitory & Stationary & Stationary & Stationary \\ \hline
        \textbf{Frequency (Hz)} & 12,000 & 20 & [500; 5,000] & [5,000. 20,000] & [300; 3,000] \\ \hline
        \textbf{Bandwidth} & 20kHz & 2Hz & 100Hz & 20Hz & 50Hz \\ \hline
        \textbf{Duration (sec)} & 0.001 & 1 & [0.5; 2] & [1; 2] & [0.5; 1] \\ \hline
%        \textbf{Interval between signals (sec)} & [0.1, 1] & [10, 25] &  &  &  \\ \hline
    \end{tabular}}
    \caption{Summary of the target signals for the detection systems built through this thesis. For transitory waves, the frequency denotes the approximate centroid frequency, when For stationary waves it is the range}
    \label{tab:recap_signals}
\end{table}

\subsection{Fin whale (\textit{\ac{Bp}}) 20Hz pulses}
As the second largest animal on earth, the fin whale produces very low-pitched vocalizations, barely noticeable to the human ear. So far, bioacousticians have documented 3 main types of signals emitted by fin whales : 100-30Hz down-sweeps, 30Hz rumbles, and 20Hz pulses. They supposedly serve group cohesion \cite{payne1971orientation, watkins1981activities}, food signaling \cite{Romagosa}, and mate attraction \cite{Watkins_Tyack_Moore_Bird_1987, croll2002only}.


\begin{figure}[!htb]
   \begin{minipage}{0.5\textwidth}
     \centering
     \includegraphics[width=\linewidth]{fig/20Hzpulse.pdf}
   \end{minipage}\hfill
   \begin{minipage}{0.5\textwidth}
     \centering
     \includegraphics[width=\linewidth]{fig/20Hzpulse_wave.pdf}
   \end{minipage}\hfill
   \caption{Spectrogram (left) and waveform (right) of a fin whale pulse recorded by Bombyx (the two figures share the same abscissa). \ac{STFT} parameters are : $fs=100Hz$, $NFFT=128$, $padding=50\%$, $hopsize=3$.}\label{fig:20Hzpulse}
\end{figure}

In this thesis, I will focus on the most common signal : the 20Hz pulse. It is often further classified into two sub categories, named A and B, or classic pulse and back-beat \cite{sciacca2015annual}. They highly resembles a Gabor wavelet : a sine wave enveloped by a Gaussian (see Fig.\ref{fig:20Hzpulse}), and can be emitted either as single pulses, or in patterned sequences, termed as songs \cite{simon2010singing}. The pulses and the sequences they take part in are highly stereotyped : pulses show very low variability both in frequency and duration, and when in sequences, the \ac{INI} remains highly stable.

Fin whale song characteristics, especially the \ac{INI}, are population specific \cite{delarue2009geographic, castellote2012fin}. They also are subject to seasonal cyclic variations \cite{Oleson, morano2012seasonal}, as well as long-term trends \cite{weirathmueller2017spatial, helble2020fin} (e.g. linear increase of the \ac{INI} through years).


\subsection{Sperm whale (\textit{Physeter Macrocephalus}) clicks}
Sperm whales produce echolocation clicks to navigate and locate preys during hunts. Their large head contains a series of oil sacks and a reverberation system that allows for the amplification of the impulses \cite{norris1972theory}, making it the most powerful sonar in the animal kingdom \cite{mohl2003monopulsed} (the loudest recorded click was at 230\,dB re: 1$\mu$Pa rms).

%The reverberation mechanism yields several pulses within a click, and the interval between them is often used as a proxy to estimate the animal's size \cite{gordon1991evaluation}. This interval, typically of around half a millisecond, is usually termed \ac{IPI}, but not to be confused with fin whale's \ac{IPI}, which describes 

\begin{figure}
    \centering
    \includegraphics[width=\linewidth]{fig/cacha_clicks.pdf}
    \caption{Sequence of sperm whale echolocation clicks recorded by Bombyx in july 2018. \ac{STFT} parameters are : $fs=50kHz$, $NFFT=1,024$, $hopsize=896$, $padding=0\%$.}
    \label{fig:cacha_clicks}
\end{figure}

Echolocation clicks usually come in sequences (see Fig.\ref{fig:cacha_clicks}), with the \ac{ICI} ranging between 0.01 and 1sec, usually decreasing when approaching a prey \cite{fais2016sperm}. The clicks lie around relatively low frequencies (between 3kHz and 30kHz), especially compared to other smaller odontocetes. % pas sur This could enable the echolocation of objects at further distances (lower frequencies are subject to a lower propagation loss).


\subsection{Orca (\textit{Orcinus Orca}) calls}
Orcas produce three types of signals : clicks, pulsed calls, and whistles, as described by \citet{ford1989acoustic}. Alike for most dolphin species, clicks presumably serve echolocation, when the two remaining stationary signals would rather be used for communication. Pulsed calls are highly harmonic , typically lying between 500Hz and 5kHz, and lasting up to 1.5sec (see Fig. \ref{fig:orca_call}). On the other hand, whistles show little or no harmonic, are found between 6kHz and 12kHz, and can last up to 12sec. In this thesis, I will focus on the pulsed calls, referring to them as calls or vocalizations.

\begin{figure}
    \centering
    \includegraphics[width=\linewidth]{fig/orca_call.pdf}
    \caption{Sequence of orca tonal calls recorded at OrcaLab in September 2016. \ac{STFT} parameters are $fs=22050Hz$, $NFFT=1024$, $hopsize=50$, $padding=0\%$. The N.. labels denote each call type, with a `?' showing an ambiguous one.}
    \label{fig:orca_call}
\end{figure}

As shown in Fig. \ref{fig:orca_call}, orca calls have stereotyped frequency contours that have been classified into discrete types. These were proven to be community specific (dialectic) \cite{ford1987catalogue}, and subject to cultural evolution \cite{deecke2000dialect, filatova2015cultural}. The identification of call types strongly contributed to the study of the orca's social structures, and is widely accepted by the scientific community. Difficulties remain however, for some calls to be attributed to one class or another, especially for non experts. Indeed, despite calls being stereotyped, they still are prone to variability which might lead to overlap between classes' characteristics.


\subsection{Dolphin (\textit{Delphinidae}) whistles}
Exceptionally for this type of signal, we do not target a single species, but rather a family of species, the \textit{Delphinidae} which includes sub-families such as \textit{Globicephalinae}, \textit{Delphininae}, and \textit{Orcininae}. They all produce whistles, which are typically high pitched, tonal, and narrow-band. Their frequency contour can be stereotyped \cite{vester2017vocal}, individual specific \cite{caldwell1965individualized}, and serve group cohesion \cite{janik1998context}.

\begin{figure}
    \centering
    \includegraphics[width=\linewidth]{fig/whistles.pdf}
    \caption{Sequence of dolphin whistles from the Carimam dataset. \ac{STFT} parameters are $fs=256kHz$, $NFFT=8192$, $hopsize=512$, $padding=0\%$, Mel transformed from 5 to 40kHz, and \ac{PCEN} normalized. The stationary signal around 12kHz is the remaining self noise of the sound card used \cite{barchasz2020novel} (despite heavy mitigation via the \ac{PCEN}).}
    \label{fig:whistle}
\end{figure}


\subsection{Humpback whale (\textit{Megaptera Novaeangliae}) calls}
The humpback whale song is among the most widely studied cetacean acoustic signals. These sequences are mostly emitted by males during the reproductive season, presumably playing a role in courtship \cite{herman2017multiple} (male-female and/or male-male interaction). They follow strict hierarchical structures : series of units form phrases that are arranged into themes, themselves combined in songs that can last several hours \cite{payne1971songs}.

\begin{figure}
    \centering
    \includegraphics[width=\linewidth]{fig/HB_calls.pdf}
    \caption{Extract of a humpback whale song from the Carimam dataset. \ac{STFT} parameters are $fs=22050Hz$, $NFFT=4096$, $hopsize=48$, $padding=50\%$.}
    \label{fig:HB_song}
\end{figure}

Each components of the hierarchical structure of the humpback whale songs are stereotyped, as demonstrated in Fig. \ref{fig:HB_song} with a sequence of stereotyped units. Moreover, song structures are shared by individuals at a given place and time, with cultural implications for their spatio-temporal evolution \cite{whitehead2015cultural}.


\section{Data at hand}

\subsection{Data from Toulon University}\label{chap:data_Toulon}
Table \ref{tab:recap_data} summarises some of the data yielded by the partnerships and projects that H. Glotin set up at Toulon University. They are stored locally in a \ac{NAS} system, funded by the DYNI team projects and maintained by the LIS laboratory.

% \begin{sidewaystable}
% \centering
\begin{table}
\centering \resizebox{\linewidth}{!}{
\begin{tabular}{|l|c|c|c|c|c|c|}
\hline
 \textbf{Data source} & \textbf{Boussole} \cite{parsuivi} & \textbf{Bombyx} \cite{vamos2017} & \textbf{OrcaLab} \cite{orcalab} & \textbf{Carimam} \cite{carimam} & \textbf{Antares} \cite{giacomelli2009antares} \\ \hline
\textbf{Location} & Côte d'Azur & Côte d'Azur & British Columbia & Caribbean & Côte d'Azur \\ \hline
\textbf{Depth (m)} & 10 - 25 & 25 & 0 - 20 & 5 - 20 & 2,440 \\ \hline 
\textbf{Recording year} & 2008-2009 & 2015-2018 & 2015-2021 & 2021-today & 2021-today \\ \hline
\textbf{Sampling rate (Hz)} & 32,000 & 50,000 & 22,050 - 44,100 & 384,000 & 195,312 \\ \hline
%OL 22,050 until March 18, then 44,100
\textbf{ON/OFF protocole (min)} & 5/10 & 1/5 - 5/15 & Continuous & 1/5 & Continuous \\ \hline
%Bombyx 1/5 until Oct. 17, then 5/15
\textbf{Channels} & 2 & 2 & 5 & 15 & 3 \\ \hline
\textbf{Recorded time (hours)} & 1,752 & 3,533 &  & 5,677 & 514 \\ \hline
\end{tabular}}
\caption{\label{tab:recap_data}Summary of the recording characteristics for each data source available at Toulon University.}
\end{table}
% \end{sidewaystable}

\begin{figure}[!htb]
   \begin{minipage}{0.7\textwidth}
     \centering
     \includegraphics[width=\linewidth]{fig/bombyx.png}
   \end{minipage}\hfill
   \begin{minipage}{0.25\textwidth}
     \centering
     \includegraphics[width=\linewidth]{fig/boussole.png}
   \end{minipage}\hfill
   \caption{(left) Installation of the Bombyx antenna. (right) Structure of the Boussole antenna}\label{fig:antennas}
\end{figure}


%TODO add map boussole + bombyx + KM3Net

%TODO add map orcalab & carimam


\subsection{Blue and Fin whale acoustic trends dataset} \label{chap:acTrend}

In early 2021, a large acoustic dataset of antarctic mysticetes was made publicly available \cite{miller2021open}. It was built by a working group from the \ac{SOOS} titled as : Acoustic Trends of Antarctic blue and fin whales (Acoustic Trends Working Group; ATWG). The following is an extract from their terms of reference \cite{atwg_tor}:
\\
\textbf{\underline{SOOS Capability Working Group Key Objective(s):}}

Continue to develop and mature a long term acoustic research program to understand trends in Southern Ocean blue and fin whale distribution, seasonal presence, and population growth through the use of passive acoustic monitoring techniques. Implementation of these objectives will occur via:

\begin{enumerate}
    \item analysis and interpretation of existing ad-hoc acoustic datasets in from the Southern Ocean,
    \item the development and implementation of an ongoing network of long-term circumpolar underwater listening stations, and
    \item development of novel and efficient methods for standardized analysis of acoustic data collected in the Antarctic and sub-Antarctic
\end{enumerate}

It is regarding this third axis of work that the Acoustic Trends dataset was built and published, especially to share performance metrics for detection systems. It gathers annotations from a group of experts, on data yielded by several recorders at different locations from 2005 to 2017 (see Fig. \ref{fig:soos_map} and Tab. \ref{tab:soos_data}). Target signals are of 7 classes, 4 vocalization types from blue whales (\textit{Balaenoptera Musculus}) and 3 vocalization types from fin whales (\textit{Balaenoptera Physalus}).

Since this data has not been subject to custom annotations, I won't expand on the target signals which are described in the dataset publication \cite{miller2021open}.

\begin{figure}
    \centering
    \includegraphics[width=.7\linewidth]{fig/soos_ATW_map.png}
    \caption{Map of the recording stations used in the Acoustic Trends dataset. The map was published by \citet{miller2021open}.}
    \label{fig:soos_map}
\end{figure}


\begin{table}[ht]
\centering
    \begin{tabular}{c|c|c|c}
    \textbf{Location} & \textbf{Year} & \textbf{Instrument} & \textbf{Recordings (hours)}\\ \hline
    Balleny Islands & 2015 & PMEL-AUH & 204\\
    Elephant Island & 2013 & AURAL & 707 \\
    Elephant Island & 2014 & AURAL & 216 \\
    Greenwich 64S & 2015 & Sono.Vault & 32 \\
    MaudRise & 2014 & AURAL & 80 \\
    Ross Sea & 2014 & PMEL-AUH & 184 \\
    Casey & 2014 & AAD-MAR & 194 \\
    Casey & 2017 & AAD-MAR & 187 \\
    Kerguelen 1 & 2005 & ARP & 200 \\
    Kerguelen 2 & 2014 & AAD-MAR & 200  \\
    Kerguelen 2 & 2015 & AAD-MAR & 200 \\
    \end{tabular}
    \caption{Summary of recorders' characteristics and amounts of data available in the Acoustic Trends dataset.}\label{tab:soos_data}
\end{table}
