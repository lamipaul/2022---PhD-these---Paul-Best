\chapter{Material}\label{chap:material}
\minitoc
This chapter presents the material used for the experiments conducted throughout the thesis. It takes form as underwater acoustic data, containing several types of signals, and recorded at different places and times. This chapter will thus revolve around two axis: the studied signals, and the recording setups. 


\section{Target species and signals}

The target signals described here are those for which detection and classification systems were built. Their characteristics are summarised in Table~\ref{tab:recap_signals}, and each subsection underpins the current knowledge about them, especially regarding their context of emission.

Note that for the signal types of Table \ref{tab:recap_signals} and throughout this thesis, stationary refers to ``signals locally stable in frequency'' (calls, whistles) as opposed to transitory signals (clicks).

\begin{table}[ht]
\centering\resizebox{\linewidth}{!}{
\begin{tabular}{|l|c|c|c|c|c|}
        \hline
        \textbf{Species} & \textbf{Sperm whale} & \textbf{Fin whale} & \textbf{Orca} & \textbf{Dolphin} & \textbf{Humpback whale} \\ \hline
        \textbf{Sub-order} & Odontoceti & Mysticeti & Odontoceti & Odontoceti & Mysticeti \\ \hline
        \textbf{Signal} & clicks & 20\,Hz pulses & pulsed calls & whistles & calls \\ \hline
        \textbf{Signal type} & Transitory & Transitory & Stationary & Stationary & Stationary \\ \hline
        \textbf{Frequency (Hz)} & 12,500 & 20 & [500; 5,000] & [5,000. 20,000] & [300; 3,000] \\ \hline
        \textbf{Bandwidth (Hz)} & 20,000 & 6 & 100 & 20 & 50 \\ \hline
        \textbf{Duration (sec)} & 0.001 & 1 & [0.5; 2] & [1; 2] & [0.5; 1] \\ \hline
%        \textbf{Interval between signals (sec)} & [0.1, 1] & [10, 25] &  &  &  \\ \hline
    \end{tabular}}
    \caption{Summary of the target signals for the detection systems built throughout this thesis. For transitory waves, the frequency denotes the approximate centroid frequency, for stationary signals it denotes its range}
    \label{tab:recap_signals}
\end{table}


\subsection{Sperm whale (\textit{Physeter Macrocephalus}) clicks}

Sperm whales produce echolocation clicks to navigate and locate preys during hunts. Their large head contains a series of oil sacks surrounded by sound-reflecting air sacs that amplify impulses \cite{norris1972theory}, making it the most powerful sonar in the animal kingdom \cite{mohl2003monopulsed} (the loudest recorded click was at 230\,dB re:~1$\mu$Pa rms).

%The reverberation mechanism yields several pulses within a click, and the interval between them is often used as a proxy to estimate the animal's size \cite{gordon1991evaluation}. This interval, typically of around half a millisecond, is usually termed \ac{IPI}, but not to be confused with fin whale's \ac{IPI}, which describes 

\begin{figure}
    \centering
    \includegraphics[width=\linewidth]{fig/cacha_clicks.pdf}
    \caption{Sequence of sperm whale echolocation clicks recorded by Bombyx in July 2018. \ac{STFT} parameters are: $fs=50kHz$, $NFFT=1,024$, $hop=128$, $padding=0\%$.}
    \label{fig:cacha_clicks}
\end{figure}

Echolocation clicks usually come in sequences (see Fig.\ref{fig:cacha_clicks}), with the \ac{ICI} ranging between 0.01 and 1\,sec, usually decreasing when approaching a prey \cite{fais2016sperm}. The clicks lie around relatively low frequencies compared to other smaller odontocetes (between 3\,kHz and 30\,kHz). % pas sur This could enable the echolocation of objects at further distances (lower frequencies are subject to a lower propagation loss).


\subsection{Fin whale (\textit{Balaenoptera Physalus}) 20Hz pulses}

As the second largest animal on earth, the fin whale produces very low-pitched vocalisations, barely noticeable to the human ear. So far, bioacousticians have documented 3 main types of signals emitted by fin whales: 100-30\,Hz down-sweeps, 30\,Hz rumbles, and 20\,Hz pulses. They supposedly serve group cohesion \cite{payne1971orientation, watkins1981activities}, food signaling \cite{Romagosa}, and mate attraction \cite{Watkins_Tyack_Moore_Bird_1987, croll2002only}.


\begin{figure}[!htb]
   \begin{minipage}{0.5\textwidth}
     \centering
     \includegraphics[width=\linewidth]{fig/20Hzpulse.pdf}
   \end{minipage}\hfill
   \begin{minipage}{0.5\textwidth}
     \centering
     \includegraphics[width=\linewidth]{fig/20Hzpulse_wave.pdf}
   \end{minipage}\hfill
   \caption{Spectrogram (left) and waveform (right) of a fin whale pulse recorded by Bombyx. \ac{STFT} parameters are: $fs=100Hz$, $NFFT=128$, $hop=3$, $padding=50\%$.}\label{fig:20Hzpulse}
\end{figure}

In this thesis, I will focus on the most common signal: the 20\,Hz pulse. It is often further classified into two sub categories, named A and B, or classic pulse and back-beat \cite{sciacca2015annual}. They highly resembles a Gabor wavelet: a sine wave enveloped by a Gaussian (see Fig.\ref{fig:20Hzpulse}), and can be emitted either as single pulses, or in patterned sequences, termed as songs \cite{simon2010singing}. The pulses and the sequences they take part in are highly stereotyped: pulses show very low variability both in frequency and duration, and when in sequences, the \ac{INI} remains highly stable.

Fin whale song characteristics, especially the \ac{INI}, are population specific \cite{delarue2009geographic, castellote2012fin}. They also are subject to seasonal cyclic variations \cite{Oleson, morano2012seasonal} and long-term trends \cite{weirathmueller2017spatial, helble2020fin} (e.g.~linear increase of the \ac{INI} through years).


\subsection{Orca (\textit{Orcinus Orca}) calls}\label{chap:mat_calltype}

Orcas produce three types of signals: clicks, pulsed calls, and whistles \cite{ford1989acoustic}. As for most dolphin species, clicks presumably serve echolocation, while the two stationary signals would rather be used for communication. Pulsed calls are highly harmonic, typically lying between 500\,Hz and 5\,kHz, and lasting up to 1.5 seconds (Fig.~\ref{fig:orca_call}). On the other hand, whistles show little or no harmonic structure, lay between 6\,kHz and 12\,kHz, and can last up to 12 seconds. This thesis focuses on the pulsed calls, referring to them as calls or vocalisations.

\begin{figure}
    \centering
    \includegraphics[width=\linewidth]{fig/orca_call.pdf}
    \caption{Sequence of orca tonal calls recorded at OrcaLab in September 2016. \ac{STFT} parameters are $fs=22050Hz$, $NFFT=1024$, $hop=50$, $padding=0\%$. The N.. labels denote each call type \cite{ford1987catalogue}, with a `?' showing an ambiguous one.}
    \label{fig:orca_call}
\end{figure}

As shown in Fig.~\ref{fig:orca_call}, some orca calls have stereotyped frequency contours that have been classified into discrete types \cite{ford1987catalogue}. These were proven to be community specific (dialectic), and subject to cultural evolution \cite{deecke2000dialect, filatova2015cultural}. The identification of call types strongly contributed to the study of the orca's social structures, and its categorisation is widely accepted by the scientific community. Difficulties remain however, for some calls to be attributed to one class or another, especially for non experts. Indeed, despite calls being stereotyped, they still are prone to variability which might lead to overlap between classes' characteristics \cite{ford1989acoustic}.


\subsection{Dolphin (\textit{Delphinidae}) whistles}

Exceptionally for this type of signal, we do not target a single species, but rather a family of species: the \textit{Delphinidae}. It includes several sub-families such as \textit{Globicephalinae}, \textit{Delphininae}, and \textit{Orcininae}. They all produce whistles, which are typically high pitched, tonal, and narrow-band. Their frequency contour can be stereotyped \cite{vester2017vocal}, individual specific \cite{caldwell1965individualized}, and serve group cohesion \cite{janik1998context}.

\begin{figure}
    \centering
    \includegraphics[width=.9\linewidth]{fig/whistles.pdf}
    \caption{Sequence of dolphin whistles from the Carimam dataset. \ac{STFT} parameters are $fs=256kHz$, $NFFT=8,192$, $hop=512$, $padding=0\%$, Mel transformed from 5 to 40\,kHz, and \ac{PCEN} normalised. The stationary signal around 12\,kHz is the remaining self noise of the sound card used \cite{barchasz2020novel} (despite heavy mitigation via the \ac{PCEN}).}
    \label{fig:whistle}
\end{figure}


\subsection{Humpback whale (\textit{Megaptera Novaeangliae}) calls}

The humpback whale song is among the most widely studied cetacean acoustic signals. These vocalisation sequences are mostly emitted by males during the reproductive season, presumably playing a role in courtship \cite{herman2017multiple} (male-female and/or male-male interaction). They follow strict hierarchical structures: series of units form phrases that are arranged into themes, themselves combined in songs that can last several hours \cite{payne1971songs}.

\begin{figure}
    \centering
    \includegraphics[width=.9\linewidth]{fig/HB_calls.pdf}
    \caption{Extract of a humpback whale song from the Carimam dataset. \ac{STFT} parameters are $fs=22050Hz$, $NFFT=4096$, $hopsize=48$, $padding=50\%$.}
    \label{fig:HB_song}
\end{figure}

Each component of the hierarchical structure of the humpback whale songs are stereotyped, as seen in Fig.~\ref{fig:HB_song} with a sequence of stereotyped units. Moreover, song structures are shared by individuals at a given place and time, with cultural implications for their spatio-temporal evolution \cite{whitehead2015cultural}.


\section{Data at hand}

In order to experiment on detection and classification mechanisms for the target signals aforementioned, datasets are needed. Through this thesis, work was conducted on recordings from both local and publicly available databases. They involve a variety of recorders, locations and time spans which are described in this section. It starts with the local projects of the DYNI team (Toulon University) which is followed by the public Blue and Fin whale Acoustic Trends dataset.


\subsection{Data from DYNI}\label{chap:data_Toulon}

Table \ref{tab:recap_data} summarises some of the sources of data H. Glotin co-set up at Toulon University. Three of them are antennas located in the Ligurian Sea (Fig.~\ref{fig:map_med}). The data is stored locally in a \ac{NAS} system, funded by the DYNI team projects and maintained by the LIS laboratory. Each are briefly introduced in the following sections.

\begin{figure}
    \centering
    \includegraphics[width=.75\linewidth]{fig/map_ligure.png}
    \caption{Map of the three Mediterranean antennas used throughout this thesis.}
    \label{fig:map_med}
\end{figure}

\begin{sidewaystable}
\centering
\begin{tabular}{|l|c|c|c|c|c|c|}
\hline
 \textbf{Data source} & \textbf{Boussole} \cite{parsuivi} & \textbf{Bombyx} \cite{vamos2017} & \textbf{KM3Net} \cite{aiello2021km3net} & \textbf{OrcaLab} \cite{orcalab} & \textbf{Carimam} \cite{carimam} \\ \hline
\textbf{Location} & Côte d'Azur & Côte d'Azur & Côte d'Azur & British Columbia & Caribbean \\ \hline
\textbf{Depth (m)} & 10 - 25 & 25 & 2,440 & 0 - 20 & 5 - 20 \\ \hline 
\textbf{Recording year(s)} & 2008-2009 & 2015-2018 & 2020-2021 & 2015-2020 & 2021-today \\ \hline
\textbf{Sampling rate (Hz)} & 32,000 & 50,000 & 195,312 & 22,050 - 44,100 & 256,000 \\ \hline
%OL 22,050 until March 18, then 44,100
\textbf{ON/OFF protocol (min)} & 5/10 & 1/5 - 5/15 & Continuous & Continuous & 1/5 \\ \hline
%Bombyx 1/5 until Oct. 17, then 5/15
\textbf{\# Hydrophones} & 1 & 2 & 3 & 5 & 15 \\ \hline
\textbf{Recorded time (hours)} & 1,752 & 3,533 & 514 & $\approx$ 40,000 & 5,677 \\ \hline
\end{tabular}
\caption{\label{tab:recap_data}Summary of the recording characteristics for each data source available at Toulon University.}
\end{sidewaystable}

\begin{figure}[!htb]
   \begin{minipage}{0.6\textwidth}
     \centering
     \includegraphics[width=\linewidth]{fig/bombyx.png}
   \end{minipage}\hfill
   \begin{minipage}{0.4\textwidth}
     \centering
     \includegraphics[width=.6\linewidth]{fig/boussole.png}
   \end{minipage}\hfill
   \caption{(left) Installation of the Bombyx stereophonic antenna \cite{vamos2017}. (right) Structure of the Boussole antenna \cite{parsuivi}}\label{fig:antennas}
\end{figure}


\subsubsection{Boussole}

This project consisted in a partnership between GIS3M, Pelagos marine mammal sanctuary, and Port-Cros National Park. In order to study marine mammals acoustic activity, a monophonic recording system was placed on the Boussole buoy \cite{parsuivi}. Originally dedicated to marine optics and designed to be transparent to the swell, this buoy was moored on the 2,440 meters deep sea floor, off the coast of Nice (France). During 4 phases between October 2008 and September 2009, the system recorded at 32\,kHz, enabling the detection of vocalisations from sperm whales, fin whales, and delphinids of the area (\textit{Stenella coeruleoalba, Globicephala melas, Grampus griseus, Tursiops truncatus and Delphinus delphis}).

A study prior to this thesis intended to monitor the acoustic presence of sperm whales and fin whales in these recordings. The automatic detection of sperm whale clicks was successful but the fin whale 20\,Hz pulses analysis was hindered by the self noise of the system \cite{parsuivi} (Fig.~\ref{fig:boussole_noise}).

\begin{figure}
    \centering
    \includegraphics[width=\linewidth]{fig/boussole_noise.pdf}
    \caption{Spectrogram of a noisy recording from the Boussole antenna ($f_s=32kHz$, $NFFT=32,768$, $hop=5,568$). White dots denote the temporal position of some confirmed fin whale 20\,Hz pulses.}
    \label{fig:boussole_noise}
\end{figure}


\subsubsection{Bombyx}

The Bombyx antenna was set up by a partnership between Toulon University, Port-Cros National Park, TVT Innovation, and the Pelagos marine mammal sanctuary. Being placed right on the rift of a 2,000 meters deep canyon, it allows to monitor sperm whales hunting in the area \cite{vamos2017}, and did so during several phases spread across 4 years (2015 to 2018). The area is of interest because of the nearby canyons prone to sperm whale hunts \cite{fiori2014geostatistical}, but also because of the ferries that travel across on a daily basis. In addition to the noise that the latter generate, Bombyx recordings are also subject to self noise (Fig.~\ref{fig:bombyx_noise}).

\begin{figure}
    \centering
    \includegraphics[width=\linewidth]{fig/bombyx_noise.pdf}
    \caption{Example of signal from the Bombyx antenna (high pass filtered, order 3 butterworth at 3\,kHz). Orange dots denote sperm whale clicks, and red ones self noise from the recording device. (top) Waveform. (bottom) Spectrogram ($f_s=5kHz$, $NFFT=512$, $hop=256$). }
    \label{fig:bombyx_noise}
\end{figure}


\subsubsection{KM3Net}

The ORCA detector of the KM3Net observatory is an array of detection units allowing the measurement of neutrino particles \cite{aiello2021km3net}. It was installed on the seabed, 2,440 meters deep, connected to the shore of Toulon (France) via a fiber cable. Hydrophones are used as part of a positioning system, but as a by-product, also enable the \ac{PAM} of local cetaceans.


\subsubsection{OrcaLab} \label{chap:OL}

In the 1970s, Paul Spong founded OrcaLab, an in-situ observatory in the Johnstone Strait \cite{orcalab} (British Columbia, Fig.~\ref{fig:ol_map}). It serves the visual and acoustic monitoring of orcas, especially the population that feeds on the local salmon every summer: the \ac{NRKW}. From 2015 to 2020, the 5 hydrophones' signal has been recorded continuously  (at 22,050\,Hz until march 2018, then at 44,100\,Hz).

The fact that the orcas regularly come to this relatively confined space represents an unique opportunity to observe and listen to them 24/7 from the shore. Most importantly, it guarantees no behavioural disturbance and a continuous supply of power and data storage (the main constraints of many \ac{PAM} approaches).

\begin{figure}
    \centering
    \includegraphics[width=.8\linewidth]{fig/OL_map.png}
    \caption{Map of the OrcaLab observatory, with its 5 hydrophones and associated acoustic range.}
    \label{fig:ol_map}
\end{figure}


\subsubsection{Carimam}

\begin{figure}
    \centering
    \includegraphics[width=.8\linewidth]{fig/effort_T1+T2.jpg}
    \caption{Map of recording stations for the Carimam project, in the Caribbean archipelago. Colours denote the recording effort as of the end of 2021.}
    \label{fig:carimam_effort}
\end{figure}

The Carimam project, led by a consortium composed of AGOA, the OFB and Toulon University, is a network of 16 monophonic acoustic stations spread through the Caribbean archipelagos \cite{carimam} (Fig.~\ref{fig:carimam_effort}). It aims at monitoring the rich marine mammal activity of the area. To manage such a wide number of stations, low-cost recording devices (HighBlue \cite{barchasz2020novel}) were sent to local environmental managers, who set them up on existing mooring lines close to the shore.


\subsubsection{Spatialisation}

In Table \ref{tab:recap_data}, the number of hydrophones is given for each recording system. When synchronised and with overlapping acoustic coverage, multi-channel data can serve the spatialisation of acoustic sources. It can be done via the computation of \acp{TDOA} for signals to be triangulated.
\begin{itemize}\setlength{\itemsep}{1pt}
    \item For Bombyx, since two hydrophones record 1.8 meters apart (on the same horizontal plane, see Fig.~\ref{fig:antennas}), two possible azymuths can be computed (for acoustic sources of the same horizontal plane).
    \item For KM3Net, the 3 hydrophones are synchronised and approximately 30 meters apart. With prior knowledge on the depth of a source, its coordinates could be estimated.
    \item For Carimam, the stations' acoustic coverage do not overlap: the spatial precision is the acoustic range of the antennas.
    \item Finally, for the OrcaLab network, hydrophones are several kilometers apart but sent to a centralised \ac{DAC} via radio waves, which makes them temporally synchronised. Therefore, spatialisation could be performed in the zones of acoustic overlap.
\end{itemize}


\subsection{Blue and Fin whale acoustic trends dataset} \label{chap:acTrend}

In early 2021, a large acoustic dataset of antarctic mysticetes was made publicly available \cite{miller2021open}. It was built by a working group from the \ac{SOOS} titled Acoustic Trends of Antarctic blue and fin whales (Acoustic Trends Working Group; ATWG). The following is an extract from their terms of reference \cite{atwg_tor}:
\\
\textbf{\underline{SOOS Capability Working Group Key Objective(s):}}
\textit{
Continue to develop and mature a long term acoustic research program to understand trends in Southern Ocean blue and fin whale distribution, seasonal presence, and population growth through the use of passive acoustic monitoring techniques. Implementation of these objectives will occur via:
\begin{enumerate} \setlength{\itemsep}{1pt}
    \item analysis and interpretation of existing ad-hoc acoustic datasets from the Southern Ocean,
    \item the development and implementation of an ongoing network of long-term circumpolar underwater listening stations, and
    \item development of novel and efficient methods for standardised analysis of acoustic data collected in the Antarctic and sub-Antarctic
\end{enumerate}}

\begin{figure}
    \centering
    \includegraphics[width=.6\linewidth]{fig/soos_ATW_map.png}
    \caption{Map of the recording stations used in the Acoustic Trends dataset. The map was published by \citet{miller2021open}.}
    \label{fig:soos_map}
\end{figure}

It is regarding this third axis of work that the Acoustic Trends dataset was built and published \cite{miller2021open}, especially to share performance metrics for detection systems. It gathers annotations from a group of experts, on data yielded by several recorders at different locations from 2005 to 2017 (see Fig.~\ref{fig:soos_map} and Tab.~\ref{tab:soos_data}).

\begin{table}[ht]
\centering
    \begin{tabular}{c|c|c|c}
    \textbf{Location} & \textbf{Year} & \textbf{Instrument} & \textbf{Recordings (hours)}\\ \hline
    Balleny Islands & 2015 & PMEL-AUH & 204\\
    Elephant Island & 2013 & AURAL & 707 \\
    Elephant Island & 2014 & AURAL & 216 \\
    Greenwich 64S & 2015 & Sono.Vault & 32 \\
    MaudRise & 2014 & AURAL & 80 \\
    Ross Sea & 2014 & PMEL-AUH & 184 \\
    Casey & 2014 & AAD-MAR & 194 \\
    Casey & 2017 & AAD-MAR & 187 \\
    Kerguelen 1 & 2005 & ARP & 200 \\
    Kerguelen 2 & 2014 & AAD-MAR & 200  \\
    Kerguelen 2 & 2015 & AAD-MAR & 200 \\
    \end{tabular}
    \caption{Summary of recorders' characteristics and amounts of data available in the Acoustic Trends dataset.}\label{tab:soos_data}
\end{table}

Annotations gather 7 target signals: 4 types of vocalisation from blue whales (\textit{\ac{Bm}}) and 3 types of vocalisation from fin whales (\textit{\ac{Bp}}). They all lie in low frequencies (between 20\,Hz and 100\,Hz) lasting from 1 to 15 seconds (their time-frequency distributions are shown in Figure \ref{fig:misti_calls}). 

\begin{figure}
    \centering
    \includegraphics[width=.9\linewidth]{fig/calls_misti.png}
    \caption{Distributions of lengths and frequencies for each of the 7 call types of the Acoustic Trends dataset. (left) \textit{\acl{Bm}}, (right) \textit{\acl{Bp}}. The figure comes from \citet{miller2021open}.}
    \label{fig:misti_calls}
\end{figure}
