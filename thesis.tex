
% This one will format for two-sided binding (ie left and right pages have mirror margins; blank pages inserted where needed):
\documentclass[a4paper,twoside]{gamosinos}

%\fancyfoot[C]{\emph{Paul Best, \today}}  %HERE
\usepackage{pdfpages}
\usepackage{listings}
\usepackage{ctable}
\usepackage{acronym}
\usepackage{xcolor}
\usepackage[numbers]{natbib}
\usepackage{xcolor,colortbl}
\usepackage{tabularx}
\usepackage{algorithm}
\usepackage{rotating}
\usepackage{algorithmic}
\usepackage{minitoc}
\usepackage{mdframed}
\usepackage{makecell}
\definecolor{codegreen}{rgb}{0,0.6,0}
\definecolor{codegray}{rgb}{0.5,0.5,0.5}
\definecolor{codepurple}{rgb}{0.58,0,0.82}
\definecolor{backcolour}{rgb}{0.95,0.95,0.92}

\usepackage[utf8]{inputenc}
% Default fixed font does not support bold face
\DeclareFixedFont{\ttb}{T1}{txtt}{bx}{n}{12} % for bold
\DeclareFixedFont{\ttm}{T1}{txtt}{m}{n}{12}  % for normal
\DeclareMathOperator*{\argmax}{argmax}
\DeclareMathOperator*{\argmin}{argmin}
\DeclareMathOperator*{\median}{median}
\DeclareMathOperator*{\std}{std}
\DeclareMathOperator*{\VGG}{VGG}

% Custom colors
\usepackage{color}
\definecolor{deepblue}{rgb}{0,0,0.5}
\definecolor{deepred}{rgb}{0.6,0,0}
\definecolor{deepgreen}{rgb}{0,0.5,0}

% Python style for highlighting
\newcommand\pythonstyle{\lstset{
language=Python,
basicstyle=\ttm,
morekeywords={self},              % Add keywords here
keywordstyle=\ttb\color{deepblue},
emph={MyClass,__init__},          % Custom highlighting
emphstyle=\ttb\color{deepred},    % Custom highlighting style
stringstyle=\color{deepgreen},
frame=tb,                         % Any extra options here
showstringspaces=false
}}
\lstnewenvironment{python}[1][]
{
\pythonstyle
\lstset{#1}
}
{}
\newcommand\pythoninline[1]{{\pythonstyle\lstinline!#1!}}

%%
%% OpenSCAD definition
%%
\lstdefinelanguage{OpenSCAD}%
  {keywords={sphere,cylinder,acos,atan2,translate,rotate,norm},
      morekeywords=[2]{module,false,true},
  keywordstyle=[2]\color{cyan},
   sensitive=true,
   alsoother={},
   morecomment=[l]{//},
   morestring=[s]{"}{"},%
   morestring=[m]{'}{'},%
}[keywords,comments,strings]%

\lstset{%
    language         = OpenSCAD,
    basicstyle       = \ttfamily,
    keywordstyle     = \bfseries\color{blue},
    stringstyle      = \color{magenta},
    commentstyle     = \color{codegreen},
    showstringspaces = false,
}
%%
%% BDF definition
%%
\lstdefinelanguage{BDF}%
  {keywords={ID,SOL,TIME,TITLE,SUBTITLE,LOAD,SPC,DISP,STRESS,STRAIN,ELFORCE,SPCFORCE,BAILOUT,MDLPRM,SPC1,FORCE,GRID,MAT1,CBAR,PBARL,BAR,ALL},
  morekeywords=[2]{CEND,BEGIN,PARAM,BULK,ENDDATA},
  keywordstyle=[2]\color{cyan},
  sensitive=true,
  comment=[l]{\$},
}[keywords,comments,strings]%

\lstset{%
    language         = BDF,
    basicstyle       = \ttfamily,
    keywordstyle     = \bfseries\color{blue},
    stringstyle      = \color{magenta},
    commentstyle     = \color{codegreen},
    showstringspaces = false,
    literate =  {=}{{{\color{red}=}}}1
                {,}{{{\color{red},}}}1,
}

%%
%% VTU definition
%%

\lstdefinelanguage{VTU}{%
  language     = XML,
  morekeywords = {VTKFile,UnstructuredGrid,Piece,Points,DataArray,Cells,CellData},
}

%%
%% STL definition
%%

\lstdefinelanguage{STL}{%
  morekeywords = {solid,endsolid,facet,normal,outer,loop,vertex,endloop,endfacet},
}

%%
%% DOS/Terminal definition
%%

\lstdefinelanguage{DOS}{
    backgroundcolor=\color{black},
    basicstyle=\scriptsize\color{white}\ttfamily,
    numbers=none,
}

\usepackage{afterpage}

\newcommand\blankpage{%
    \null
    \thispagestyle{empty}%
    \addtocounter{page}{-1}%
    \newpage}

%\makeglossaries

\newglossaryentry{supervised learning}
{
    name=supervised learning,
    description={Class of machine learning algorithms that train models with samples along with target labels}
}

\newglossaryentry{unsupervised learning}
{
    name=unsupervised learning,
    description={Class of machine learning algorithms that train models exclusively with unannotated samples}
}

\correctionstrue

% To make text superscripts shortcuts
	\renewcommand{\th}{\textsuperscript{th}} % ex: I won 4\th place
	\newcommand{\nd}{\textsuperscript{nd}}
	\renewcommand{\st}{\textsuperscript{st}}
	\newcommand{\rd}{\textsuperscript{rd}}
    \renewcommand\tabularxcolumn[1]{m{#1}}% for vertical centering text in X column

%----------------------------------------------------------------------------------------
%	START DOCUMENT
%----------------------------------------------------------------------------------------

\begin{document}
\dominitoc
\pagenumbering{gobble}

%%%%% TITLE PAGE INFORMATION
% Everybody needs to complete the following:
\title{Automated Detection and Classification of Cetacean Acoustic Signals}
\author{\textbf{Paul Best}} % HERE
\college{Université de Toulon}
\degree{Doctor of Philosophy}
\degreedate{2022} % HERE


%%%%% CHOOSE YOUR LINE SPACING
% This is the official option.  Use it for your submission copy and library copy:
\setlength{\textbaselineskip}{22pt plus2pt}
% This is closer spacing (about 1.5-spaced) that you might prefer for your personal copies:
%\setlength{\textbaselineskip}{18pt plus2pt minus1pt}

% You can set the spacing here for the roman-numbered pages (acknowledgements, table of contents, etc.)
\setlength{\frontmatterbaselineskip}{17pt plus1pt minus1pt}

% Leave this line alone; it gets things started for the real document.
\setlength{\baselineskip}{\textbaselineskip}

%%%%% CHOOSE YOUR SECTION NUMBERING DEPTH HERE
% You have two choices.  First, how far down are sections numbered?  (Below that, they're named but
% don't get numbers.)  Second, what level of section appears in the table of contents?  These don't have
% to match: you can have numbered sections that don't show up in the ToC, or unnumbered sections that
% do.  Throughout, 0 = chapter; 1 = section; 2 = subsection; 3 = subsubsection, 4 = paragraph...

% The level that gets a number:
\setcounter{secnumdepth}{2}
% The level that shows up in the ToC:
\setcounter{tocdepth}{2}



% JEM: Pages are roman numbered from here, though page numbers are invisible until ToC.  This is in
% keeping with most typesetting conventions.

%\afterpage{\blankpage}
\maketitle

\pagenumbering{arabic}
\begin{romanpages}

%\afterpage{\blankpage}
%----------------------------------------------------------------------------------------
%	QUOTATION PAGE
%----------------------------------------------------------------------------------------

%\vspace*{0.1\textheight}
%\newpage
%\vspace*{0.2\textheight}
%\noindent\enquote{\itshape Thanks to my solid academic training, today I can write hundreds of words on virtually any topic without possessing a shred of information, which is how I got a good job in journalism.}\bigbreak

%\hfill Dave Barry
%\afterpage{\blankpage}
%----------------------------------------------------------------------------------------
%	ACKNOWLEDGEMENTS -- Nothing to do here except comment out if you don't want it.
%----------------------------------------------------------------------------------------
%\begin{acknowledgements}
%\end{acknowledgements}



%%%%% MINI TABLES
% This lays the groundwork for per-chapter, mini tables of contents.  Comment the following line
% (and remove \minitoc from the chapter files) if you don't want this.  Un-comment either of the
% next two lines if you want a per-chapter list of figures or tables.
\dominitoc % include a mini table of contents
%\dominitoc  % include a mini list of figures
%\dominilot  % include a mini list of tables
%\dominilol
%\dominilol

% This aligns the bottom of the text of each page.  It generally makes things look better.
\flushbottom

% This is where the whole-document ToC appears:
\tableofcontents

%\listoffigures
%	\mtcaddchapter
% \mtcaddchapter is needed when adding a non-chapter (but chapter-like) entity to avoid confusing minitoc

% Uncomment to generate a list of tables:
%\listoftables
%	\mtcaddchapter

%\lstlistoflistings
	%\mtcaddchapter
	
%\listofalgorithms
	%\mtcaddchapter
	
	%\mtcaddchapter
%%%%% LIST OF ABBREVIATIONS,symbols, constants...
% This example includes a list of abbreviations.  Look at text/abbreviations.tex to see how that file is
% formatted.  The template can handle any kind of list though, so this might be a good place for a
% glossary, etc.
%\include{text/symbols}
\chapter*{List of Acronyms}
\addcontentsline{toc}{chapter}{List of Acronyms}
\begin{acronym}
\acro{AE}{Auto-Encoder}
\acro{ANN}{Artificial Neural Network}
\acro{ARU}{autonomous recording unit}
\acro{ASHA}{Async Successive Halving Algorithm}
\acro{ASV}{Autonomous Surface Vehicule}
\acro{AUC}{Area Under the ROC Curve}
\acro{Bm}{Balaenoptera Musculus}
\acro{BCE}{Binary Cross Entropy}
\acro{Bp}{Balaenoptera Physalus}
\acro{CE}{Cross-Entropy}
\acro{CNN}{Convolutionnal Neural Network}
\acro{DAC}{Digital Analog Converter}
\acro{DBSCAN}{Density Based Spatial Clustering of Applications with Noise}
\acro{DCLDE}{Detection Classification Localisation and Density Estimation of marine mammals}
\acro{DEC}{Deep Embedded Clustering}
\acro{DFT}{Discrete Fourier Transform}
\acro{EM}{Expectation Maximisation}
\acro{FFT}{Fast Fourier Transform}
\acro{GMM}{Gaussian Mixture Model}
\acro{GPU}{Graphical Processing Unit}
\acro{HMM}{Hidden Markov Model}
\acro{ICI}{Inter CLick Interval}
\acro{IIC}{Invariant Information Clustering}
\acro{IIR}{infinite impulse response}
\acro{INI}{Inter Note Interval}
\acro{IPI}{Inter Pulse Interval}
\acro{KDE}{Kernel Density Estimate}
\acro{KL}{Kullback-Leibler}
\acro{LTSA}{Long Term Spectral Average}
\acro{mAP}{mean Average Precision}
\acro{MCU}{Microcontroller Unit}
\acro{MFCC}{Mel Frequency Cepstral Coefficients}
\acro{MI}{Mutual Information}
\acro{MSE}{Mean Square Error}
\acro{NAS}{Network-Attached Storage}
\acro{NMI}{Normalised Mutual Information}
\acro{NRKW}{Northern Resident killer whales}
\acro{NRW}{Northern Right Whale}
\acro{PAM}{Passive Acoustic Monitoring}
\acro{PCEN}{Per-Channel Energy Normalisation}
\acro{PLC}{Power Law Coefficient}
\acro{PR}{Precision Recall}
%\acro{PSD}{Power Spectral Density}
\acro{ReLU}{Rectified Linear Unit}
\acro{RNN}{Recurent Neural Networks}
\acro{ROC}{Reveiving Operating Characteristics}
\acro{SGD}{Stochastic Gradient Descent}
\acro{SOOS}{Southern Ocean Observing System}
\acro{SOTA}{State Of The Art}
\acro{SNR}{Signal to Noise Ratio}
\acro{SSL}{Self Supervised Learning}
\acro{STFT}{Short Term Fourier Transform}
\acro{SVM}{Support Vector Machine}
\acro{TDOA}{Time Difference Of Arrival}
\acro{TK}{Teager-Kaiser}
\acro{UDA}{Unsupervised Data Augmentation}
\acro{UI}{User Interface}
\acro{UMAP}{Uniform Manifold Approximation and Projection}
\acro{VGG}{Visual Geometry Group}
\end{acronym}


% The Roman pages, like the Roman Empire, must come to its inevitable close.
\end{romanpages}
\adjustmtc

%----------------------------------------------------------------------------------------
%	ADD CHAPTERS AND APPENDIX HERE
%----------------------------------------------------------------------------------------

%%%%% CHAPTERS
% Add or remove any chapters you'd like here, by file name (excluding '.tex'):
\flushbottom
% intro.tex:

\chapter{Introduction}

\minitoc
\section{Cetaceans and acoustics}

Despite a first appearance on land, some mammalian species are now commonly found in planet Earth's oceans, forming the marine mammals group. Families evolving from 3 distinct orders have physiologically evolved to thrive in the marine environment: Pinnipeds (\textit{Carnivora}, e.g. seals and walruses), Sirenians (\textit{Afrotheria}, e.g. manatees), and Cetaceans (\textit{Cetartiodactyla}, e.g. whales and dolphins). The following thesis will focus on the latter. Cetaceans are classified into two suborders: Odontocetes (toothed cetaceans, such as dolphins, orcas or sperm whales) and Mysticetes (baleen cetaceans, such as blue whales or humpback whales, see Fig.~\ref{fig:evolution}).

\begin{figure}[!htb]
   \begin{minipage}{0.5\linewidth}
     \centering
     \includegraphics[width=\linewidth]{fig/evolution.png}
   \end{minipage}\hfill
   \begin{minipage}{0.5\linewidth}
     \centering
     \includegraphics[width=\linewidth]{fig/cetaceans.png}
   \end{minipage}\hfill
   \caption{(left) Evolution of the marine mammals. (right) Cetacean evolutionary relationships (top: Odontocetes, bottom: Mysticetes). Both figures are taken from \citet{whitehead2015cultural}.}\label{fig:evolution}
\end{figure}

Returning to the sea some 50 million years ago \cite{whitehead2015cultural}, cetaceans now show a complete adaptation to their marine environment, with their powerful flukes, streamlined body, and nostrils displaced on top of their head (allowing for efficient breathing while swimming). Another important adaptation, especially relevant to this study, is the development of their acoustic capabilities, both as emitters and as receivers. Indeed, light typically fades out after a few dozen meters in water, which makes of vision a quite limited sense. In contrast, the higher density of water (compared to air) makes sound travel faster and further. Cetaceans make use of this property to communicate and/or echolocate up to great distances. Blue whale calls can be heard 200km away \cite{vsirovic2007blue}, and sperm whales are able to detect a 1m object at 470m \cite{ferrari2020study}.


\subsection{Echolocation}

One of the uses cetaceans make of underwater acoustics is echolocation. Alike an active sonar, emitting a sound and measuring how it bounces back to you (its echo) allows to sense distance from surrounding objects, their shape \cite{pack1996dolphins}, or even their texture \cite{grunwald2004classification} (Fig.~\ref{fig:echolocation}). Bats use echolocation to navigate and hunt in dark caves, odontocetes use echolocation in a similar way underwater. 

Short impulse like sounds commonly named `clicks' (transitory waves) are mostly associated with echolocation purposes \cite{au2000echolocation}. However, there is not one single type of click used for echolocation: it might coincide with habitats and feeding behaviours \cite{ketten1994functional}. Using short duration clicks, more can be sent in a small period of time without them mixing up, thus increasing the potential temporal resolution of the echolocation. This is typically suited for hunting at high speeds, like small odontocetes do. On the other hand, clicks at lower frequencies will travel further, and thus would be more suited for hunting from long distances like sperm whales do (extremely high Kogia clicks go against this hypothesis).

Finally, despite the old consensus that only odontocetes echolocate with their high frequency clicks, new studies suggest that mysticetes might also make use of their low frequency signals as sonars \cite{iii2018sonar}.

\begin{figure}
    \centering
    \includegraphics[width=.7\linewidth]{fig/echolocation.pdf}
    \caption{Illustration of the dolphin echolocation mechanism for hunting purposes (image credit: Uko Gorter - American Cetacean Society).}
    \label{fig:echolocation}
\end{figure}


\subsection{Communication}

The second major use of sound by cetaceans is communication, a broad concept that can be divided into two main categories: song and social communication systems \cite{janik2014cetacean}.


\subsubsection{Song}

The term song has been first used for cetacean signals by \citet{payne1971songs}, listening to humpback whales whose vocalisations met the following definition: "a series of notes, generally of more than one type, uttered in succession and so related as to form a recognisable sequence or pattern in time”. Similarly to bird songs, they have shown a role in reproductive behaviours: mostly males are observed singing, during the reproductive season, potentially to attract females, fend off other males, or a combination of both \cite{darling2001interactions, smith2008songs}. Songs usually come in strictly patterned sequences, shared by whole species or communities \cite{whitehead2015cultural}. Among cetaceans, they have yet been observed only in mysticetes, with the most renowned one probably being the humpback whale song.


\subsubsection{Social communication}

On the other hand, communication is also observed in odontocetes social groups. Alike in songs, these signals are patterned vocalisations, some of which being identified in discrete categories \cite{ford1987catalogue, weilgart1993coda}. However, they are not restricted to reproductive contexts, and appear in relatively less deterministic sequences. The term song therefore seems less appropriate for this phenomenon, which is rather associated with social bonding functions \cite{schulz2008overlapping, ford1989acoustic}.
% vocal learning ? dolphins learning sentences\cite{herman1984comprehension}.

In most cases, these vocal signals occur with tonal, whistled or pulsed calls. Their associated categories (`call type') are commonly defined by characteristics on their time / frequency contour. As an exception, sperm whales produce clicks in stereotyped rhythmic sequences (named codas) that were also attributed to communication purposes \cite{weilgart1993coda}. It is however not excluded that other odontocetes use clicks as means of communication, but no similar stereotyped sequences have yet been observed among them.


\subsection{Culture}

The term culture is often encountered when describing cetacean communication systems \cite{filatova2015cultural, garland2020cultural, rendell2003vocal}. It seems appropriate to describe the vocal divergences observed between cetacean communities. In a broad sense, culture is defined as `behaviour or information shared within a community, that is acquired from conspecifics through some form of social learning' \cite{whitehead2015cultural}. %Culture is especially relevant as a community marker, helping to identify close relatives within the species \cite{todo}.
In cetaceans, it takes form as specialisation in diets or hunting techniques (e.g. with orcas) or as specific vocal patterns. For instance, sperm whale codas \cite{rendell2003vocal} (Fig.~\ref{fig:codas}), orca stereotyped calls \cite{deecke2000dialect}, humpback whale songs \cite{garland2011dynamic} (Fig.~\ref{fig:HB_culture}) or fin whale pulse sequences \cite{castellote2012fin}, are all community specific, some evolving through the years, and thus are described as cultural phenomenons. This is only possible thanks to the vocal production learning capacity that cetaceans demonstrate \cite{janik2014cetacean}, a relatively rare characteristic among mammals.

\begin{figure}[!htb]
   \begin{minipage}{0.5\linewidth}
     \centering
     \includegraphics[width=\linewidth]{fig/codas_map.jpeg}
   \end{minipage}\hfill
   \begin{minipage}{0.5\linewidth}
     \centering
     \includegraphics[width=.6\linewidth]{fig/codas.jpeg}
   \end{minipage}\hfill
   \caption{(left) Location of two sperm whale communities. (right) Differences in codas patterns (dialects) between the two communities (K: Kumano coast, O: Ogasawara Islands) . Figures are taken from \citet{amano2014differences}}\label{fig:codas}
\end{figure}


\subsection{Human activity impacts}

In the twentieth century, close to 3 million large whales were caught by whalers \cite{rocha2014emptying}. Seeing some whale species coming close to extinction has motivated a large majority of the international community to cease commercial whaling in the late 20th century. However, cetaceans are still heavily impacted by human marine activities in numerous ways (Fig.~\ref{fig:human_impact}). We will focus here on the ones related to acoustics.

There exist a wide variety of anthropogenic acoustic disturbances in the marine environment, which has triggered the development of a new fields of research focusing solely on ambient noise levels \cite{merchant2012averaging}. Marine traffic, seismic surveys using airguns (often to search for oil patches), pile driving (for marine constructions such as offshore wind turbines), military sonars and explosive tests are the most widespread, with several consequences on cetaceans.

We hear better in a silent environment. This implies the first consequence of acoustic disturbances: acoustic masking. With increasing ambient noise levels, the hearing capacities of cetaceans decrease, thus hindering their ability to communicate, hunt, and navigate \cite{erbe2016communication}. More generally, dense marine traffic has also been shown to cause stress to some cetaceans species \cite{rolland2012evidence}. 

The second main consequence is acoustic impairment: temporary or permanent injuries of the hearing apparatus. Powerful sounds such as those emitted by airguns or military tests have been shown to cause deafness in some cetaceans, sometimes leading to mass strandings \cite{parsons2017impacts}.

Eventually, arising from a dense marine traffic, presumably combined with disorientation due to acoustic masking, the collision problem has also attracted the attention of the cetacean conservation community. Especially affecting large mysticetes (e.g. fin whales or right whales), records of death from collisions with boats show a significant impact on whale populations \cite{rockwood2017high}, motivating measures to mitigate collision risks.

\begin{figure}
    \centering
    \includegraphics[width=.8\linewidth]{fig/human_impact.pdf}
    \caption{Evolution of the worldwide sperm (top) and fin (bottom) whale populations and the main human-induced direct mortality threats. The threats are expressed in relative value. This figure is taken from \citet{sebe2020interdisciplinary}.}\label{fig:human_impact}
\end{figure}


\section{Passive Acoustic Monitoring of cetaceans}

To reveal the aforementioned complexity and diversity of cetacean's uses of acoustics, scientists also have put forward their hearing sense. \ac{PAM} is a field of bioacoustic studies that combines several scientific and technical domains, from electronics for recording hardware, to signal processing and statistical analysis. The term passive refers to the notion of listening from a distance, without interfering with the animals, as opposed to active sonar systems or attaching acoustic tags to the animals. The analysis of the cetaceans acoustic activity is providing important insights on their behaviour, population dynamics, social structures or even physiology.


\subsection{Comparison of acoustic and visual surveys}

Besides PAM, visual surveys is the second main approach to the biological study of cetaceans. Each comes with its pros and cons. The acoustic approach enables long term surveys at relatively low costs: placing a fixed antenna allows to monitor biological activities for several consecutive months, requiring human intervention only for the installation and extraction of the recording system. In contrast, the visual approach demands a continuous human implication throughout the survey, in the relatively inaccessible marine environment.

In terms of detection capacities, cetaceans can be heard from great distances (up to 200km for the blue whale \cite{vsirovic2007blue}), even during deep dives, while they can be visually detected from relatively short distances (around 1km, depending on weather conditions) only when surfacing (less than a third of the time for sperm whales \cite{watwood2006deep}). However, species had first to be classified visually before we could learn on their associated acoustic behaviour, and photo identification is still to this day the only reliable way to recognise individuals. Moreover, the observation of group sizes, behaviour, and body conditions still mostly relies on vision. The two approaches thus really are complementary.


\subsection{Antenna types}

\ac{PAM} starts by placing hydrophone(s) (underwater microphones) to listen or record the acoustic environment. They can be fixed on the sea floor (bottom mounted), to a buoy (sonobuoy), to a cable towed by a boat (towed array), or directly to the hull of a boat or \ac{ASV} (Fig.~\ref{fig:sphyrna}). When recording with multiple synchronised hydrophones, one can also triangulate (infer the position of) sound sources, by measuring their \acp{TDOA} for instance. The types of recording devices, their placement in the water column, and their layout between each other have crucial impacts on the yielded recordings, facilitating or not the following signal processing analysis.

For that matter, when implementing acoustic recording systems, one has to make compromises. Indeed, the functioning time of a recorder is limited by its available resources (battery power and data storage). On the other hand, settings that allow for a more detailed view of the acoustic scene (increased number of channels, sampling frequency and/or the byte depth) also imply a higher rate of consumption of these resources.

\begin{figure}
    \centering
    \includegraphics[width=.6\linewidth]{fig/sphyrna.png}
    \caption{Example of a multi-hydrophone antenna mounted on an \ac{ASV}, taken from \citet{poupard2019real}.}
    \label{fig:sphyrna}
\end{figure}


\subsection{PAM for biological studies}

Once recordings have been collected, the first step of their analysis typically comes down to the detection and classification of cetacean vocalisations. The amount of detection through time in long term surveys already provides significant information on the animals' lives. From these, one can infer population density \cite{thomas2012passive} and seasonal or dial presence patterns \cite{bombyx}. When combining several antennas, a spatial dimension can be also integrated.

The analysis of the detected signals can then bring further knowledge on the recorded animals, such as community membership, current behaviour (hunting, socialising, courting), and individual's characteristics (e.g. sexual maturity or body size for sperm whales \cite{bombyx}). These measures can themselves be put in a space-time perspective, potentially revealing patterns. In this way, \ac{PAM} becomes useful to cetacean ethology and stock structure assessment.

A second field \ac{PAM} provides to is the study of animal communication systems. Indeed, cetaceans represent a significant part among the vocal learning species (along with birds, bats, seals, elephants, mice and primates). Identifying patterned sequences and associating them with species, communities and/or behaviours yields exemplary data on the development of vocal interaction in the animal kingdom \cite{fitch2006biology}. Moreover, acoustic behaviour studies revealing cultural differences has provided knowledge on population dynamics \cite{owen2019migratory} (Fig.~\ref{fig:HB_culture}) and social structures \cite{gero2016individual}. There is therefore a great diversity of biological wonders that \ac{PAM} contributes to unveil.
%Again, \ac{PAM} has provided exemplary data on phenomenons such as cultural evolution and transmission, helping advance on these open research questions.

\begin{figure}
    \centering
    \includegraphics[width=.5\linewidth]{fig/humpback_song.jpg}
    \caption{Example of passive acoustic monitoring findings: long term cultural transmission of humpback whale songs eastward through the Southern Pacific Ocean (each colour represents identified song types). Taken from \citet{garland2011dynamic}.}
    \label{fig:HB_culture}
\end{figure}


\subsection{Cetacean conservation}

Some may question the amount of effort put into cetacean biology studies, considering that knowledge of nature is not in itself a sufficient driver. In that regard, it is to be kept in mind that cetaceans are at the top of the ocean's food web, consequently being significant regulators of their ecosystem as a whole. Moreover, the oceanic ecosystem is not only an important provider of food to humans, but also crucial to breathe (it is responsible for around 70\% of the atmosphere's oxygen production \cite{harris2012phytoplankton}). This field of study thus matters not only for the knowledge of planet earth's animal kingdom, but simply to our long term survival.

\begin{figure}
    \centering
    \includegraphics[width=.8\linewidth]{fig/speed_canada.png}
    \caption{Example of conservation measure in the Gulf of St. Lawrence in Canada \cite{speed_canada}. Reduced speed zones are put in place all year round (red) and seasonally (green) to protect North Atlantic Right Whales.}\label{fig:speed_canada}
\end{figure}

As stated previously, human activities heavily impact cetacean species, putting some of them close to extinction \cite{kraus2016recent}. Considering their aforementioned importance, it seems relevant that we learn how to mitigate this impact and work on cetacean conservation policies. Some regulation measures have already been put in place like the Marine Mammal Protection Act \cite{fisheries2013marine}, speed regulations \cite{fisheries2012environmental} (Fig.~\ref{fig:speed_canada}), and the definition of marine mammal sanctuaries \cite{notarbartolo2008pelagos, tetley2022important}. Monitoring the efficiency and necessity of regulations, as well as maximising their relevance (e.g. habitats and/or seasons of importance) can only be done via the knowledge of the animals, which justifies their study.


\section{Neural Networks and PAM}

\subsection{Automated PAM before Neural Networks}

As previously stated, acoustic detections are needed to carry out long term cetacean surveys. They can be gathered by manually inspecting signals, especially their time-frequency representation (spectrograms). However, this is very costly in human efforts and non-repeatable, which motivated the development of automatic detection mechanisms. With such systems at hand, researchers can seamlessly process months of data to yield results such as spatio-temporal presence statistics.

Detection mechanisms have long been implemented with handcrafted algorithms \cite{gillespie2008pamguard}. They can be sufficient for some use cases, but often come quite limited, as the variety of sounds to detect and potential noises increase. Analysing long streams of data across recording devices and antenna locations demands highly robust systems, for which handcrafted algorithms remain unsatisfactory.

As an analogy, let us consider our ability to recognise our kin by the sound of their voice. Formally describing how to robustly differentiate talking individuals seems nearly impossible, especially in a computer language. However, we know that given a hearing sense and sufficient cognitive capacities, by listening to a voice several times, we acquire the capacity to recognise it. This led the scientific community to start shifting towards machine learning algorithms, which are introduced in the following section.


\subsection{Artificial neural networks}

Training \acp{ANN} is the chosen approach for the automation of \ac{PAM} throughout this thesis. It is one of the most popular techniques of machine learning, a field of computer sciences that approaches problem solving without programming solutions explicitly. Specifically, in machine learning, the algorithm is designed to find (or learn) a solution to a problem, often formulated as a mathematical framework. An analogy could be made that genes encode a brain structure for it to learn but genes do not encode knowledge directly. Similarly, in machine learning, a learning framework is programmed but the task's solution is to be learnt.

\acp{ANN} represent a major branch of today's machine learning, solving tasks in computer vision and speech recognition with performances and robustness highly superior to that of traditional handcrafted algorithms. This motivated the application of \acp{ANN} to the field of \ac{PAM}, making it the central topic of this thesis as described in the following section.


\section{Thesis objectives}

This thesis was co-financed by the GIAS European project, aiming at improving navigation security in the Mediterranean sea (western bassin). It takes part in one axis of this project: the mitigation of whale-ship collision risk. For that purpose, a `smart bioacoustic buoy' was designed, with the intent to acoustically detect large cetaceans of the zone (sperm whales and fin whales). Then, alerts can be transmitted close to real-time, for ships to adapt their speed or route accordingly.

Being a thesis in computer science, its goal is to design and implement the acoustic detection algorithms embedded in the buoy (collaborating with third parties on the hardware development). To this end, motivated by the recent advances in the field, the \ac{ANN} approach was chosen.

The work of training \acp{ANN} for the detection of cetacean vocalisations quickly expanded well beyond the initial needs of the GIAS project. Indeed, the team participates in a variety of projects (see section \ref{chap:data_Toulon}). In each of them, our role is typically to analyse large amounts of recordings to advance on biological questions. Hence the need for cetacean acoustic detection and classification mechanisms. Moreover, the performance demonstrated by \acp{ANN} in early experiments motivated to use them extensively on other species. This is how the objective of this work dissociated from the GIAS implementation to become a general study of applying \acp{ANN} to cetacean acoustic detection and classification.


\subsection{Structure of the manuscript}

\begin{figure}
    \centering
    \includegraphics[width=\linewidth]{fig/thesis_flowchart.pdf}
    \caption{Flowchart of the typical process when using \acp{ANN} for bioacoustics. The main steps covered by this thesis are shown by arrows with their associated chapters. If a model is already available for the target signal, the two first steps can be skipped.}
    \label{fig:thesis_flowchart}
\end{figure}

This document is organised as follows. First, chapter \ref{chap:sota} introduces the \ac{SOTA} of the deep learning techniques that will be used in this study, along with their precedent use for automated \ac{PAM}. Then, chapter \ref{chap:material} will go through the species of interest for this work, the signals they emit, and the recordings available. The rest of the manuscript then revolves around the three main steps needed to address \ac{PAM} with \acp{ANN} (Fig.~\ref{fig:thesis_flowchart}). Chapter \ref{chap:annotation} starts with the construction of training databases, describing annotation procedures suited for a variety of constraints (depending on the recordings at hand and the target signals). Then, to train \acp{ANN} on these databases, chapter \ref{chap:training} describes architectures and frameworks to yield robust detection and classification mechanisms (this time depending on constraints of computational power and target signals). Finally, for some of the trained models, chapters \ref{chap:conservation} and \ref{chap:com} illustrate their applications around two main uses: species conservation (for Mediterranean fin whales and sperm whales) and communication modelling (for fin whales and orcas).
\chapter{State of the art}\label{chap:sota}
\minitoc

The following chapter introduces the main technical aspects relevant to the subsequent work, lying between pedagogic and bibliographic objectives. It starts with the main techniques involved in building and training \acp{ANN}, within their most prevalent context in the literature (computer vision). Then,I review cetacean \ac{PAM} automation (in general and using \acp{ANN}). Finally, this thesis is put in perspective w.r.t its field of application.


\section{Neural networks for computer vision}

If computer vision techniques can be used to tackle acoustic tasks, it is in part because sound can be represented as time-frequency images such as spectrograms. They describe content such that vocalisations appear as patterns with identifiable shapes, in a way similarly to our hearing system which processes sound via frequency decomposition with the cochlea. We will therefore first go through the \ac{SOTA} in image pattern recognition before applying these methods to our acoustic tasks. This is obviously not an exhaustive review of deep neural networks, but rather an overview of the key elements used in this thesis to build detection and classification systems.


\subsection{Introduction to Artificial Neural Networks}

The idea of emulating brain neural systems computationally emerged in the mid 20th century \cite{fitch1944mcculloch}. It is however only recently that \ac{ANN}s have taken such an important part in applied mathematics and computer sciences, with the increasing availability of data and computational power. The underlying approach to \acp{ANN} is to reproduce advanced processes emerging from the accumulation of simple operations, alike brains with neurons. Put mathematically, neurons would typically take the form of a simple linear transformation of an input $x$ into an output $y$ (\(y = wx + b \)). With their combination into large networks emerges the capacity of modelling high level functions such as classifying cat and dog images for instance.

An \ac{ANN} is defined by a network architecture (interconnection of neurons) and its neurons' weights (the linear transformations' coefficients, namely \(w\) and \(b\)). Like so, we can formulate an \ac{ANN} model as a function $g$ (the composition of linear layers $l_{\boldsymbol{\theta}_i}$) and the concatenation of all its weights $\boldsymbol{\theta}$ (Eq.~\ref{eq:net}). Stacking together a large number of layers gave the appellation `deep learning'.

\begin{equation}\label{eq:net}
    g_{\boldsymbol{\theta}}(\mathbf{x}) = l_{\boldsymbol{\theta}_1} \circ l_{\boldsymbol{\theta}_2} \circ l_{\boldsymbol{\theta}_3} \circ ... l_{\boldsymbol{\theta}_n} (\mathbf{x})
\end{equation}

We first design an architecture $g$ before optimising its weights $\boldsymbol{\theta}$ for our task, typically with supervised learning. This paradigm consists in feeding the model examples with their associated labels. For instance with our cats and dogs task, this means giving the model images of each class and asking it to predict the associated label, namely `cat' or `dog'. An error $\mathcal{L}$ (called loss) is then computed between the expected and the predicted labels. Like so, the training objective can be formulated as Eq.~\ref{eq:optim} to find the weights $\boldsymbol{\hat{\theta}}$ that minimise the error $\mathcal{L}$.

\begin{equation}\label{eq:optim}
    \boldsymbol{\hat{\theta}} = \underset{\boldsymbol{\theta}}{\argmin}\ \mathcal{L}(\mathbf{y}, g_{\boldsymbol{\theta}}(\mathbf{x}))
\end{equation}

Under the hood, the network learns a projection of the input images (called embedding) from the pixel space to a new abstract one. Put simply, the more neurons in a network, the more complex the resulting projection can be. Training becomes trying to learn the optimum embedding space to solve a given task.

There are two main limitations here, the first being the necessary computational power. Training a large \ac{ANN} typically demands thousands of iterations, each of which consists in an update of millions of neurons. This is in part why we had to wait for the development of parallel computation with \acp{GPU} to see the democratisation of \acp{ANN}. The second limitation, this time a human effort cost, is the necessary training data. To learn a robust solution, training typically demands thousands of examples for each class, with their associated label (often manually annotated) for the computation of the loss that will be optimised.

\begin{figure}
    \centering
    \includegraphics[width=.8\linewidth]{fig/cats and dogs.pdf}
    \caption{Illustration of the concepts of underfitting and overfitting, for the cats and dogs classification task. Lines denote discrimination boundaries, in a two-dimensional abstract embedding space.}
    \label{fig:catsanddogs}
\end{figure}

This leads us to the major challenge of training \ac{ANN}s and modelling in general: robustness, or generalisation. Optimising a performance metric on a limited amount of examples might bring the curse of overfitting: when the model finds a solution that works for its given training data, but not the generalised solution that we desire (Fig.~\ref{fig:catsanddogs}). As an example, coming back to the cats and dogs task, if all the cats we show the \ac{ANN} are white and all dogs are black, it might just discriminate based on average pixel colours. This will lead to great performances on the training data, but will fail as soon as we try our \ac{ANN} on a black cat image. As we will see throughout this thesis, most of the struggle in training \ac{ANN}s comes down to enforcing generalised solutions with limited training data.

%\section{Convolutional Neural Networks}
%In the following section, we will dive deeper into the category of \acp{ANN} that was used through this thesis: \acp{CNN}. I will start by describing its building blocks (layers of neurons), then going through the typical architectures they are put together in, and ending with how to iteratively optimise its neurons' weights while avoiding overfitting. I will use the original use case of \acp{CNN} for demonstration here (image classification), before joining in on this thesis' tasks in the next section.



\subsection{Performance optimisation}

As previously mentioned, training \acp{ANN} comes down to trying to find the optimum weights for a task. This optimisation takes form as the minimisation of some loss function (Eq.~\ref{eq:optim}). This section describes the methods involved in optimising this loss, especially with \ac{SGD}. Then, the different loss functions that will be needed in this thesis are introduced. Finally, we will go through the `second level' of performance estimation and optimisation, employed to account for architecture and training quality once weights have converged.


\subsubsection{Optimising the loss}

Depending on our task and label availability, let's consider a differentiable loss $\mathcal{L}$ to be minimised. A straightforward way of finding some function's minima is to follow the slope downwards iteratively (``gradient descent''). Furthermore, having multiple data points to account for in the computation of the loss, a stochastic estimate of the gradient can be used. This is the approach followed by the \ac{SGD} algorithm \cite{robbins1951stochastic} to iteratively update model weights as expressed by Eq.~\ref{eq:sgd}. The amplitude of the update is defined by the learning rate $\alpha \in [0,1]$.

\begin{equation}\label{eq:sgd}
    \boldsymbol{\theta}^{(i+1)} = \boldsymbol{\theta}^{(i)} - \alpha \times \mathbb{E}	_{\mathbf{x}, \mathbf{y}}[\nabla_{\boldsymbol{\theta}} \mathcal{L}(\mathbf{y}, g_{\boldsymbol{\theta}}(\mathbf{x})].
\end{equation}

The choice of learning rate is critical to achieving convergence of the model's parameters. Indeed, a too small learning rate might result in getting stuck in a local minima, whereas with a too large one the actual minima might be skipped and the procedure may diverge. No generic learning rate is good for every task, so it will be one of the hyper-parameters to be tuned (see section \ref{chap:HP_tuning}).

To enhance convergence quality and speed, the community is now opting for learning rates that evolve through the course of the optimisation. This evolution (termed learning rate scheduling) can be a simple exponential decay, a decay when the loss plateaus, or more advanced periodic schedules with warm restarts \cite{loshchilov2016sgdr}. No definite agreement has yet been made on the right schedule, and the answer might again be task specific.

Additionally, the data used to compute the loss and update weights at each step needs to be defined: it is called a batch. Using the whole dataset at each step would be too costly in memory and computation, and using only one sample would hardly converge (the gradient would oscillate in different directions). Mini-batch \ac{SGD} thus consists in using only a sub-sample of the available data at each weight update. In a compromise between computation cost and each batch being representative of a global direction to follow, a ``batch size'' (number of samples per batch) needs to be defined. It is also part of the hyper-parameters to be tuned (see section \ref{chap:HP_tuning}).

Methods like \ac{SGD} to iteratively update the model's parameters depending on the loss gradient are called optimisers. Several variations of \ac{SGD} have been proposed since its original formulation, especially via smoothing the gradient across batches. The nesterov momentum \cite{sutskever2013importance} as well as the gradients' moments \cite{kingma2014adam} serve that purpose.% Other techniques such as batch normalisation also indirectly work towards gradient smoothing to ease the optimiser's work \cite{santurkar2018does}.


\subsubsection{Classification and detection losses}

Because it will be needed for \ac{SGD}, the chosen loss to optimise needs to be convex and differentiable. For our classification tasks, the accuracy is therefore not suitable since it relies on the $\argmax$ of the output vector. We will rather choose the \ac{CE} instead, and will keep the accuracy for model evaluation, selection and validation (section \ref{chap:valid}).

The \ac{CE} is a mean of measuring the agreement between two vectors, being here the class prediction and target label. The definition of the \ac{CE} classification loss $\mathrm{H}$ is given in Eq.~\ref{eq:CE}, with $\mathbf{y}$ the one-hot encoded label\footnote{Vector of zeros except for the true class which is one.}, $\mathbf{\hat{y}}$ the vector of predicted probabilities for each class, and $C$ the set of possible classes.
%
\begin{equation}\label{eq:CE}
    \mathrm{H}(\mathbf{y}, \mathbf{\hat{y}}) = -\sum_{c\in C}y_c \log(\hat{y}_c).
\end{equation}
%
To get normalised predictions of the model homogeneous to a probability distribution (summing to 1), we use the SoftMax function described in Eq.~\ref{eq:softmax}, given the unnormalised model output $\mathbf{z}$ (also called logits).
%
\begin{equation}\label{eq:softmax}
    \hat{y}_c = p_{g,\boldsymbol{\theta}}(x|c) = \mathrm{SoftMax}(\mathbf{z})_c = \frac{e^{z_c}}{\sum_{k}e^{z_k}} % \in [0, 1], \;    \sum_{c} p_{g,\boldsymbol{\theta}}(x|c) = 1.
\end{equation} 
%
This is appropriate for multi-class classification tasks, when a higher confidence for a class implies lower probabilities for others. When solving multi-label classification tasks however, a sample can be assigned multiple classes, making the SoftMax assumption not appropriate. We then rather use the Sigmoid function to normalise logits to probability like values (Eq.~\ref{eq:sigmoid} and Fig.~\ref{fig:sigmoid}), and the sum of the independent \acp{BCE} as a loss.
%
\begin{equation}\label{eq:sigmoid}
    \mathrm{Sigmoid}(z_c) = \frac{1}{1+e^{-z_c}}.
\end{equation}
%
The \ac{BCE} is simply a special case of the \ac{CE}, with $C=2$. However, we can use single valued labels $y$ and predictions $\hat{y}$ for its computation (Eq.~\ref{eq:BCE} and Fig.~\ref{fig:bce}).
%
\begin{equation}\label{eq:BCE}
    \mathrm{BCE}(y, \hat{y}) = - y\log(\hat{y}) - (1-y)\log(1-\hat{y}).
\end{equation}
%
\begin{figure}[!htb]
   \begin{minipage}{0.5\textwidth}
     \centering
     \includegraphics[width=\linewidth]{fig/sigmoid.pdf}
     \caption{Sigmoid function (Eq.~\ref{eq:sigmoid}).}\label{fig:sigmoid}
   \end{minipage}\hfill
   \begin{minipage}{0.5\textwidth}
     \centering
     \includegraphics[width=\linewidth]{fig/BCE.pdf}
     \caption{\ac{BCE} loss (Eq.~\ref{eq:BCE}).}\label{fig:bce}
   \end{minipage}\hfill
\end{figure}

Through this thesis, the binary classification will be used as a proxy to solve detection tasks (one class being the target event to detect, and the other anything else). When running a classifier model post training, the predicted class will be $\argmax(\mathbf{z})$. For binary classifiers however, the output becomes a single value denoting the confidence in the presence of one class, equivalent to a detection confidence. A threshold is then set to binarise this continuous value (yielding a presence/absence decision).


\subsubsection{Representation learning losses}\label{chap:ssl}

The losses previously mentioned are suited when a sufficient amount of labels are available for supervised learning. When few or no labels are available, the literature proposes frameworks to learn semantically relevant embedding spaces, used subsequently by clustering algorithms or in supervised fine tuning. We call this process deep representation learning. Since this learning paradigm does not rely on labels for optimisation, it is referred to as \ac{SSL}.

\paragraph{Triplet loss and contrastive learning} \label{chap:triplet}
Contrastive learning is a branch of \ac{SSL} algorithms in which we enforce a models' embedding to ignore transformations applied to the input (transformations that do not imply a semantic change to the data). For that purpose, we will minimise the distance between the projection of a sample and that of its transformation w.r.t. the projection of other samples (Fig.~\ref{fig:contrastive}). In this way, rather than directly learning an embedding space for discrimination, the model is trained to learn an embedding space that reflects a desired notion of similarity and difference (the contrast). The mathematical formulation of this objective is termed as triplet loss since it uses the projection of three samples: an anchor (the original sample), a positive (the transformation of the anchor), and a negative (another unrelated sample). Several metrics have been used in the literature to measure distances between embeddings :
%
\begin{itemize}\setlength{\itemsep}{1pt}
    \item The cosine similarity (SimCLR \cite{chen2020simple})
    \item The cross-entropy (\ac{UDA} \cite{xie2019unsupervised}, fixMatch \cite{sohn2020fixmatch})
    \item The cross-correlation (Barlow \cite{zbontar2021barlow}) 
    \item The mutual information (\ac{IIC} \cite{ji2018invariant})
\end{itemize}

\begin{figure}
    \centering
    \includegraphics[width=.9\linewidth]{fig/contrastive learning.pdf}
    \caption{Illustration of the contrastive learning approach. The anchor and the negative are randomly sampled from the database, whereas the positive is a hand crafted transformation of the anchor. The distance metric to be minimised/maximised varies among implementations.}
    \label{fig:contrastive}
\end{figure}

These contrastive losses can also be combined with a regular classification loss in a semi-supervised paradigm, as seen in fixMatch and \ac{UDA} for instance. They can then be considered as a form of training regularisation (see section \ref{chap:regularization}).

%\paragraph{Triplet loss and Siamese neural networks}
%In a similar fashion than with contrastive learning, the triplet loss can be used in a supervised context. In this case, the positive of the triplet is a sample drawn from the same class as the anchor, and the negative is a sample from another class. We call this approach Siamese networks \cite{bromley1993signature, koch2015siamese}. Despite its use of labelled samples alike regular supervised classification training, this method focuses on learning an embedding space to measure samples' similarity, rather than an embedding space to discriminate among classes.

\paragraph{Reconstruction loss}\label{chap:reconstruction}
In other \ac{SSL} frameworks such as \acp{AE} we ask models to reconstruct an input image (see section \ref{chap:AE}). We will then use a reconstruction loss which reflects the fidelity of the reconstructed sample w.r.t. the original input. This can simply take the form of a \ac{MSE} between the input and the reconstructed image (pixel loss). There are also more advanced approaches such as the perceptual loss which uses the \ac{MSE} in the latent space of an independently trained encoder to yield a higher level comparison \cite{johnson2016perceptual}.


\subsubsection{Model validation}\label{chap:valid}

Once our model has optimised the chosen loss function until convergence, we usually want to measure its performance with interpretable metrics, and with new data.

\paragraph{Performance validation metrics (detection)}
For detection tasks, which are the most common in this thesis, these metrics reflect the proportion of target signals that we won't miss (recall) and the proportion of detections that will be the signal we look for (precision). This is typically described via the areas under the \ac{ROC} and \ac{PR} curves. For varying thresholds, they give average values of recall/fall-out and precision/recall respectively. Note that the area under the \ac{ROC} and \ac{PR} curves will be referred to as \ac{AUC} and \ac{mAP} respectively. Equations \ref{eq:AUC} and \ref{eq:mAP} formulate their computation with $rec$, $prec$, and $fal$ denoting recall, precision and fall-out respectively. $TP$, $P$, $PP$, $FP$, and $N$ denote numbers of true positives, positive ground truths, positive predictions, false positives, and negative ground truths respectively. Some are a function of a threshold noted $\lambda$, used to binarise continuous prediction values.

\begin{equation}
    rec(\lambda) = \frac{TP(\lambda)}{P}, \quad prec(\lambda) = \frac{TP(\lambda)}{PP(\lambda)}, \quad fal(\lambda) = \frac{FP(\lambda)}{N},
\end{equation}
\begin{equation}\label{eq:AUC}
    AUC = \int_{0}^{1} rec(\lambda) \; dfal(\lambda),
\end{equation}
\begin{equation}\label{eq:mAP}
    mAP = \int_{0}^{1} rec(\lambda) \; dprec(\lambda).
\end{equation}

These two last metrics are similar, but differ on the measurement of false alarm rate: the \ac{mAP} normalises on the number positive predictions whereas the \ac{AUC} normalises on the number of negative samples. This difference has a significant impact especially with imbalanced datasets.


\paragraph{Performance validation metrics (classification)}
For multi-label classification tasks (each sample can be assigned multiple classes), we will average the independent detection performance of each class. As for multi-class classification (each sample is assigned to a single class), we will rather compute the accuracy: the rate of correct predictions. Averaging methods for the performance metric should be chosen to account for class imbalance or not (i.e. averaging the performance per class before averaging between classes or averaging performances per samples directly).


\paragraph{Performance validation metrics (representation learning)}
Latent representations learnt via \ac{SSL} are intended to reflect semantic similarity. Thus samples' embeddings can be used for clustering (grouping samples by similarity). To measure the relevance of clusters against a set of labels, the \ac{MI} noted $\mathrm{I}(X; Y)$ can be used. It is formulated as the \ac{KL} divergence between the joint and the marginal distributions of labels $X$ and clusters $Y$ (Eq.~\ref{eq:mi}).
%
\begin{equation}\label{eq:mi}
    \mathrm{I}(X;Y) = \sum_{y \in \mathcal{Y}} \sum_{x \in \mathcal{X}}  P_{(X,Y)}(x,y) \log \left( \frac{P_{(X,Y)}(x,y)}{P_X(x)P_Y(y)} \right)
\end{equation}
%
To compute the \ac{NMI} (normalised between 0 and 1), one can divide $\mathrm{I}(X;Y)$ by the average of the entropy of $X$ and $Y$ (Eq.~\ref{eq:nmi}).
%
\begin{equation}\label{eq:nmi}
   \mathrm{NMI}(X;Y) = \frac{\mathrm{I}(X;Y) \times 2}{\mathrm{H}(X) + \mathrm{H}(Y)}
\end{equation}

\paragraph{Validating with new data}
To reduce human effort, we usually desire models to be applicable across different databases. However, \acp{ANN} have the tendency to overfit, showing a decrease in performance on data that differs from those seen in training. In machine learning, to account for this potential overfitting, models' performance are usually measured on new data (not seen in training). It is called the test set, as opposed to the training set which is used for the iterative loss optimisation.

When designing experiments, one must ensure that the test set is significantly disjoint from the training set to relevantly measure the generalisation capacity of models. For instance, in sound event detection tasks, we might want to test our model on recording devices, environments, and emitters that have not been observed during training. How well the model performs facing such domain shifts is the only reliable measure that should be taken into account, especially if we want the model to be reusable in new conditions. Conversely, if a model has been trained and tested on similar data, a large performance drop should be expected as soon as the data changes.  


\subsubsection{Hyper-parameter tuning}\label{chap:HP_tuning}

We went through the iterative optimisation of weights to minimise a loss, but other parameters can also be tuned to enhance performance: the model architecture and the optimiser have numerous settings that need to be set before training. They cannot be optimised via gradient descent alike model weights and have a huge impact in both convergence speed and the found loss minima. We call them hyper-parameters. 

Often, hyper-parameters are tuned to optimise performance on a separate set of data called ``validation set''. Doing so, we keep the test set for the final performance evaluation, and avoid finding hyper-parameters that would be specific to the test set. Throughout this thesis, accounting for the efforts put into having a test set disjoint from the training set and their sufficient size (reducing the probability of hyper-parameter overfitting), the test set was directly used to tune hyper-parameters.

Each training taking at least several minutes on a super computer, the exploration of hyper-parameter combinations to improve model performance is a challenging task. Dedicated algorithms have been proposed to efficiently explore the hyper-parameter space. They combine several principles among which the early stopping of low performing models \cite{li2020system}, as well as muting high performing ones for the next trials \cite{jaderberg2017population}.


\subsection{Layers}

As previously mentioned, the accumulation of layers of neurons (linear transformations) forms the basis of \acp{ANN}' functioning. However, several other types of layers exist. Let us dive deeper into the different layers that will be needed for this thesis, and each of their specific utility.


\subsubsection{Convolution}

Convolution is a mathematical operation that describes the integral of the point-wise product of two functions, with a varying shift on the input variable. It is usually noted with the asterisk symbol (see Eq.~\ref{eq:conv}, given a kernel $f$ of size $M$ and a function $g$).
%
\begin{equation}\label{eq:conv}
    (f*g)[n] = \sum_{m = 0}^{M} f[m] \times g[n-m]
\end{equation}
%
Typically, in image processing, we will use this operator to slide a filter (or kernel) over a larger image. The output of the convolution will be maximal where the filter matches most the image, or in other words where there is the strongest correlation. In 1995, \citet{lecun1995convolutional} introduced the concept of using convolution operators in neural networks; \acp{CNN} were born. 

Before that, pixels where given independently to input neurons. The input image size was thus fixed for a given network architecture, and a displacement of patterns within an image would mean a totally different response of the network. With \acp{CNN}, the network's neurons take the form of kernels (or filters), which are convolved onto input images. Like so, patterns are searched all over the image, independently of their placement.

This behaviour is called spatial invariance and is crucial to pattern recognition in images (e.g. looking for a cat in a picture or a vocalisation in a spectrogram, independently of their placement). This characteristic led \acp{CNN} to become unavoidable in the field\footnote{Let aside the recent rise of transformers for computer vision \cite{parmar2018image}}.

In terms of mathematical definitions, a traditional \ac{ANN} layer is described as $\mathbf{y = Wx + b}$ with $\mathbf{x} \in \mathbb{R}^{in}$ an input vector, and $\mathbf{y} \in \mathbb{R}^{out}$ an output vector. In deep neural networks, the input of a layer is the output of the preceding one (Eq.~\ref{eq:net}). The weights $\mathbf{W}$ and $\mathbf{b}$ are thus matrices defined in $\mathbb{R}^{out\times in}$ and $\mathbb{R}^{out}$ respectively, with $in$ and $out$ being the number of neurons in the preceding and current layers respectively.

As for \acp{CNN}, a layer is no longer composed of a stack of neurons, but rather a stack of kernels. The behaviour of a kernel of width $w_k$ and height $h_k$ is formulated by Eq.~\ref{eq:kernel}, given an input of width $w$, height $h$, and depth $d$ (also called number of features).
%
\begin{equation}\label{eq:kernel}
\mathbf{Y = W*X}+ b, \quad \mathbf{X} \in \mathbb{R}^{h \times w \times d}, \mathbf{A} \in \mathbb{R}^{h_k \times w_k \times d}, b \in \mathbb{R}
\end{equation}
%
The convolution integration (sum) is done over the 3 dimensions, but the shift will occur on the width and height dimensions only, making $\mathbf{Y} \in \mathbb{R}^{h \times w}$. The outputs of each kernel of the layer will eventually be stacked to form the depth dimension for the input of the next layer\footnote{The colour dimension of input images are also put as depth dimension} (see Fig.~\ref{fig:convolution}).

\begin{figure}
    \centering
    \includegraphics[width=.8\linewidth]{fig/conv_layer.pdf}
    \caption{Convolution layer. Blue denotes a slice of the image, a kernel, and the resulting point in the output image (the sum of the point-wise product between the two). The number of kernels will define the depth of the output cuboid.}
    \label{fig:convolution}
\end{figure}

A \ac{CNN} layer is thus defined by the number of input features it processes, its number of kernels, and their width and height. The number of trainable parameters in a layer is given by Eq.~\ref{eq:card_conv}. 
%
\begin{equation}\label{eq:card_conv}
    \# \boldsymbol{\theta} = d_{in} \times w_k \times h_k \times d_{out} + d_{out}.
\end{equation}


\subsubsection{Depth-wise separable convolution}\label{chap:sota_dw}

As presented in the previous section, convolution kernels are cuboids, with a depth that fits the depth of the input. The filters are designed in order to find patterns that are interconnected depth-wise. However, sometimes we might want patterns to be filtered independently through the input image depth, for a subsequent depth-wise combination. This is the idea introduced by depth-wise separable convolutions, first used in the context of a \ac{CNN} by \citet{chollet2017xception}.

In this new type of convolution layer, we dissociate the spatial filtering and the feature combination in two stages, as opposed to regular convolutions that process it all at once. A kernel remains cubic, but the convolutions are separated depth-wise, thus yielding a cuboid, when a regular convolution kernel yields a flat image. The feature combination then happens with the point-wise stage, similar to a convolution with a kernel of width and height 1. This stage can be repeated to obtain an output depth (see Fig.~\ref{fig:depthwiseConv}).

For comparison with the regular convolutions, the number of trainable parameters in a depth-wise separable convolution layer is given by Eq.~\ref{eq:card_dw}.
%
\begin{equation}\label{eq:card_dw}
    \# \boldsymbol{\theta} = d_{in} \times ( w_k \times h_k + 1 + 2\times d_{out})
\end{equation}
%
\begin{figure}
    \centering
    \includegraphics[width=.8\linewidth]{fig/dw_conv_these.pdf}
    \caption{Depth-wise separable convolution. In the depth-wise stage, each depth bin is convolved with its own kernel independently. For the point-wise stage, the depth dimension is combined point by point by various vector kernels, each of which will result in a depth bin in the output.}
    \label{fig:depthwiseConv}
\end{figure}

Having less parameters involved in a network means less computational complexity for inference and for weight updates. Moreover, this type of convolution has shown improved generalisation performances for computer vision tasks \cite{chollet2017xception}. Indeed, limiting feature inter-dependence could limit potential overfitting, alike the dropout \cite{srivastava2014dropout} technique introduced later on.


\subsubsection{Pooling}

As previously mentioned, convolution enables spatial invariance. However, it doesn't treat the scale problem. Indeed, some patterns might have to be detected independently of their scale in input images. Moreover, detecting large patterns would require large kernels, which are expensive in computation and memory. For this purpose, pooling layers enable a progressive decrease in image resolution (on the width and height dimensions), so that deeper layers can have a larger scale view without requiring larger kernels.

%The preferred type of pooling layer is often max pooling, but it can come with any mathematical reduction operation. The end goal is the sub-sampling of the image, or reducing the resolution, while conserving the information that matters. A kernel will be slided over the input, with a stride superior to 1. The amount of stride for each dimension will imply the factor of decrease in resolution.

Often, we want to simply denote if features were activated in a given area, with a lower resolution. Max-pooling layers are well suited for this, simply keeping the maximum value in a window with a $stride>1$ (the stride is the amplitude of sliding windows' steps in pixels). Typically, max-pooling layers are placed after every 2 or 3 convolution layers.% Also, they can come practical when we need to reduce down the spatial dimension to 1, as a last layer to the network for instance.


\subsubsection{Non-linearity layers}

Even if they are spatialised, convolution layers remain a simple linear transformation of the input, and accumulating linear transformations successively is equivalent to a single linear transformation (Eq.~\ref{eq:twolinear}).

\begin{equation}\label{eq:twolinear}
    w_2(w_1x + b_1) + b_2 = (w_2w_1)x + (w_2b_1 + b_2)
\end{equation}

Therefore, building deep networks by accumulating layers of neurons would not add to the complexity the network is able to model. In order to model non-linear functions up to a great complexity, non-linearity layers are needed. Common non-linearity layers are \ac{ReLU} ($y = \max(0, x)$), leaky ReLU, TanH, among others.

Additionally, functions such as \ac{ReLU} insert zeros in numerous dimensions of vectors. This serves the stabilisation of gradients during the optimisation and has an effect of sparsity enhancement (latent representations lie in lower-rank manifolds).


\subsection{Architectures}

\subsubsection{\ac{CNN} encoder}

Before \acp{ANN}, non linear \acp{SVM} \cite{aizerman1964theoretical} were used for a similar purpose: learning the optimum projection of data points to make them linearly separable. Only their approach to optimisation differs. In our case study of \acp{CNN}, we typically want to project an image from the pixel space to a lower dimensional space that embeds semantic content. We often refer to \acp{CNN} as encoders for this projection property.

The projection is usually the last operation of a network, done using linear layers after flattening the image (compression of the width, height, and depth into a single dimension, see Fig.~\ref{fig:vgg}). In the case of classifiers, the dimensionality of the output projection is defined by the number of possible classes, each dimension denoting the confidence for one class.

\begin{figure}
    \centering
    \includegraphics[width=.7\linewidth]{fig/vgg16.png}
    \caption{\acs{VGG}-16 architecture \cite{simonyan2014very}. Dimensions at each layer are given in this order: $height \times width \times depth$ (image taken from \citet{ferguson2017automatic}). The convolutional part is in blue and red, and the projection part is in green.}
    \label{fig:vgg}
\end{figure}


%\subsubsection{Interpretation of the model's output}

%For classifiers, the dimensionality of the output is defined by the number of possible classes for our task. Indeed, each output feature will describe the confidence of the model on the presence of one class in the input. We will thus train our model to, given an input and its associated class(es), maximise the confidence value of the `present' class(es), while minimising those of the `absent' class(es). For the case of binary classification, networks can have either one or two output feature(s)\footnote{Single output models can be seen as detection systems, as we will see through several use cases in this thesis.}.


\subsubsection{Standard architectures}

Some encoder architectures have become standard and are commonly used by the deep learning community: in most cases starting the design of a new architecture from scratch seems unnecessary and counterproductive. Through this thesis, experiments will make use of two types of architectures coming from the ImageNet computer vision benchmark \cite{deng2009imagenet}: \ac{VGG} \cite{simonyan2014very} and ResNet-18/50 \cite{he2016deep}. 
%
\begin{itemize} \setlength{\itemsep}{1pt}
    \item The \ac{VGG} architecture (Fig.~\ref{fig:vgg}) is a `classic' convolutional encoder tailed by fully connected layers.
    \item The ResNet-18 and ResNet-50 architectures are composed of residual blocks, which make use of `skip-connections' (the output of a block is the sum of its processed input and the original input). Their associated number denotes the number of layers that compose them.
\end{itemize}
%
These two types of architectures were chosen as they are (or have been) the baseline in image classification tasks, therefore considered standard \ac{CNN} architectures, even for bioacoustic tasks (see section \ref{chap:PAM_use_cases}).


\subsubsection{Auto-Encoders} \label{chap:AE}

Encoders can serve classification tasks, but they can also take part in other systems such as \aclp{AE}. \acp{AE}, among other applications, may serve tasks of dimensionality reduction for clustering. To enforce the conservation of information while reducing dimensionality, the encoder is followed by a decoder that reconstructs the input image from the low dimensional space (called bottleneck). The encoder and decoder combination (called \ac{AE}) is trained to compress and reconstruct the input most faithfully, despite the low dimensional bottleneck.

The compression that \acp{AE} offer enables a removal of random or unstructured information (denoising), and a lower dimensional space which facilitates clustering (clustering relies on distance estimations that are unreliable in the pixel space and suffer the curse of dimensionality).


\subsection{Training regularisation} \label{chap:regularization}

Methods employed during training to reduce potential overfitting and enhance generalisation are called regularisation. They are especially relevant when a limited amount of training data is available (the case of many bioacoustics tasks). Some of these approaches come down to increasing variability, both in the input and in the activations of the network.


\subsubsection{Data Augmentation}\label{chap:data_augm}

Introducing variability to the input data is widely used to avoid overfitting. The idea is to virtually increase the dataset size without needing more annotation by generating new data samples out of existing ones. To do so, we apply randomised transformations, realistic or not, with the only constraint that we must ensure not to change the sample's class. For image classification, RandAugment \cite{cubuk2020randaugment} might be the most common augmentation policy, combining texture and shape transformations (Fig.~\ref{fig:randAugm}). We will go through data augmentation for acoustic tasks in section \ref{accoustic_augm}.

\begin{figure}
    \centering
    \includegraphics[width=.7\linewidth]{fig/randaugmt.pdf}
    \caption{Using data augmentation enables new samples to be derived from original ones, while conserving the label. Transformations are randomly sampled among several texture and position alterations.}\label{fig:randAugm}
\end{figure}

Another branch of data augmentation worth mentioning is MixUp \cite{zhang2017mixup}, which combines two input samples and their labels, creating ``in-between'' data points. The combination takes form as a simple weighted mean of inputs and labels, which we will feed our model with (like a regular sample). This simple concept of giving mixtures of 2 instances as training samples has shown to improve generalisation in most computer vision tasks with standard architectures \cite{zhang2017mixup}.


\subsubsection{Within-network regularisation}

By introducing perturbations and variability within the network, we can mitigate its dependency to highly specific events, presumably increasing its robustness. Dropout \cite{srivastava2014dropout} follows that incentive by randomly deactivating neurons or kernels (putting their activation to 0). The probability of discarding is defined by the dropout hyper-parameter $p$, commonly set to 0.25.

A second common way to regularise the network while training is to enforce the model to rely on as few weights as possible \cite{krogh1992simple}. To do so, we introduce a new term in the loss: the $L_2$ norm of all parameters, weighted to control its influence on weight updates. We call this method weight decay, and its weight introduces another hyper-parameter to the learning framework.


\subsubsection{Leveraging unlabelled samples}\label{chap:fixMatch}

As seen in section \ref{chap:triplet}, contrastive losses can be used to train encoders for resilience to data augmentation. Several algorithms have been published to incorporate this, often termed as consistency training. Table \ref{fig:perf_fixmatch} summarises their performances on semi-supervised learning datasets, with varying proportions of labelled samples. The fixMatch algorithm \cite{sohn2020fixmatch} combines a supervised loss, pseudo labelling, and consistency training in one framework to achieve \ac{SOTA} performances for the tasks with the fewest labels.

\begin{figure}
    \centering
    \includegraphics[width=\linewidth]{fig/table_fixmatch.png}
    \caption{Error rates on CIFAR-10, CIFAR-100, SVHN and STL-10 on 5 different folds (taken from \citet{sohn2020fixmatch}).}
    \label{fig:perf_fixmatch}
\end{figure}
\input{tex/2-2_SOTA_PAM}
\input{tex/2-3_challenge_opportunities}
\chapter{Material}\label{chap:material}
\minitoc

In this chapter will be introduced the material used for the experiments conducted throughout this thesis. It takes form as underwater acoustic data, containing several types of signals, and recorded at different places and times. This chapter will thus revolve around two axis: the studied signals, and the recording setups. 


\section{Target species and signals}

The diverse set of target signals described here are those for which detection and classification systems were built. Their characteristics are summarised in Table~\ref{tab:recap_signals}, and each subsection then underpins the current knowledge about them, especially regarding their context of emission.

Note that for the signal types of Tab.~\ref{tab:recap_signals}, and throughout this thesis, stationary refers to ``signals locally stable in frequency'' (calls, whistles) as opposed to transitory signals (clicks).

\begin{table}[ht]
\centering\resizebox{\linewidth}{!}{
\begin{tabular}{|l|c|c|c|c|c|}
        \hline
        \textbf{Species} & \textbf{Sperm whale} & \textbf{Fin whale} & \textbf{Orca} & \textbf{Dolphin} & \textbf{Humpback whale} \\ \hline
        \textbf{Sub-order} & Odontoceti & Mysticeti & Odontoceti & Odontoceti & Mysticeti \\ \hline
        \textbf{Signal} & clicks & 20\,Hz pulses & pulsed calls & whistles & calls \\ \hline
        \textbf{Signal type} & Transitory & Transitory & Stationary & Stationary & Stationary \\ \hline
        \textbf{Frequency (Hz)} & 12,500 & 20 & [500; 5,000] & [5,000. 20,000] & [300; 3,000] \\ \hline
        \textbf{Bandwidth} & 20\,kHz & 2\,Hz & 100\,Hz & 20\,Hz & 50\,Hz \\ \hline
        \textbf{Duration (sec)} & 0.001 & 1 & [0.5; 2] & [1; 2] & [0.5; 1] \\ \hline
%        \textbf{Interval between signals (sec)} & [0.1, 1] & [10, 25] &  &  &  \\ \hline
    \end{tabular}}
    \caption{Summary of the target signals for the detection systems built throughout this thesis. For transitory waves, the frequency denotes the approximate centroid frequency, for stationary signals it denotes its range}
    \label{tab:recap_signals}
\end{table}


\subsection{Fin whale (\textit{Balaenoptera Physalus}) 20Hz pulses}

As the second largest animal on earth, the fin whale produces very low-pitched vocalisations, barely noticeable to the human ear. So far, bioacousticians have documented 3 main types of signals emitted by fin whales: 100-30\,Hz down-sweeps, 30\,Hz rumbles, and 20\,Hz pulses. They supposedly serve group cohesion \cite{payne1971orientation, watkins1981activities}, food signaling \cite{Romagosa}, and mate attraction \cite{Watkins_Tyack_Moore_Bird_1987, croll2002only}.


\begin{figure}[!htb]
   \begin{minipage}{0.5\textwidth}
     \centering
     \includegraphics[width=\linewidth]{fig/20Hzpulse.pdf}
   \end{minipage}\hfill
   \begin{minipage}{0.5\textwidth}
     \centering
     \includegraphics[width=\linewidth]{fig/20Hzpulse_wave.pdf}
   \end{minipage}\hfill
   \caption{Spectrogram (left) and waveform (right) of a fin whale pulse recorded by Bombyx (the two figures share the same abscissa). \ac{STFT} parameters are: $fs=100Hz$, $NFFT=128$, $padding=50\%$, $hopsize=3$.}\label{fig:20Hzpulse}
\end{figure}

In this thesis, I will focus on the most common signal: the 20\,Hz pulse. It is often further classified into two sub categories, named A and B, or classic pulse and back-beat \cite{sciacca2015annual}. They highly resembles a Gabor wavelet: a sine wave enveloped by a Gaussian (see Fig.\ref{fig:20Hzpulse}), and can be emitted either as single pulses, or in patterned sequences, termed as songs \cite{simon2010singing}. The pulses and the sequences they take part in are highly stereotyped: pulses show very low variability both in frequency and duration, and when in sequences, the \ac{INI} remains highly stable.

Fin whale song characteristics, especially the \ac{INI}, are population specific \cite{delarue2009geographic, castellote2012fin}. They also are subject to seasonal cyclic variations \cite{Oleson, morano2012seasonal}, as well as long-term trends \cite{weirathmueller2017spatial, helble2020fin} (e.g. linear increase of the \ac{INI} through years).


\subsection{Sperm whale (\textit{Physeter Macrocephalus}) clicks}

Sperm whales produce echolocation clicks to navigate and locate preys during hunts. Their large head contains a series of oil sacks surrounded by sound-reflecting air sacs that allows for the amplification of the impulses \cite{norris1972theory}, making it the most powerful sonar in the animal kingdom \cite{mohl2003monopulsed} (the loudest recorded click was at 230\,dB re: 1$\mu$Pa rms).

%The reverberation mechanism yields several pulses within a click, and the interval between them is often used as a proxy to estimate the animal's size \cite{gordon1991evaluation}. This interval, typically of around half a millisecond, is usually termed \ac{IPI}, but not to be confused with fin whale's \ac{IPI}, which describes 

\begin{figure}
    \centering
    \includegraphics[width=\linewidth]{fig/cacha_clicks.pdf}
    \caption{Sequence of sperm whale echolocation clicks recorded by Bombyx in july 2018. \ac{STFT} parameters are: $fs=50kHz$, $NFFT=1,024$, $hopsize=896$, $padding=0\%$.}
    \label{fig:cacha_clicks}
\end{figure}

Echolocation clicks usually come in sequences (see Fig.\ref{fig:cacha_clicks}), with the \ac{ICI} ranging between 0.01 and 1\,sec, usually decreasing when approaching a prey \cite{fais2016sperm}. The clicks lie around relatively low frequencies compared to other smaller odontocetes (between 3\,kHz and 30\,kHz). % pas sur This could enable the echolocation of objects at further distances (lower frequencies are subject to a lower propagation loss).


\subsection{Orca (\textit{Orcinus Orca}) calls}\label{chap:mat_calltype}

Orcas produce three types of signals: clicks, pulsed calls, and whistles \cite{ford1989acoustic}. As for most dolphin species, clicks presumably serve echolocation, while the two other more stationary signals would rather be used for communication. Pulsed calls are highly harmonic, typically lying between 500\,Hz and 5\,kHz, and lasting up to 1.5 seconds (Fig.~\ref{fig:orca_call}). On the other hand, whistles show little or no harmonic structure, lay between 6\,kHz and 12\,kHz, and can last up to 12 seconds. In this thesis, I will focus on the pulsed calls, referring to them as calls or vocalisations.

\begin{figure}
    \centering
    \includegraphics[width=\linewidth]{fig/orca_call.pdf}
    \caption{Sequence of orca tonal calls recorded at OrcaLab in September 2016. \ac{STFT} parameters are $fs=22050Hz$, $NFFT=1024$, $hopsize=50$, $padding=0\%$. The N.. labels denote each call type, with a `?' showing an ambiguous one.}
    \label{fig:orca_call}
\end{figure}

As shown in Fig.~\ref{fig:orca_call}, some orca calls have stereotyped frequency contours that have been classified into discrete types. These were proven to be community specific (dialectic) \cite{ford1987catalogue}, and subject to cultural evolution \cite{deecke2000dialect, filatova2015cultural}. The identification of call types strongly contributed to the study of the orca's social structures, and its categorisation is widely accepted by the scientific community. Difficulties remain however, for some calls to be attributed to one class or another, especially for non experts. Indeed, despite calls being stereotyped, they still are prone to variability which might lead to overlap between classes' characteristics \cite{ford1989acoustic}.


\subsection{Humpback whale (\textit{Megaptera Novaeangliae}) calls}

The humpback whale song is among the most widely studied cetacean acoustic signals. These sequences are mostly emitted by males during the reproductive season, presumably playing a role in courtship \cite{herman2017multiple} (male-female and/or male-male interaction). They follow strict hierarchical structures: series of units form phrases that are arranged into themes, themselves combined in songs that can last several hours \cite{payne1971songs}.

\begin{figure}
    \centering
    \includegraphics[width=\linewidth]{fig/HB_calls.pdf}
    \caption{Extract of a humpback whale song from the Carimam dataset. \ac{STFT} parameters are $fs=22050Hz$, $NFFT=4096$, $hopsize=48$, $padding=50\%$.}
    \label{fig:HB_song}
\end{figure}

Each component of the hierarchical structure of the humpback whale songs are stereotyped, as seen in Fig.~\ref{fig:HB_song} with a sequence of stereotyped units. Moreover, song structures are shared by individuals at a given place and time, with cultural implications for their spatio-temporal evolution \cite{whitehead2015cultural}.


\subsection{Dolphin (\textit{Delphinidae}) whistles}

Exceptionally for this type of signal, we do not target a single species, but rather a family of species, the \textit{Delphinidae} which includes sub-families such as \textit{Globicephalinae}, \textit{Delphininae}, and \textit{Orcininae}. They all produce whistles, which are typically high pitched, tonal, and narrow-band. Their frequency contour can be stereotyped \cite{vester2017vocal}, individual specific \cite{caldwell1965individualized}, and serve group cohesion \cite{janik1998context}.

\begin{figure}
    \centering
    \includegraphics[width=\linewidth]{fig/whistles.pdf}
    \caption{Sequence of dolphin whistles from the Carimam dataset. \ac{STFT} parameters are $fs=256kHz$, $NFFT=8192$, $hopsize=512$, $padding=0\%$, Mel transformed from 5 to 40\,kHz, and \ac{PCEN} normalised. The stationary signal around 12\,kHz is the remaining self noise of the sound card used \cite{barchasz2020novel} (despite heavy mitigation via the \ac{PCEN}).}
    \label{fig:whistle}
\end{figure}


\section{Data at hand}

In order to experiment on detection and classification mechanisms for the target species and signals aforementioned, datasets are needed. Through this thesis, work has been conducted on both recordings from local projects and publicly available ones. They involve a variety of recorders, locations and time spans which are described in this section, starting with the local projects of the DYNI team (Toulon University), and followed by the public Blue and Fin whale acoustic trends dataset.


\subsection{Data from DYNI}\label{chap:data_Toulon}

\begin{figure}
    \centering
    \includegraphics[width=.7\linewidth]{fig/map_ligure.png}
    \caption{Map of the 3 Mediterranean antennas used throughout this thesis.}
    \label{fig:map_med}
\end{figure}

Table~\ref{tab:recap_data} summarises some of the data yielded by the partnerships and projects that H. Glotin co-set up at Toulon University. They are stored locally in a \ac{NAS} system, funded by the DYNI team projects and maintained by the LIS laboratory. Each are briefly introduced in the following sections.

\begin{sidewaystable}
\centering
\begin{tabular}{|l|c|c|c|c|c|c|}
\hline
 \textbf{Data source} & \textbf{Boussole} \cite{parsuivi} & \textbf{Bombyx} \cite{vamos2017} & \textbf{OrcaLab} \cite{orcalab} & \textbf{Carimam} \cite{carimam} & \textbf{KM3Net} \cite{aiello2021km3net} \\ \hline
\textbf{Location} & Côte d'Azur & Côte d'Azur & British Columbia & Caribbean & Côte d'Azur \\ \hline
\textbf{Depth (m)} & 10 - 25 & 25 & 0 - 20 & 5 - 20 & 2,440 \\ \hline 
\textbf{Recording year} & 2008-2009 & 2015-2018 & 2015-2021 & 2021-today & 2020-2021 \\ \hline
\textbf{Sampling rate (Hz)} & 32,000 & 50,000 & 22,050 - 44,100 & 384,000 & 195,312 \\ \hline
%OL 22,050 until March 18, then 44,100
\textbf{ON/OFF protocol (min)} & 5/10 & 1/5 - 5/15 & Continuous & 1/5 & Continuous \\ \hline
%Bombyx 1/5 until Oct. 17, then 5/15
\textbf{Channels} & 1 & 2 & 5 & 15 & 3 \\ \hline
\textbf{Recorded time (hours)} & 1,752 & 3,533 &  & 5,677 & 514 \\ \hline
\end{tabular}
\caption{\label{tab:recap_data}Summary of the recording characteristics for each data source available at Toulon University.}
\end{sidewaystable}


\subsubsection{Boussole}

\begin{figure}[!htb]
   \begin{minipage}{0.7\textwidth}
     \centering
     \includegraphics[width=\linewidth]{fig/bombyx.png}
   \end{minipage}\hfill
   \begin{minipage}{0.25\textwidth}
     \centering
     \includegraphics[width=\linewidth]{fig/boussole.png}
   \end{minipage}\hfill
   \caption{(left) Installation of the Bombyx stereophonic antenna \cite{vamos2017}. (right) Structure of the Boussole antenna \cite{parsuivi}}\label{fig:antennas}
\end{figure}

This project consisted in a partnership between GIS3M, Pelagos marine mammal sanctuary, and Port-Cros National Park. In order to study marine mammals acoustic activity, a monophonic recording system was placed on the Boussole buoy. Originally dedicated to marine optics, this buoy designed to be transparent to swell was moored on the 2,440 meters deep sea floor, off the coast of Nice (France). During 4 phases between October 2008 and September 2009, the system recorded at 32\,kHz, enabling the detection of vocalisations from sperm whales, fin whales, and delphinids of the area (\textit{Stenella coeruleoalba, Globicephala melas, Grampus griseus, Tursiops truncatus and Delphinus delphis}).

A study prior to this thesis intended to monitor the acoustic presence of sperm whales and fin whales in the yielded recordings. Sperm whale clicks were successfully detected automatically but the processing of fin whale 20\,Hz pulses was hindered by the self noise of the system \cite{parsuivi} (Fig.~\ref{fig:boussole_noise}).

\begin{figure}
    \centering
    \includegraphics[width=\linewidth]{fig/boussole_noise.pdf}
    \caption{Spectrogram of a noisy recording from the Boussole antenna ($f_s=32kHz$, $NFFT=32768$, $hop=5568$). White dots denote the temporal position of some confirmed fin whale 20\,Hz pulses.}
    \label{fig:boussole_noise}
\end{figure}


\subsubsection{Bombyx}

The Bombyx antenna was set up by a partnership between Toulon University, Port-Cros National Park, TVT Innovation, and the Pelagos marine mammal sanctuary. Being placed right on the rift of a 2000 meters deep canyon, it intends to allow the monitoring of sperm whales of the area \cite{vamos2017}. It did so during several phases spread across 4 years (2015 to 2018). The area is of interest because of the nearby canyons prone to sperm whale hunts \cite{fiori2014geostatistical}, but also because of the ferries that travel across on a daily basis. In addition to the noise that the latter generate, Bombyx recordings are also subject to self noise (Fig.~\ref{fig:bombyx_noise}).

\begin{figure}
    \centering
    \includegraphics[width=\linewidth]{fig/bombyx_noise.pdf}
    \caption{Example of signal from the Bombyx antenna (high pass filtered, order 3 butterworth at 3\,kHz). Grey dots denote sperm whale clicks, and red ones self noise from the recording device. (top) Waveform. (bottom) Spectrogram ($f_s=5kHz$, $NFFT=512$, $hop=256$). }
    \label{fig:bombyx_noise}
\end{figure}


\subsubsection{OrcaLab}

Paul Spong founded OrcaLab in the 1970s \cite{orcalab}, an in-situ observatory in the Johnstone Strait (British Columbia, Fig.~\ref{fig:ol_map}). It serves the visual and acoustic monitoring of orcas, especially the population that feeds on the local salmon every summer, the \ac{NRKW}. From 2015 to 2020, the 5 hydrophones' signals have been recorded continuously  (at 22,050\,Hz until march 2018, then at 44,100\,Hz).

The fact that the orcas regularly come to this relatively confined space represents an unique opportunity to observe and listen to them 24/7 from the shore. Most importantly, it guarantees no behavioral disturbance and continuous power and data storage supply, the main constraints of most \ac{PAM} approaches.

\begin{figure}
    \centering
    \includegraphics[width=.8\linewidth]{fig/OL_map.png}
    \caption{Map of the OrcaLab observatory, with its 5 hydrophones and associated acoustic range.}
    \label{fig:ol_map}
\end{figure}


\subsubsection{KM3Net}

The ORCA detector of the KM3Net observatory is an array of detection units allowing the measurement of neutrino particles \cite{aiello2021km3net}. It was installed on the seabed 2,440 meters deep, connected to the shore of Toulon (France) via fiber cable. Hydrophones are used as part of a positioning system, but as a by-product, also serve the \ac{PAM} of local cetaceans.


\subsubsection{Carimam}

\begin{figure}
    \centering
    \includegraphics[width=.8\linewidth]{fig/effort_T1+T2.jpg}
    \caption{Map of recording stations with their recording effort for the Carimam project, in the Caribbean archipelago.}
    \label{fig:carimam_effort}
\end{figure}

The Carimam project, led by a consortium composed of AGOA, the OFB and Toulon University, is a network of 16 monophonic acoustic stations spread through the Caribbean archipelagos. It aims at monitoring the rich marine mammal activity of the area. To manage such a wide number of stations, low-cost and easy to install recording devices \cite{barchasz2020novel} were sent to local environmental managers, who set them up on existing mooring lines close to the shore.


\subsubsection{Spatialisation}

In Table \ref{tab:recap_data}, number of channels of each recording system are given. When synchronised and with overlapping acoustic coverage, multi-channel data can serve the spatialisation of acoustic sources. This is done via the computation of \acp{TDOA} for signals to be triangulated. For Bombyx, since two hydrophones record 1.8 meters apart (on the same horizontal plane, see Fig.~\ref{fig:antennas}), the two possible azymuths of acoustic sources can be computed. For Carimam, the stations' acoustic coverage do not overlap: the spatial precision is the acoustic range of the antennas. For KM3Net, the 3 hydrophones are approximately 30 meters apart. With prior knowledge on the depth of a source, its coordinates could be estimated (see section \ref{chap:km3net_triang}). Finally, for the OrcaLab network, hydrophones are several kilometers apart, but sent to a centralised \ac{DAC} via radio waves, which makes them temporally synchronised. Therefore, spatialisation could be performed in the zones of acoustic overlap.


\subsection{Blue and Fin whale acoustic trends dataset} \label{chap:acTrend}

In early 2021, a large acoustic dataset of antarctic mysticetes was made publicly available \cite{miller2021open}. It was built by a working group from the \ac{SOOS} titled Acoustic Trends of Antarctic blue and fin whales (Acoustic Trends Working Group; ATWG). The following is an extract from their terms of reference \cite{atwg_tor}:
\\
\textbf{\underline{SOOS Capability Working Group Key Objective(s):}}
\textit{
Continue to develop and mature a long term acoustic research program to understand trends in Southern Ocean blue and fin whale distribution, seasonal presence, and population growth through the use of passive acoustic monitoring techniques. Implementation of these objectives will occur via:
\begin{enumerate} \setlength{\itemsep}{1pt}
    \item analysis and interpretation of existing ad-hoc acoustic datasets in from the Southern Ocean,
    \item the development and implementation of an ongoing network of long-term circumpolar underwater listening stations, and
    \item development of novel and efficient methods for standardised analysis of acoustic data collected in the Antarctic and sub-Antarctic
\end{enumerate}}

It is regarding this third axis of work that the Acoustic Trends dataset was built and published, especially to share performance metrics for detection systems. It gathers annotations from a group of experts, on data yielded by several recorders at different locations from 2005 to 2017 (see Fig.~\ref{fig:soos_map} and Tab.~\ref{tab:soos_data}). Target signals are of 7 classes, 4 vocalisation types from blue whales (\textit{Balaenoptera Musculus}) and 3 vocalisation types from fin whales (\textit{Balaenoptera Physalus}).

Since this data has not been subject to custom annotations, I won't expand on the target signals which are described in the dataset publication \cite{miller2021open}.

\begin{figure}
    \centering
    \includegraphics[width=.7\linewidth]{fig/soos_ATW_map.png}
    \caption{Map of the recording stations used in the Acoustic Trends dataset. The map was published by \citet{miller2021open}.}
    \label{fig:soos_map}
\end{figure}


\begin{table}[ht]
\centering
    \begin{tabular}{c|c|c|c}
    \textbf{Location} & \textbf{Year} & \textbf{Instrument} & \textbf{Recordings (hours)}\\ \hline
    Balleny Islands & 2015 & PMEL-AUH & 204\\
    Elephant Island & 2013 & AURAL & 707 \\
    Elephant Island & 2014 & AURAL & 216 \\
    Greenwich 64S & 2015 & Sono.Vault & 32 \\
    MaudRise & 2014 & AURAL & 80 \\
    Ross Sea & 2014 & PMEL-AUH & 184 \\
    Casey & 2014 & AAD-MAR & 194 \\
    Casey & 2017 & AAD-MAR & 187 \\
    Kerguelen 1 & 2005 & ARP & 200 \\
    Kerguelen 2 & 2014 & AAD-MAR & 200  \\
    Kerguelen 2 & 2015 & AAD-MAR & 200 \\
    \end{tabular}
    \caption{Summary of recorders' characteristics and amounts of data available in the Acoustic Trends dataset.}\label{tab:soos_data}
\end{table}

\chapter{Optimising annotation processes}

\minitoc

\section{Context and objective}

Given the large amount of available recordings presented in the previous section, the objective of this thesis is to build robust detection and classification mechanisms for the vocalisations of species of interest. For this purpose and with the chosen approach of \acp{ANN}, annotated databases are needed. In the following chapter, procedures and \acp{UI} suited for bioacoustic use cases are proposed, with an objective of optimising annotation quantity while minimising human effort. For all tasks, the annotation procedure can be summarised in 5 steps that are introduced Fig.~\ref{fig:flowchart}.

\begin{figure}
    \centering
    \includegraphics[width=\linewidth]{fig/annot_flowchart.pdf}
    \caption{Flow chart of procedures employed in the annotation processes.}
    \label{fig:flowchart}
\end{figure}

This chapter starts by introducing a versatile and efficient approach to annotation (thumbnail picking), which will be needed in the subsequent experiments. Then, algorithms and \acp{UI} are proposed for several use cases, each being adapted to specific constraints:
\begin{itemize}\setlength{\itemsep}{1pt}
    \item To detect stationary signals (orca calls) and given some samples to tune a handcrafted algorithm, a spectrogram binarisation approach is described (section~\ref{chap:spec_bin}).
    \item Looking for transitory signals (sperm whale clicks) in stereophonic recordings, I propose an interface to visualise and annotate \acp{TDOA} tracks (section~\ref{chap:bombyx_annot}).
    \item For a case when no target signals are available a priori, a generic extraction of spectral distributions is used to cluster similar acoustic events (section~\ref{chap:HB_annot}).
    \item In contrast, when a large quantity of signals of interest are available, an \ac{AE} demonstrated the ability to learn relevant features to measure similarity and enhance annotation efficiency (section~\ref{chap:autoencoder}).
\end{itemize}

Finally, after these methods were employed to gather an initial set of annotations, active learning was conducted until a satisfying amount of labels are available (section~\ref{chap:active}). They are presented with their chosen train / test split in the last section of this chapter.


\section{Thumbnail picking} \label{chap:png_annot}

Often during annotation procedures, we want to manually sort out true and false positives from a set of detections. It occurred numerous times during this thesis, after the aforementioned preliminary detection algorithms or during the active learning process (section~\ref{chap:active}). Picking spectrogram images from their thumbnails in file explorers appeared to be the most efficient way to do it (see Fig.~\ref{fig:png_annot}).

\begin{figure}
    \centering
    \includegraphics[width=\linewidth]{fig/thunar.png}
    \caption{Example of thumbnails ready to be annotated (picked), using the Thunar file explorer \cite{thunar}. Here, files are clustered spectrograms of orca calls (see section~\ref{chap:autoencoder}).}
    \label{fig:png_annot}
\end{figure}

The typical scenario in which this procedure was used is to pick false positives from a set of detections. In a few minutes, an annotator can browse hundreds of samples (exhaustively or not), and select dozens of files to move them to a new folder. Using table identifiers as filenames then allows to retrieve the annotator's decision and save it for later use.

Annotating by organising of thumbnails in folders is not only efficient in time, but also very generic (it requires no specific software installation). This comes practical especially when needing annotation efforts from different people with different operating systems for example.


\section{Gathering regions of interest}

When choosing machine learning to build detection systems, we must first gather annotations. For this purpose, we can start by running an algorithm that filters the data using our prior knowledge of the target signal(s). These handcrafted algorithms present limitations (as argued in section~\ref{chap:limitations}), but avoid having to go through the whole set of available recordings to find our first training examples.

In detection algorithms, the user usually sets a threshold to binarise continuous prediction values. For instance with cetacean vocalisation detection tasks, the threshold is typically on the energy level at a specific frequency, or on the cross-correlation coefficient (template matching approaches). The lower we set this threshold, the lower the specificity (higher risk of false detections) but also the higher the sensitivity (lower risk of missed detections). Conversely, by increasing this threshold, we increase the specificity but decrease the sensitivity.

This trade-off is to be kept in mind when tuning handcrafted algorithms to build a first database: we want just enough sensitivity to yield some true positives (perhaps the ones with the highest \ac{SNR}), while keeping the number of detections low enough so that we can go through them in a reasonable amount of time. 

The following paragraphs introduce two case studies of such approaches, one with stationary signals (orca calls) and one with transitory ones (sperm whale clicks).


\subsection{Spectrogram energy thresholding (orca calls)}\label{chap:spec_bin}
\textit{This work was conducted in collaboration with Jan Schlüter and Marion Poupard, on the OrcaLab data (see section~\ref{chap:data_Toulon}).} \\
The chosen approach to the preliminary detection of orca calls was inspired by \citet{lasseck2014large} on spectrogram segmentation for bird call detection. We first binarise spectrograms (see Fig.~\ref{fig:fig_binarisation}) with adaptive thresholds using rows and columns moments. The original formulation proposed by \citet{lasseck2014large} for the threshold $T_{f,t}$ given a log compressed spectrogram $\mathbf{E}$ is given by Eq.~\ref{lasseck}. The goal being to detect pixels with energy values above the distribution of their row and column, we propose to rather use Eq.~\ref{binarise}.

\begin{equation}\label{lasseck}
    \mathrm{T}_{f,t} = \max(3 \times \underset{j}{\median}(\mathrm{E}_{f,j}), \; 3 \times \underset{i}{\median}(\mathrm{E}_{f,i})).
\end{equation}
\begin{equation}\label{binarise}
    \mathrm{T}_{f,t} = \max( \underset{j}{\median}(\mathrm{E}_{f,j}) + 2 \times \underset{j}{\std}(\mathbf{E}_{f,j}), \; \underset{i}{\median}(\mathbf{E}_{i,t}) + \underset{i}{\std}(\mathbf{E}_{i,t})).
\end{equation}

\begin{figure}
    \centering
    \includegraphics[width=\linewidth]{fig/lasseck.pdf}
    \caption{Comparison of the spectrogram binarisation procedure following Eq.~\ref{lasseck} (middle) and Eq.~\ref{binarise} (right).}
    \label{fig:fig_binarisation}
\end{figure}

Connected positive pixels are later grouped by regions, from which we will extract features such as minimum and maximum frequencies, length, and mean and maximum decibels. We finally use our prior knowledge of orca calls to filter out impossible regions (out of range features), and plot them for annotation via thumbnail picking (see section~\ref{chap:png_annot}).


\subsection{\acs{TDOA} tracking (sperm whale clicks)}\label{chap:bombyx_annot}
\textit{This work was conducted in collaboration with Maxence Ferrari and Marion Poupard, on the Bombyx data (see section~\ref{chap:data_Toulon}).}\\
For sperm whale clicks, time domain signal processing is more appropriate than the spectral based energy detection presented above. In a first pre-processing step, sperm-whale clicks are emphasised by correlating the signal with a sinus of their centroid frequency (12.5\,kHz). Then, the permissive detection mechanism is based on the \ac{TK} energy operator (inspired by \citet{kandia2006detection}). The \ac{TDOA} of the detections were then computed between the two hydrophones of the antenna, as we will see that spatial information is quite useful for the identification of sperm whales.

\iffalse
\begin{python}[caption={Detect clicks using find\_peaks  \cite{scipy}, and estimate \ac{TDOA}}]
import numpy as np
import soundfile as sf
from scipy import signal

fs = 50_000
win = int(0.1 * fs)
bandpass = signal.butter(3, [9_000 * 2/fs, 13_000 * 2/fs], 
                         btype='bandpass', output='sos')
sig, fs = sf.read('filename.wav')
# apply forward-backward digital filter
sig = signal.sosfiltfilt(bandpass, sig)
# find clicks using find_peaks 
peaks = signal.find_peaks(sig[:,0],
                          prominence=0.02,
                          distance=int(0.045 * fs))[0]
for p in peaks:
    tdoa = np.argmax(np.convolve(sig[p-win//2:p+win//2, 0],
                                 sig[p-win//2:p+win//2, 1]))
\end{python}
\fi

In our data the three main signals that trigger such a detector are those produced by sperm whales, boats, and other odontoceti such as long-finned pilot whales (\textit{Globicephala melas}). To discriminate between these three for annotation, while browsing the large amount of recordings, the custom \ac{UI} shown in Figure~\ref{fig:cacha_annot_UI} was built.

\begin{figure}
    \centering
    \includegraphics[width=\linewidth]{fig/interace_bombyx.png}
    \caption{Custom \ac{UI} built in matplotlib \cite{matplotlib} for the annotation of sperm whale clicks. (top) \acp{TDOA} versus time of detected clicks, with vertical bars denoting gaps between recorded files. (bottom) Spectrogram of the signal surrounding the selected click, shown with a red dot on the top panel.}
    \label{fig:cacha_annot_UI}
\end{figure}

This \ac{UI} shows a scatter plot of \ac{TDOA} of preliminary detections versus time. This allows for the identification of tracks revealing moving acoustic sources, with the slope reflecting angular speed relative to the antenna. With such a plot, we display 10 hours of signal at once, enabling a quick browse through large amounts of data. When clicking on a point of the scatter plot, it is signaled with a red dot, and the surrounding signal's spectrogram is displayed on the bottom pane while the sound is played. This allows for the identification of the source responsible for the selected track. The user can eventually click on buttons to save an annotation along with its timestamp (noise, pilot whale or sperm whale).


\section{Feature extraction and filtering}

Clustering allows for a strong optimisation of the annotation process. Indeed, once signals are grouped by similarity, browsing and sorting becomes much more efficient, especially by avoiding to go through large amounts of void.

The key to clustering quality is the extraction of relevant features for similarity measurement. Hereby are presented three feature extraction approaches on different kinds of signals : humpback whale vocalisations, toothed whale clicks, and orca calls.

Once features were extracted, they were usually projected using \ac{UMAP} \cite{mcinnes2018umap} before a \ac{DBSCAN} clustering \cite{ester1996density}. \citet{allaoui2020considerably} have shown that dimensionality reduction using \ac{UMAP} would improve the performance of clustering such as density based ones. The distribution of projection in turn motivated a density based approach to clustering such as \ac{DBSCAN} (Fig.~\ref{fig:cluster_interface_HB}).


\subsection{Spectro-temporal features (humpback whale calls)} \label{chap:HB_annot}
\textit{This work was conducted on the Carimam dataset (see section~\ref{chap:data_Toulon}).}\\
The objective of the following procedure is to explore a large dataset with no samples of target signals given a priori. For this purpose, a relatively generic feature extraction was conducted before plotting and clustering their projection. Like so, we intend to isolate groups of similar events, and allow for a more efficient exploration of the data. Especially, the events we hope to find are click trains and cetacean vocalisations, but we also expect to retrieve events from other noise sources.

\begin{figure}
    \centering
    \includegraphics[width=\linewidth]{fig/HB_specextract.pdf}
    \caption{Main steps of the spectral feature extraction procedure. The spectrogram is first max-pooled time-wise by a given factor. Then, each frequency bin is sorted (time-wise) to make abstraction of the temporal position of events.}
    \label{fig:feat_spec}
\end{figure}

The extracted features process spectrogram chunks in two main steps in order to emphasis potential signals of interests (Fig.~\ref{fig:feat_spec}). First, to get a representation that preserves short events (such as clicks) but with a reduced size, we max-pool spectrograms time wise. Second, to make abstraction of the temporal information (whether an event is at the beginning of a chunk or at its end) we sort each frequency bin (time wise) in descending order.

Doing so, the resulting matrix is no longer a spectrogram, but rather a representation of the energy distribution for each frequency bin. This allows to select specific columns as discriminating features, the first denoting the highest energy in the chunk for each frequency bin, and the last their lowest. For instance, looking for short events, we can select the first and second columns. Chunks with a large gap between the two should contain a short but strong acoustic event. It could thus be differentiated from chunks with stationary strong energy and chunks with a low overall energy, and this for specific frequency bins. For simplicity, this set of columns to be kept will be referred to as `quantiles'.
The full procedure for this analysis is described in Listing~\ref{hb_cluster}.

\begin{python}[caption={Feature extraction and clustering for humpback whale vocalisations. Steps are operated over a batch of signals on \ac{GPU} for computation speed.}, label={hb_cluster}]
from torchaudio.functional import resample
import torch
from sklearn.cluster import DBSCAN
from umap import UMAP
gpu = torch.device('cuda')

# load a batch of signals using pyTorch DataLoader
sigs = ... 
sigs = sigs.to(gpu)
sigs = resample(sigs, source_fs, fmax * 2)
# compute the magnitude spectrogram using the STFT
specs = torch.stft(sigs, n_fft=1024, hop_length=512)
specs = 20 * torch.log10(specs.norm(p=2, dim=-1))
# substract a background noise estimate
specs = specs - specs.median(dim=1, keepdim=True)[0]
# apply the mel-transform
spec = torch.matmul(melbank, specs)
# undersample the spectrogram over the time dimension
specs = torch.nn.MaxPool1d((uds,))(specs)
# rearange the tensor into a list of time chunks
specs = specs.permute(1, 0, 2)
specs = specs.reshape(specs.shape[0], -1, chunksize)
specs = specs.permute(1, 0, 2)
# sort frequency bins and select quantiles
features = torch.sort(specs, dim=2, descending=True)[0]
features = features[:,:,quantiles].numpy()
# project and cluster each time chunk
features = features.reshape((specs.shape[0], -1))
embeddings = UMAP().fit_transform(features)
clusters = DBSCAN().fit_predict(embeddings)
\end{python}

The variables \pythoninline{fmax}, \pythoninline{uds}, and \pythoninline{chunksize} need to be tuned to the type of signals we desire to isolate, especially in terms of spectrum range and spectrogram temporal resolution. \pythoninline{fmax} determines $f_s$ at which the signal is resampled, \pythoninline{uds} determines the downsampling factor used for max-pooling, and \pythoninline{chunksize} determines the sampling rate of the feature extraction process. As for the humpback whales, they were set to 8,000\,Hz, 14 and 20 respectively (chunks of 10\,sec with 20 time bins). Then, the first seconds and fifth quantiles were chosen.

These choices were made via intuition and empiric testing. Once annotations were gathered, experiments were carried out to measure which configuration would have been the most efficient.

Trials with varying values for the size of chunks and the choice of quantiles were conducted, using the \ac{NMI} between clusters and annotations as a metric of configuration quality. The choice of configuration appeared to have a relatively small impact on the resulting \ac{NMI}, with values ranging between 0.15 and 0.18 (the random baseline being under 0.01). The highest scoring configuration was to cut chunks of size 10 with only the first quantile.

Once features have been extracted for a large amount of samples, we reduce their dimensionality (using \ac{UMAP}), and cluster them (using \ac{DBSCAN}). A custom made interface then enables a seamless browsing of this clustered projection (see Fig.~\ref{fig:cluster_interface_HB}). 

\begin{figure}
    \centering
    \includegraphics[width=\linewidth]{fig/cluster_vocs_interface.png}
    \caption{Interface for browsing clusters. The left panel displays clustered \ac{UMAP} projections of audio chunks spectral features, with red dots signaling points that have been clicked on. The right panel displays the spectrogram of the last selected audio as well as its metadata.}
    \label{fig:cluster_interface_HB}
\end{figure}

Users can select an audio chunk by clicking on its projection on the scatter plot. The interface will then play the corresponding sound extract and display its spectrogram on a secondary window. This allows for the identification of discriminant clusters to be retained (containing only vocalisations, or only noise for instance). Eventually, we can plot samples belonging to selected clusters as .png files and use thumbnail picking (see section~\ref{chap:png_annot}) to sort out misclassified samples.


\subsection{Impulses' features (toothed whale clicks)}
\textit{This work was strongly inspired by \citet{frasier2021machine}, conducted in collaboration with Maxence Ferrari and Marion Poupard, on the Carimam data (see section~\ref{chap:data_Toulon}).}\\
In a similar approach, we might want to cluster clicks for their spectral features to infer \ac{ICI} characteristics, which helps discriminate toothed whales click trains from reef noise. To do so, using the \ac{STFT} as seen in the previous section is not appropriate. We would rather use a generic impulse detection mechanisms on the waveform, and extract their features. I thus propose the following steps :

\begin{itemize}  \setlength{\itemsep}{1pt}
\item Generic impulse detection:
    \begin{itemize}
    \item high pass the signal $x(t)$ at 5\,kHz,
    \item compute the Hilbert transform $H(t)$ of $x(t)$,
    \item compute a running average $a(t)$ to smooth $H(t)$,
    \item convert $a(t)$ into decibels with $20 \times \log_{10}(a(t))$,
    \item compute the $\median$ and $\std$ of $a(t)$,
    \item find peaks of $20 \times \log_{10}(x(t))$ that are 3dB above the noise level expressed as $\median + 3 \times \std$,
    \item retain peaks with widths between 0.008 and 1.2ms, and retain its highest sample.
    \end{itemize}
\item Feature extraction:
    \begin{itemize}
        \item compute the \ac{FFT} of a 1ms window surrounding the detected impulse,
        \item compute the 3\,dB centroid frequency,
        \item cluster impulses by their centroid frequency,
        \item compute \acp{ICI} as the time difference between impulses of the cluster,
        \item fit a gaussian \ac{KDE} on the \ac{ICI} distribution of the cluster,
        \item estimate the peak of the \ac{KDE},
        \item for each cluster, save the peak of the \ac{KDE} (most frequent \ac{ICI}), its width (\ac{ICI} variability), and the mean 3dB centroid frequency.
    \end{itemize}
\end{itemize}

The user can eventually filter data on the \ac{KDE} peaks height and width depending on the desired specificity. An interface then displays a scatter plot of \acp{ICI} vs centroid frequencies (Fig.~\ref{fig:interface_clicks}). Again, a click on a point triggers a spectrogram display of the corresponding signal, which can be further analysed and eventually saved for annotation.

\begin{figure}
    \centering
    \includegraphics[width=\linewidth]{fig/annot_click.png}
    \caption{Interface for browsing cluster of clicks. The left panel displays the mean \ac{ICI} vs 3dB centroid frequency for each cluster of impulses. Red dots signal the selected cluster of clicks. The right panel displays the spectrogram of audio surrounding the selected cluster as well as its meta data.}
    \label{fig:interface_clicks}
\end{figure}

This method was used to explore the data in view 


\subsection{AE embeddings (orca calls)} \label{chap:autoencoder}
\textit{This work was conducted on the OrcaLab data (see section~\ref{chap:data_Toulon}), and has been subject to a workshop intervention \cite{bestvihar}.}\\
For this section, we are interested in the classification of pre-detected orca calls (dataset of 114k orca calls detected by a \ac{CNN} presented in section~\ref{chap:orca_detec}). Call types, as defined by \citet{ford1987catalogue} for \acp{NRKW}, are determined by their temporal pitch patterns. First experiments were thus conducted using a pitch based feature extraction to cluster calls \cite{poupard2019large}. However, the estimation of the pitch appeared to be quite unreliable in low and medium \ac{SNR} conditions (see section~\ref{chap:sota_pitch}). This led to a switch towards a larger scale extraction of shape (as opposed to local pitch estimates).

Auto-encoders (introduced in section~\ref{chap:AE}) are trained to compress data in a lower dimensional embedding space while being able to reconstruct it. Moreover, since the reconstruction relies on learning structure in the data, the noise in the input (random and unstructured) is omitted in the output. This motivates the use of \acp{AE} for the feature extraction of orca calls, expecting the bottleneck to contain the shape of the call in a low dimensional space.

\begin{figure}
    \centering
    \includegraphics[width=\linewidth]{fig/encoder.pdf}
    \caption{Architecture of the encoder part of the \ac{AE}. (Bottom) shapes of volumes as $(depth \times height \times width )$. (Top) Operations and kernel shapes as $( height x width )$.}
    \label{fig:encoder}
\end{figure}

The training framework of the \ac{AE} was designed as follows (see Fig.~\ref{fig:AE_archi}):
\begin{itemize}
    \item Compute Mel-spectrogram on windows of 2sec around detections ($f_s=22050$, $NFFT=1024$, $hop=330$, $\# Melbands=128$, $f_{min}=300$, $f_{max}=11,025$),
    \item Run the encoder to compress the 128x128 image to 16 dimensions (Fig.~\ref{fig:encoder}). Each convolution is followed by batch normalisation and leaky rectifier linear units. The resolution is lowered via strides of 2 for each convolution, and a max-pooling layer,
    \item Run the decoder as the mirror of the encoder. The first 16x4x2 volume is created via a linear layer, and each resolution increase consists in upsampling by nearest value followed by two convolution layers of kernels 3x3,
    \item Compute the \ac{VGG} embedding of the input and the reconstructed images as the activations after the 6th convolutionnal layer,
    \item Use the \ac{MSE} between the two \ac{VGG} embeddings as the loss \\ ($\mathcal{L} = \Sigma \lVert \VGG(\mathbf{E}) - \VGG(\mathbf{\hat{E}})\rVert^2$).
\end{itemize}

The \ac{VGG} mentioned, used for the perceptual loss \cite{johnson2016perceptual}, is a \ac{VGG}16 pretrained on the ImageNet dataset \cite{deng2009imagenet}.

\begin{figure}
    \centering
    \includegraphics[width=.7\linewidth]{fig/AE_archi.pdf}
    \caption{Architecture of the training framework for the \ac{AE} of orca calls using a perceptual loss \cite{johnson2016perceptual}.}
    \label{fig:AE_archi}
\end{figure}

The size of the bottleneck was empirically chosen as the minimum that still enables satisfactory reconstructions. Fig.~\ref{fig:reconstructions} demonstrates how, in reconstructions, details of some calls are omitted, and background noise becomes patterned. Indeed, due to the limited amount of information that the bottleneck can fit, the decoder is forced to learn common data structures to reconstruct the data. This is actually beneficiary for our end goal of grouping roughly similar shapes together, and it explains why random background noise, transient clicks, and small variations in call shapes are not found in output spectrograms.

\begin{figure}
    \centering
    \includegraphics[width=\linewidth]{fig/AE_outputs.pdf}
    \caption{Comparison of input and \ac{AE} reconstructed spectrograms.} 
    \label{fig:reconstructions}
\end{figure}

The bottleneck embeddings were later used as features for DBSCAN clustering (after \ac{UMAP} dimensionality reduction \cite{mcinnes2018umap}). This enabled a drastic reduction of the annotation effort by grouping similar calls together. Thumbnail picking (see Fig.~\ref{fig:png_annot}) was then conducted to verify clusters and associate them with the orca call types defined by \citet{ford1987catalogue}.

%TODO show some clusters



\section{Active learning}\label{chap:active}

Active learning is the process of iteratively training and annotating to improve a database (qualitatively and/or quantitatively). It is relevant when one has a training database that is not large enough to ensure satisfactory performances. By correcting the model's predictions at each iteration, we emphasis on difficult examples and guide it towards robustness.

This active learning process was conducted via thumbnail picking to gather annotations for fin whale pulses, dolphin whistles, humpback whale vocalisations, and orca calls.


\subsection{Transfer learning (fin whale pulses, dolphin whistles)}

Pre-training a model on a database before fine tuning on a different one is called transfer learning. Similar approaches were used to kick-start the active learning process on two detection tasks, as described in the following paragraphs.


\subsubsection{Fin whale pulses} \label{chap:fin_activelearning}

To gather a database of fin whale 20\,Hz pulses from the recordings of Bombyx and Boussole (see section~\ref{chap:data_Toulon}), several handcrafted algorithms were first tested (looking for strong energy peaks in realistic time and frequency ranges). Without any exemplary signal to tune them, and with the wide variety of noises present on both banks of recordings, this approach failed to yield any fin whale signals.

We were lucky to eventually get some help from M. Giani Pavan, who shared some of his recordings of Mediterranean fin whale songs containing 100 pulses \cite{pavan2004passive}. Despite the limited amount of samples, training a small neural network (see section~\ref{chap:low_archi}) on this data allowed to find similar signals on the Bombyx and Boussole datasets (see section~\ref{chap:data_Toulon}). This demonstrates the capacity of small neural networks to generalise to different recorders even with very few training samples.

Active learning with thumbnail picking further helped increase the database to a satisfactory size (see Tab.~\ref{tab:recap_annot}).


\subsubsection{Dolphin whistles}
\textit{This work has been conducted in collaboration with Marion Poupard.}\\
For this task, as for the fin whale pulses, we used other sources of data available at the lab as a starting point to the active learning process. This time though, the variability of the signals to be detected prevented the use of a low complexity architecture.

Thus, to enforce the generalisation of the model to other recording systems, the available data was augmented with negative samples from the target recording system (Carimam). By mixing annotated foreign inputs with negative samples from Carimam, we teach the model to be robust to common Carimam perturbations (self noise from the sound card, reef noise), while training on positive samples of the target signal. This `mixing' takes form as a simple summation of the waveforms after their standardisation.

Active learning with thumbnail picking further helped increase the database to a satisfactory size (see Tab.~\ref{tab:recap_annot}).


\section{Resulting annotations and train / test splits} \label{chap:splits}

The methods proposed in this section yielded enough annotations to train \ac{ANN} models on each detection / classification tasks (Tab.~\ref{tab:recap_annot}).

\begin{table}[ht]
    \centering
    \begin{tabular}{l|c|c|c}
    \textbf{Target signal} & \textbf{Positives} & \textbf{Negatives} & \textbf{Total} \\ \hline
    Sperm whale clicks &  42\% & 58\% & 5,554 \\
    Fin whale 20\,Hz pulses & 14\% & 86\% & 5,790 \\
    Orca calls & 78\% & 22\% & 6,004 \\
    Humpback whale calls & 42\% & 58\% & 1,377 \\
    Dolphin whistles & 12\% & 88\% & 1,595 \\
    \end{tabular}
    \caption{Summary of the annotations gathered on the data at hand for detection task.}
\label{tab:recap_annot}
\end{table}

The performance measurement methods employed need to reflect our end goal, namely training robust detections models. Robustness, can be defined as the capacity to ignore perturbations, some kind of resilience. In our case, perturbations are sound events and background noises, especially those not seen in training. To measure robustness, our test data must thus contain new acoustic content, somewhat different from training.

The randomly sampled train / test splits often seen in the machine learning community is insufficient in that sense. Indeed, train and test samples will be extracted from the same vocalisation / noise sequences, thus sharing most of their characteristics. On the other hand, choosing a specific source of data, or if not available a distinct time period, should yield novelty in the test set, and give relevant robustness measures for our models.


\subsubsection{Fin whale pulses} \label{chap:rorq_dataset}

The gathered annotated database of fin whale 20\,Hz pulses offers three different data sources (Table~\ref{tab:recap_annot_fin}). Thus, in the experiments, three folds were used : each using two sources for training and the remaining one for testing. The Magnaghi data corresponds to the extracts provided by G. Pavan (see~\ref{chap:fin_activelearning}).

\begin{table}[ht]
    \centering
    \begin{tabular}{l|c|c|c}
    \textbf{Data Source} & \textbf{Positives} & \textbf{Negatives} & \textbf{Total} \\ \hline
    Magnaghi &  15\% & 85\% & 688 \\
    Boussole & 9\% & 91\% & 4,528 \\
    Bombyx & 49\% & 51\% & 574 \\
    \textbf{Total} & 14\% & 86\% & 5,790 \\
    \end{tabular}
    \caption{Distribution of annotations of 20\,Hz fin whale pulses. Each source of data was used as test set in a 3 fold manner.}
\label{tab:recap_annot_fin}
\end{table}


\subsubsection{Sperm whale clicks} \label{chap:cach_dataset}

The annotated database of sperm whale clicks coming from only one source of data (Bombyx), The year 2017 was chosen for testing and the remaining for training (Table~\ref{tab:recap_annot_sperm}). This choice is motivated by the fact that 2015 has too few samples for the test to be relevant, 2016 has a positive / negative distribution too different than the global dataset, and 2018 has the largest amount of samples which is desirable for training. To improve the  annotation comes from separate files.

Experiments showed that the model would tend to lack sensitivity, with the exception of pilot whale samples which would trigger a low specificity. To tackle this issue, and accounting for the imbalance in the data (Tab.~\ref{tab:recap_annot_sperm}), sperm whale and pilot whale samples were over-sampled during training, by a factor 3 and 10 respectively.

\begin{table}[ht]
    \centering
    \begin{tabular}{c|c|c|c|c}
    \textbf{Recording year} & \textbf{Sperm whale} & \textbf{Boat / Noise} & \textbf{Pilot whale} & \textbf{Total} \\ \hline
    2015 &  48\% & 52\% &  & 256 \\
    2016 & 75\% & 23\% & 2\% & 1,383 \\
    2017 & 32\% & 67\% & 1\% & 1,363 \\
    2018 & 28\% & 68\% & 4\% & 2,552 \\
    \textbf{Total} & 42\% & 55\% & 3\% & 5,554
    \end{tabular}
    \caption{Distribution of annotations for the sperm whale click detection task. The year 2017 was used as test set.}
\label{tab:recap_annot_sperm}
\end{table}


\subsubsection{Humpback whale calls}

For the detection of humpback whale calls, the data recorded from Sint Eustatius island was selected as the test set (Table~\ref{tab:recap_hump}).  The Sint Eustatius antenna was chosen for the test set as it has a representative distribution of classes and is neither too small nor too big ($\sim 10\%$).

\begin{table}[ht]
    \centering
    \begin{tabular}{l|c|c|c}
         \textbf{Station} & \textbf{Positives} & \textbf{Negatives} & \textbf{Total} \\ \hline
         Anguilla & 100 & 0 & 100 \\
         Bahamas & 0 & 45 & 45 \\
         Bermude & 276 & 27 & 303 \\
         Guadeloupe & 666 & 26 & 692 \\
         Jamaica & 0 & 11 & 11 \\
         Martinique & 354 & 37 & 391 \\
         Saint Barthélémy & 173 & 0 & 173 \\
         Sint Eustatius & 204 & 103 & 307 \\
         Saint Martin & 163 & 242 & 405 \\
        \textbf{Total} & 67\% & 33\% & 2,427 \\
    \end{tabular}
    \caption{Distribution of humpback whale calls annotations through the Carimam recording stations. The Sint Eustatius data source was used as a test set.}
    \label{tab:recap_hump}
\end{table}


\subsubsection{Dolphin whistles}

For the detection of dolphin whistles, the data recorded from Guadeloupe Breach was selected as test set (Table~\ref{tab:recap_dolph}). 

\begin{table}[ht]
    \centering
    \begin{tabular}{l|c|c|c}
         \textbf{Station} & \textbf{Positives} & \textbf{Negatives} & \textbf{Total}  \\ \hline
         Guadeloupe Breach & 36 & 354 & 390 \\
         Gualdeloupe Anse Bertrand & 0 & 49 & 49 \\
         Saint Barthélémy & 0 & 16 & 16 \\
         Sint Eustatius & 37 & 111 & 148 \\
         Saint Martin & 0 & 34 & 34 \\
         Jamaica & 24 & 10 & 34 \\
         Bonaire & 74 & 25 & 99 \\
         Bermude & 25 & 439 & 464 \\
         Bahamas & 0 & 16 & 16 \\
         Anguilla & 0 & 345 & 345 \\
         \textbf{Total} & 12\% & 88\% & 1,595 \\
    \end{tabular}
    \caption{Distribution of dolphin whistles annotations through the Carimam recording stations. The Guadeloupe Breach data source was used as test set.}
    \label{tab:recap_dolph}
\end{table}


\subsubsection{Orca call detection} \label{chap:orca_dataset}

A special recording session was run at OrcaLab in 2019 by \citet{poupard2021intra}, for the study of group dynamics via triangulation. The manual annotations gathered for this experiment were used in this thesis, bringing an opportunity to measure the impact of a change in recording hardware on detection mechanisms with no additional annotation effort. Two test sets were thus used for the orca call detection task, one from the same antenna than in training but in a different year and one from a different antenna (see Tab.~\ref{tab:recap_orca}). A preliminary study using this dataset was published in a conference paper \cite{best2020deep}.

\begin{table}[ht]
    \centering
    \begin{tabular}{c|c|c|c|c}
         \textbf{Recorder} & \textbf{Year} & \textbf{Positives} & \textbf{Negatives} & \textbf{Total} \\ \hline
         OrcaLab network & 2015 - 2017 & 846 & 3,777 & 4,623 \\\hline
         OrcaLab network & 2019 & 111 & 177 & 288 \\
         \citet{poupard2021intra} & 2019 & 368 & 725 & 1,093 \\
    \end{tabular}
    \caption{Distribution of orca calls binary annotations. The data from 2019 (two different antennas) was used as test set.}
    \label{tab:recap_orca}
\end{table}


\subsubsection{Orca call classification} \label{chap:orca_clf_dataset}

Given the diversity of classes and the singular recording source, for the orca call classification task, the train / test split was simply done by sorting by date and choosing a proportion for test and the rest for train. For instance, the first 10\% of each class were taken for test, and the remaining 90\% were used to train the model.

\begin{figure}
    \centering
    \includegraphics[width=\linewidth]{fig/types.pdf}
    \caption{Examples of each class of orca call types annotated using clusters of \ac{AE} embeddings. The terminology as defined by \citet{ford1987catalogue} has been used by associating calls with their closest class in the catalogue.}
    \label{fig:types}
\end{figure}

\begin{table}[ht]
    \centering
    \begin{tabular}{c|c}
         \textbf{call type} & \textbf{instances} \\ \hline
         N1 & 854 \\
         N2 & 191 \\
         N3 & 192 \\ 
         N4 & 1213 \\
         N5 & 209 \\
         N9 & 609 \\
         N23 & 469 \\
         other & 109 \\
         Noise & 814
    \end{tabular}
    \caption{Distribution of annotations of orca call types \cite{ford1987catalogue}. The `other' class corresponds to infrequent calls that did not have enough occurrences to form an independent class.}
\end{table}


\subsubsection{Antarctic blue and fin whale calls}

Table~\ref{tab:recap_soos} summarises the distribution of labels for each data source available in the Acoustic Trends dataset \cite{miller2021open}. The Kerguelen 2005 data source was chosen as a test set. Its specific recording system and location, as well as its sufficient support of all classes motivated this choice. The remaining recordings were used for training.

\begin{sidewaystable}
\centering
    \begin{tabular}{|c|c|c|c|c|c|c|c|c|c|c|c|} \hline
\textbf{Location} & \textbf{Year} & \textbf{Instrument} & \textbf{Bm A} & \textbf{Bm B} & \textbf{Bm Z} & \textbf{Bm D} & \textbf{Bp 20\,Hz} & \textbf{Bp 20+} & \textbf{Bp DS} & \textbf{Negatives} \\ \hline
Balleny Islands & 2015 & PMEL-AUH &  4\% &  1\% &  1\% &  &  7\% &  2\% &  1\% & 10\% \\ \hline 
Elephant Island & 2013 & AURAL &  10\% &  26\% &  6\% &  71\% &  28\% &  24\% &  16\% & 30\% \\ \hline 
Elephant Island & 2014 & AURAL &  28\% &  14\% &  4\% &  7\% &  38\% &  38\% &  64\% & 6\% \\ \hline 
Greenwich 64S & 2015 & Sono.Vault &  3\% &  2\% &  1\% &  &  &  &  1\% & 1\% \\ \hline 
MaudRise & 2014 & AURAL &  9\% &  1\% &  1\% &  &  &  &  & 3\% \\ \hline 
Ross Sea & 2014 & PMEL-AUH &  &  &  &  &  &  &  & 9\% \\ \hline 
Casey & 2014 & AAD-MAR &  15\% &  20\% &  43\% &  4\% &  &  &  & 8\% \\ \hline 
Casey & 2017 & AAD-MAR &  7\% &  8\% &  5\% &  4\% &  1\% &  3\% &  & 8\% \\ \hline 
Kerguelen 1 & 2005 & ARP &  6\% &  3\% &  7\% &  3\% &  6\% &  1\% &  7\% & 9\% \\ \hline 
Kerguelen 2 & 2014 & AAD-MAR &  10\% &  17\% &  22\% &  3\% &  15\% &  24\% &  5\% & 8\% \\ \hline 
Kerguelen 2 & 2015 & AAD-MAR &  8\% &  8\% &  9\% &  8\% &  4\% &  9\% &  5\% & 8\% \\ \hline 
\textbf{Total} & & & 25,177 & 6,903 & 2,515 & 15,339 & 12,933 & 7,761 & 6,381 & 357,765 \\ \hline 
    \end{tabular}
    \caption{Distribution of annotations published by \citet{miller2021open}. The Kerguelen 2005 was chosen as test set.}\label{tab:recap_soos}
\end{sidewaystable}


\section{Discussion}

As seen throughout this chapter, techniques employed in pre-detection, feature extraction and filtering need to be adapted to the type of target signal and the available recordings. For that matter, Tab.~\ref{tab:steps} recapitulates the choices made for each of the 6 annotation procedures conducted.

\begin{table}[ht]
    \centering\resizebox{\linewidth}{!}{    
    \begin{tabular}{c|ccc}
         \textbf{Target signal} & \textbf{Pre-detection} & \textbf{Feature extraction} & \textbf{Filtering} \\ \hline
         sperm whale clicks & TK filter & \ac{TDOA} & custom \ac{UI} \\ 
         humpback whale calls & NA & spectral features & custom \ac{UI} and clustering \\
         orca call detection & spectrogram thresholding & region statistics & hand-crafted rules \\ 
         orca call classification & \ac{CNN} & auto-encoder & clustering \\ \hline
         20\,Hz fin whale pulses & \multicolumn{3}{c}{transfer learning} \\
         dolphin whistles & \multicolumn{3}{c}{transfer learning} \\
    \end{tabular}}
    \caption{Summary of steps employed in the initial annotation process of each target signal.}
    \label{tab:steps}
\end{table}

Let us get an idea of the time it would have taken to annotate the sperm whale click database via random sampling for instance. Sperm whales were confirmed on 6\% of the files from Bombyx (see section \ref{chap:cacha_presence}). If we consider 30 seconds to manually check a file (between 1 and 5 minutes long), to yield the 2,300 positive labels collected here (they are each on separate files), one would need 320 hours. It took approximately 40 hours in total collect this database with the annotation approach described in \ref{chap:bombyx_annot}.

The following paragraphs summarise the advantages of some methods employed through these experiments, along with potential pitfalls to be kept in mind.

\subsection{Active learning}

Active learning has proven to be very efficient in iteratively increasing database sizes. It can be started as soon as few dozen annotations are at hand. Indeed, in that case, rather than spending time in tuning pre-detection mechanisms to collect more samples, deep learning models help to collect occurrences of the target signal as well as disruptors (e.g. boats, signals from other species). Moreover, it is worth the efforts of developing the training procedures since they will be used subsequently, as opposed to the pre-detection algorithms which are rather a one time usage.

Nonetheless, there is a danger that comes along with active learning: the progressive specialisation of the model to detect only one type of signal. Indeed, if the initial annotations omit some type of signals from a target species, it is likely that the model will never learn to detect them. This especially comes from the fact that we often only correct the positive predictions of the model, sorting out false positives (negative predictions usually come in much larger numbers, making it fastidious to find false negatives). To mitigate this effect, one can manually browse recordings around detections and annotate full sequences, or look for false negatives in low confidence negative predictions.


\subsection{Thumbnail picking}

Thumbnail picking allows to quickly validate detections or clusters to collect annotations. The only condition is to find a visualisation that fits a small size and still allows to make a decision on sample's classes (small spectrograms work well for most stationary signals, see Fig.~\ref{fig:png_annot}). It is versatile and easily shareable to experts (the only prerequisite is to have a graphical file manager). It is fast and user friendly: you just need to click on files to select them and move them to a separate folder (cut and paste). Also, seeing multiple samples at once strongly helps the eye in discriminating singularities.

This last advantage can also be dangerous in the annotation process. Indeed, when sorting large folders to try and keep only a class of call, one might always see similar calls at a time on the screen, but through scrolling, pitch contour shapes might shift progressively. When this occurs, the annotator might feel like all the calls in the folder are similar, when in fact, the ones at the beginning are very different from the ones at the end. To mitigate this, indexes should be randomly permuted, since this progressive shift in call contour is likely to occur if files are sorted time wise, but very unlikely otherwise.


\subsection{`Generic' spectral features extraction}

Section \ref{chap:HB_annot} proposes a procedure suited to explore large banks of recordings by grouping events with similar content in terms of frequency energy distribution. It allowed to collect a first database of humpback whale calls. The success of such approach relies on several assumptions:
\begin{itemize} \setlength{\itemsep}{1pt}
    \item A minimal knowledge of the target signals is needed to configure the algorithm (e.g. frequency range, length).
    \item Events are grouped independently of the temporal distribution of the energy in the spectrogram (e.g. upwards and downwards chirps will yield the same features). This is suited to discriminate between events of different temporal support, but would not work to discriminate some pitch patterns.
    \item For the projection and the clustering to reveal a group of events, a sufficient number of instances are needed. This is the most probable explanation of why we failed to retrieve dolphin whistles and click trains using this method.
\end{itemize}
\chapter{Training detection and classification mechanisms}
\label{chap:training}
\minitoc


\section{Context and objective}

The gathered annotations previously mentioned represent an important step towards the objective of this thesis: building robust detection and classification mechanisms for several target signals. For that purpose this chapter discusses \ac{ANN} training in a supervised learning context. The detection of sperm whale clicks and fin whale 20\,Hz pulses is first experimented with a constraint on computational cost (in order to be embedded in a sono-buoy, see section \ref{chap:GIAS}) . For that matter, the effect of several complexity reduction approaches is studied. Then, heavier models are used to detect Antarctic mysticetes and orca calls. Experiments focus on the effect of network frontends, architectures and hyper-parameters on performances. Furthermore, given orca call detections, trials with deep metric learning and semi-supervised learning are reported for the call type classification task.


\section{Light weight detectors}
\label{chap:lightweight}

The initial funding of this PhD was oriented towards the implementation of a real time alert system for the presence of large cetaceans in the Ligurian (Mediterranean) sea (GIAS Project). This system takes form as a battery powered sono-buoy with acoustic and processing capacities.

Motivated by the objective of deploying detection mechanisms into this embedded systems with low computing capacity, several complexity reduction approaches have been experimented with. Some measures will be given according to the specific embedded \ac{MCU} of the buoy: the PIC32 by MicroChip.

Two large cetacean species evolve in the Ligurian sea, and therefore are to be detected by the system: sperm whales and fin whales. Two target signals are thus concerned by the following experiments on low computational detection: sperm whale clicks and fin whale 20\,Hz pulses.

This section first reports on experiments with three complexity reduction approaches (depth-wise convolution, weight pruning and weight quantisation), comparing their computational needs and performance. Then, with the chosen approach of depth-wise convolution, we investigate on optimal number of features per layer and kernel sizes via a grid search. Finally, the two selected detection mechanisms are compared with baseline algorithms of the literature.


\subsection{Complexity reduction}

The base architecture for the following experiments is a 3 layer network of 1D convolutions. It takes 64 bins Mel-spectrograms as an input :

\begin{itemize} \setlength{\itemsep}{1pt}
    \item Sperm whale clicks: $f_s=50$\,kHz, $NFFT=512$, $hop=256$, $f_{min}=2$\,kHz, $f_{max}=25$\,kHz
    \item Fin whale 20\,Hz pulses: $f_s=200$\,Hz, $NFFT=256$, $hop=32$, $f_{min}=0$\,Hz, $f_{max}=100$\,Hz
\end{itemize}

Following \citet{schluter2017deep}, the spectrograms are compressed with $\log(1 + \mathbf{S} \times 10^a)$ with $a$ a trainable parameter.

The frequency bins (spectrogram rows) are considered as input channels for the first 1D convolution. This choice was motivated by the fact that large spectral shifts are not expected for these target signals. Convolving frequency-wise is thus inappropriate. Using 1D convolution also significantly reduces training and inference time.

The following experiments make use of the annotated databases described in section \ref{chap:cach_dataset}.


\subsubsection{Depth-wise layers} \label{chap:low_archi}

As demonstrated in section \ref{chap:sota_dw}, using depth-wise separable convolutions is an efficient way of reducing the amount of multiplications needed in neural network systems. Fig.~\ref{fig:forward_nmult} compares the number of multiplications needed for an inference with regular convolution networks and depth-wise separable networks. The lower bound complexities are of $O(n^2)$ and $O(n)$ respectively (with n the number of features per layer).

\begin{figure}
    \centering
    \includegraphics[width=.6\linewidth]{fig/forward_time.pdf}
    \caption{Number of multiplications needed per forward pass against the number of features per layer, for two types of architecture (solid lines). The number of multiplications were estimated for a 3 1D convolutions architecture (64 channeled input and single channeled output), stride of 1, and a kernel of size 4. Estimated inference time on the PIC32 \ac{MCU} are also given (dashed lines).}
    \label{fig:forward_nmult}
\end{figure}


\subsubsection{Weight pruning}

In \acp{ANN}, weight pruning consists in putting to 0 a proportion of weights after training \cite{lecun1989optimal} (e.g. the ones with the smallest L1 norm). The idea is to avoid computing multiplications for weights that are of low impact for the end prediction. Experiments were conducted to measure the effect of pruning as compared to reducing the number of features per layer before training (see Fig.~\ref{fig:pruning}).

\begin{figure}
    \centering
    \includegraphics[width=.6\linewidth]{fig/pruning.pdf}
    \caption{\ac{AUC} performance on the sperm whale click detection task before and after pruning. Models consisted in 3 depth-wise layers with varying numbers of features (each randomly initialised 5 times). Green boxes denote the performance of models before pruning, with 16, 32, 64, and 128 features per layer. For each of them, pruning was applied over 10\%, 20\%, 30\%, and 40\% of the weights, whose performances are shown in white boxes.}
    \label{fig:pruning}
\end{figure}

For the model with 32 features per layer, pruning until 20\% included had a non-significant effect. As for the larger models, performances were impacted starting from 20\% of pruning. Pruning can therefore be considered a relevant option to reduce the complexity of \ac{CNN} detection systems, but can only offer a marginal gain (between 10 and 20\% of multiplications can be avoided).


\subsubsection{Weight quantisation}

The type of variable in a multiplication has an important impact on the cost of the operation. For instance, on the target \ac{MCU} of the GIAS project (see section \ref{chap:GIAS}), the PIC32 from Microchip, a multiplication of two floating point variables takes 736\,ns while multiplying two 8 bit integers takes 48\,ns \cite{pic32_bench} (a factor 16 of difference). Fig.~\ref{fig:forward_nmult} compares inference times on the PIC32 for a depth-wise architecture of floating points against a regular convolutional architecture of 8 bit integers.

Weights were thus quantised to 8 bit integer variables in an attempt to reduce computation time. To do so, using the Pytorch \cite{NEURIPS2019_9015} quantisation module, inputs weights and activations are quantised after training regularly with floating point numbers (post-training quantisation). Nonetheless, inference on a subset of the dataset is conducted to calibrate the quantisation parameters for the activations and mitigate information degradation. This quantisation approach was experimented on 3 layer architectures with regular convolution and varying number of features (Fig.~\ref{fig:quat_aucs}).

\begin{figure}
    \centering
    \includegraphics[width=.6\linewidth]{fig/quat_aucs.pdf}
    \caption{Performance for sperm whale clicks detection, before and after quantisation to 8 bits integers. 3 layer regular convolution architecture were trained 5 times for each configurations. \ac{AUC} are given for the test set (see section \ref{chap:cach_dataset})}
    \label{fig:quat_aucs}
\end{figure}

The quantisation procedure appeared to have a non-significant impact on performance (the Kruskal-Wallis H test between the two distributions gave p-values > 0.1). Quantisation can thus be a relevant approach to the complexity reduction of models.


\subsubsection{Conclusion}

The depth-wise approach shows a significant complexity reduction, even with floating point weights numbers, and this until 16 features per layer (Fig.~\ref{fig:forward_nmult}). At 128 features per layer (the chosen configuration for fin whale 20\,Hz pulse detection), such architecture yields an inference 50 times faster than a regular convolutional one, and 5 times faster than its quantised version. Depth-wise convolutions has thus been retained for the detection systems of sperm whale clicks and fin whale pulses, the two target species of the GIAS project (section \ref{chap:GIAS}).

Implementing quantised and pruned depth-wise architectures would have been possible, but appeared to be demanding in development efforts. Moreover, as section \ref{chap:GIAS} shows, the main cost of the buoy embedded analysis lies in the spectrogram computation rather than in the model inference (given the already reduced complexity of the \ac{CNN}). Accounting for this, no further efforts were put into researching complexity reduction for these detection systems.


\subsection{Hyper-parameter search}

With the chosen 3 layer depth-wise architecture, experiments were conducted to select the optimal kernel sizes and number of features per layer. These small neural networks being quite fast to train (less than 5 seconds per epoch using the \ac{GPU}), a simple exhaustive search is possible. They are summarised in Fig.~\ref{fig:bench_rorq} and Fig.~\ref{fig:bench_cach}. Networks were trained with batch normalisation, dropout ($p=0.25$) and leaky rectifier units after the two first convolution layers. Learning rate and weight decay were manually tuned before training with varying numbers of features and kernel sizes. Kernel size and number of feature per layer were chosen to study as they were found to have the largest impact on computation cost and performances.

\begin{figure}
\centering
\includegraphics[width=\linewidth]{fig/rorq_aucs_vs_kernelsize&feats.pdf}
\caption{\ac{AUC} performance for the 20\,Hz fin whale pulse detection task. Depth-wise architectures have been experimented with several combinations of hyper-parameters (number of features per layer and kernel size). For each configuration and train/test fold, 5 runs were conducted. Folds are labelled with their test set (Bombyx scores report the performance of models trained on Magnaghi and Boussole data.}\label{fig:bench_rorq}
\end{figure}

On the fin whale 20\,Hz pulse detection task, the Magnaghi test set showed a great variability to multiple network initialisation, even with the same hyper-parameters. This is perhaps a consequence of specific recording setup properties, or a large gap between convergence points accounting to the two different training sources. On the two remaining folds however, performance is relatively resilient to hyper-parameter choice and initialisation. Performances of 0.99 \ac{AUC} seem satisfactory for the test set. 

\begin{figure}
\centering
\includegraphics[width=\linewidth]{fig/cach_aucs_vs_kernelsize&feats.pdf}
\caption{\ac{AUC} performances for the sperm whale clicks detection task. Depth-wise architectures were experimented with several combinations of hyper-parameters (number of features per layer and kernel size). For each configuration and train/test fold, 5 runs were conducted.}\label{fig:bench_cach}
\end{figure}

As for the sperm whale click detection, larger kernels and deeper layers (number of features) appeared to induce some overfitting. For some configurations however, the depth-wise architectures, despite a lower amount of parameters, yield performances similar to those of regular \acp{CNN} (Fig.~\ref{fig:quat_aucs}).

For the following experiments, the architecture with kernels of size 5 and 32 features per layer was retained for the sperm whale click detection, and kernels of size 5 with 128 features per layer was retained for 20\,Hz fin whale pulse detection.


\subsection{Baseline comparison}

The performances reported in the last section only have value relatively to that of previous systems (baselines). This section first reports on a common technique used in sperm whale click detection: the \ac{TK} filter. Then, two experiments were conducted to validate the 20\,Hz fin whale pulse detection procedure: comparison to a commonly used template matching method and comparison to a state-of-the-art \ac{ANN} based system on an unseen dataset.


\subsubsection{\ac{TK} filter (sperm whale clicks)}

The chosen baseline for the sperm whale click detection is inspired from the work of \citet{ferrari2020study}. It makes use of the \ac{TK} energy operator to find impulses, before filtering them by an estimation of the background noise with a rolling median.

This algorithm was used on the whole dataset of sperm whale clicks for comparison with \ac{ANN} performances. Using the maximum energy value of samples as prediction, the \ac{AUC} score was of 0.86, around 0.07 points below most of the trained depth-wise models (Fig.~\ref{fig:roc_cach}). This translates to, for instance if we fix a 10\% fall-out (false positive rate), a precision of 62\% for the \ac{TK} filter, against 82\% in average for the depth-wise models.

\begin{figure}
    \centering
    \includegraphics[width=.7\linewidth]{fig/rocs_stft_depthwise_ovs_64_k7.pdf}
    \caption{\ac{ROC} curves for the sperm whale click detection task. Performances are given for the \ac{TK} filter (baseline) and for the 5 initialisations of the 3 layer depth-wise architecture (median $\pm$ standard deviation).}
    \label{fig:roc_cach}
\end{figure}


\subsubsection{Different base for spectrogram computation}

Through numerous research, the scientific community has looked for alternatives to the Fourier transform as feature extraction before the main neural network. The sinus base the \ac{FFT} offers seems too generic, not suited for particular signals such as the transient sperm whale clicks. Experiments were thus conducted using the sincnet frontend proposed by \citet{ravanelli2018speaker} which is based on cardinal sinuses with trainable cut frequency. Performances never exceeded 0.86 of \ac{AUC} on the sperm whale click database (6 points below average performances of \ac{FFT} based models).


\subsubsection{Template matching (20\,Hz fin whale pulses)}

As mentioned in section \ref{chap:template}, spectrogram correlation is a common approach for cetacean signals detection, especially for mysticetes. To compare our \ac{ANN} system with this baseline, we built a template of fin whale 20\,Hz pulse by averaging the Mel-spectrogram of all annotated pulses in the training set. We then threshold on the cross-correlation product of samples with the template. The resulting detection performances are presented in Figure~\ref{fig:roc_rorq}. The AUC of the template matching method is 0.898 (5 to 10 points less than the CNN model, depending on the fold).


\subsubsection{Larger \ac{ANN} architecture (20\,Hz fin whale pulses)}

The dataset published by \citet{madhusudhana2021improve} which also studies a \ac{CNN} based fin whale 20\,Hz pulse detection seems relevant to test this thesis' proposed system on foreign data. The resulting \ac{mAP} and peak F1-score are 0.96 and 0.88, when the best overall performances of the study are 0.95 and 0.91 respectively (note that the dataset published is only a subsample of the dataset used in the study, and thus scores are not reliably comparable). This demonstrates that the proposed model generalises well to new data, with scores comparable to a larger architecture that exploits the sequentiality of the pulses. % Moreover we obtain comparable performances to an approach with 33\% more parameters and which exploits the sequentiality of the pulses by using recurrent network layers (thus introducing more complex inductive biases).

\begin{figure}[!htb]
    \centering
     \includegraphics[width=.7\linewidth]{fig/rocs.pdf}
     \caption{\ac{ROC} curves for fin whale 20\,Hz pulse detection over each test set (the two remaining sources serving as training set, see section \ref{chap:rorq_dataset} for details). Performances of the template matching method and over the dataset published by \citet{madhusudhana2021improve} are also displayed.}\label{fig:roc_rorq}
\end{figure}


\subsubsection{Conclusion}

To challenge this thesis' choice of architecture, handcrafted algorithms, a different frontend, and tests on foreign data were implemented. All results comfort the fact that the \ac{FFT} based depth-wise architectures are successful at the task, and that with a relatively low computational cost, they show better performances than handcrafted algorithms.


\section{Deeper and wider models}

The remaining target signals treated in this thesis present more variability than sperm whale clicks and fin whale 20\,Hz pulses. Larger architectures than simple 3 depth-wise convolutions were thus experimented. We followed the community by opting for the ResNet architecture, widely used in image and sound classification tasks, and the most used for bioacoustics applications \cite{stowell2021computational}.

Note that when using ResNet architectures, the last layers consist of an average pooling of the spatial dimensions, followed by a fully connected layer (with the number of output channels set to the number of target classes). In bioacoustic applications, it is often more convenient to yield a sequence of predictions through time rather than one prediction regardless of the size of the input spectrogram. To retrieve this behaviour while conserving the main ResNet architecture, one can discard the average pooling and replace the fully connected layer by a 1x1 convolutional layer (kernel of size 1).

During training, the sequence of predictions can be max-pooled before the loss computation. Max pooling is more suited than average pooling for detection (or multi-label classification) tasks since we want the prediction to be invariant to the amount of void surrounding a target signal. In other words, whether there is one or 10 calls in the input, the detection should be the same: it denotes the presence of at least one event in the window. Note that when using a max-pooling layer, during back-propagation, only the temporal frame with the maximal prediction serve the gradient computation.

In this section, experiments study the effect of the choice of frontend (especially spectrogram range compression), architecture (among ResNet-18, ResNet-50 and sparrow \cite{grill2017two}), training hyper-parameters and evaluation metric. In these regards, it intends to assist decision making, by discussing on their impact to solve two detection tasks (orca calls and Antarctic mysticetes calls).


\subsection{Hyper-parameter search for orca call detection} \label{chap:orca_detec}

Contrary to the smaller architectures aforementioned, heavier models need around 1min per epoch on the orca call detection dataset (see section \ref{chap:orca_dataset}). An automatic hyper-parameter search was thus employed using \ac{ASHA} \cite{li2020system}, implemented by the Ray python package \cite{moritz2018ray}. It uses the hyperband algorithm  with successive halving to explore the hyper-parameter search space, with aggressive early stopping of low performing models. Moreover, to optimise computations, models with plateauing performance are also stopped rather than trained until the maximum number of epochs is reached.

Hyper-parameter combinations are drawn from the following search space:
\begin{itemize} \setlength{\itemsep}{1pt}
    \item Learning rate (log uniform distribution between 0.00001 and 0.1)
    \item Weight decay L2 loss (log uniform distribution between 0.00001 and 0.1)
    \item Batch size (sampled uniformly from [8, 16, 32, 64, 128])
    \item Weighting of positive samples in the loss computation (uniform distribution of integers between 1 and 5)
    \item Brown noise data augmentation (boolean)
    \item MixUp data augmentation (boolean)
    \item SpecAugm \cite{park2019specaugment} spectral data augmentation (boolean)
    \begin{itemize}
        \item maximum frequency dilation for SpecAugm (uniform distribution between 1\% and 30\%)
        \item maximum temporal dilation for SpecAugm (uniform distribution between 1\% and 30\%)
        \item maximum mask height (number of frequency bins) for SpecAugm (uniform distribution between 10 and 50)
        \item maximum mask width (number of time bins) for SpecAugm (uniform distribution between 10 and 50)
    \end{itemize}
\end{itemize}

Several architectures are studied: sparrow \cite{grill2017two} (simple \ac{VGG}-like model) and ResNet-18 models (one randomly initialised and one pretrained on ImageNet noted `resnetPT'). For each of the 3 possible architectures, logarithmic and \ac{PCEN} spectrogram range compression were tested, yielding 6 independent hyper-parameter searches. The searches were ran independently in order to have a fair comparison of the 6 types of models: each have their hyper-parameters optimised via a systematic procedure with a fixed computational budget.

The main objective of this study is to compare architectures on their best possible performance on the test set (both same antenna and different antenna). This is why no validation set was kept apart, and the whole test set \ac{mAP} was used for early stopping (both low performing and plateauing trainings), and making halving decisions.

Nonetheless, in the following, scores of the two sets are reported separately. Indeed, we will see that a change in recording system (with a different frequency response) can introduce a performance drop. To emphasis on this generalisation problem, we report performance separately on a close test set (same antenna than seen in training) and a foreign test set (different antenna).

The search algorithm was run with 100 trials, for all architectures and range compression combinations independently. This allows for a fair comparison of the architectures, each having their hyper-parameters optimised in a systematic way. 

\begin{figure}
    \centering
    \includegraphics[width=\linewidth]{fig/mAP_archis.pdf}
    \caption{Test \ac{mAP} for the two test sets of orca call detection. Scores of the 50 best trials os the \ac{ASHA} search are given for each combination of architecture and spectrogram range compression.}
    \label{fig:perfs_orcas}
\end{figure}

Figure \ref{fig:perfs_orcas} summarises the resulting performance of the 100 trials for the two test sets. The sparrow architecture appears more resilient to the choice of hyper-parameters, especially with the \ac{PCEN} range compression. The pcen-sparrow models reach the best scores, with an especially strong performance gain on the foreign test set (different antenna), demonstrating generalisation capabilities. These findings will be further studied in section \ref{chap:orca_valid}, with repeated initialisation with the best set of hyper-parameter for each of the architectures.


\subsubsection{Impact of hyper-parameters on model performances}

To learn insights from this systematic search, correlations were measured between hyper-parameters and the resulting model performances.

\begin{table}[ht]
    \centering\resizebox{\linewidth}{!}{
    \begin{tabular}{|c|c|c|c|c|c|c|}\hline
    archi & posweight & batchsize & lr & augm & mixup &  brownnoise \\ \hline 
    logMel - resnet&&-0.240&&False 0.37&& \\ \hline 
     logMel - resnetPT&&&&&False 0.06& \\ \hline 
     logMel - sparrow&&-0.216&0.371&False 0.25&&True 0.16 \\ \hline 
     pcen - resnet&&&&False 0.06&False 0.08& \\ \hline 
     pcen - resnetPT&&&&False 0.10&& \\ \hline 
     pcen - sparrow&&&0.312&&& \\ \hline 
    \end{tabular}}
    \caption{Statistical analysis of the impact of hyper-parameters on model performances (test \ac{mAP}). For numeric variables (posweight, batchsize, and lr), the Pearson correlation was computed, and its coefficient is reported for p-values < 0.05. For boolean variables (augm, mixup, brownnoise), the Kruskal-Wallis H-test was computed, and the beneficial value along with medians difference are reported for p-values < 0.05. Empty slots denote p-values below 0.05. }
    \label{tab:stats_orcas}
\end{table}

Table \ref{tab:stats_orcas} reports the statistically significant hyper-parameters on the end model performances (p-value < 0.05). Hyper-parameters appeared to have identical impacts on the same antenna and different antenna test sets, and thus the analysis was conducted on the combination of the two. This representation yields several insights:
\begin{itemize} \setlength{\itemsep}{1pt}
    \item Smaller batch sizes can improve generalisation. This is consistent with the study by \citet{kandel2020effect}. It is especially relevant for small datasets, where large batch sizes imply a reduced variability of batch compositions which can yield overfitting models.
    \item As for the learning rates, several biases have to be taken into account. A small learning rate implies slower training, and thus could be early stopped by the search algorithm before they would plateau to their top performance. Moreover, if selecting the learning rates above 0.001, the Pearson correlation coefficient changes sign with a higher p-value ($r=-0.1$, $p_{value}=0.06$).
    \item SpecAugment surprisingly not only does not improve generalisation but reduces it, despite the joint optimisation of augmentation strength. This is presumably related to the underfitting problem reported by the SpecAugment authors \cite{park2019specaugment}. Indeed, data augmentation can make learning `harder', and thus demand longer trainings and / or heavier models. Note that longer trainings are especially disadvantageous in this paradigm of hyper-parameter search with early stopping.
    \item Other hyper-parameters do not have a clear significant impact on end performances.
    %\item Weighting positive samples, MixUp data augmentation, and the addition of brown noise sometimes help, but do not yield higher performances in a statistically significant manner in most cases.
\end{itemize}


\subsubsection{Search findings validation} \label{chap:orca_valid}

\begin{table}[ht]
    \centering\resizebox{\linewidth}{!}{
    \begin{tabular}{l|cccccc}
     \textbf{Frontend} & logMel & logMel & logMel & \ac{PCEN} & \ac{PCEN} & \ac{PCEN} \\ 
     \textbf{Architecture} & resnet & resnetPT & sparrow & resnet & resnetPT & sparrow \\ \hline
     \textbf{Batchsize} & 8 & 8 & 128 & 64 & 128 & 32 \\
     \textbf{Learning rate} & $8\text{e-}3$ & $7\text{e-}4$ & $2\text{e-}3$ & $2\text{e-}2$ & $1\text{e-}2$ & $4\text{e-}2$ \\
     \textbf{Weight decay} & $4\text{e-}4$ & $9\text{e-}3$ & $8\text{e-}5$ & $1\text{e-}2$ & $1\text{e-}3$ & $2\text{e-}2$ \\
     \textbf{Posweight} & 4 & 3 & 1 & 5 & 3 & 1 \\
     \textbf{Brown noise} & False & False & True & True & False & True \\
     \textbf{SpecAugment} & False & False & False & False & False & True \\
     \textbf{MixUp} & False & False & True & False & True & True \\
     \textbf{\# epochs} & 6 & 13 & 9 & 5 & 6 & 5 \\
     \textbf{Same antenna} & 0.98 & 0.97 & 0.98 & 0.99 & 0.99 & 0.99 \\
     \textbf{Different antenna} & 0.95 & 0.90 & 0.91 & 0.96 & 0.95 & 0.98 \\
    \end{tabular}}
    \caption{Best scoring hyper-parameters resulting from the \ac{ASHA} search of 100 trials for each frontend / architecture combination. Corresponding \ac{mAP} scores are given for the two test sets.}
    \label{tab:orca_bestHP}
\end{table}

To follow up on this hyper-parameter exploration and validate its findings, using each architecture's best scoring hyper-parameters (see Tab.~\ref{tab:orca_bestHP}), 5 training procedures were run with random initialisation. Performances of the latter are displayed on Fig.~\ref{fig:perf_best}. These results reveal several insights:
\begin{itemize} \setlength{\itemsep}{1pt}
\item The pretrained ResNet (`resnetPT') shows a lower performance than its random initialised relative. For that matter, it is worth mentioning that the first convolutional layer had to be replaced prior to training (switching from a 3 channel input to a single channel input). As a result, the pre-learnt projection at initialisation might be dysfunctional, and even counterproductive for final convergence.
\item For the remaining architectures (ResNet and sparrow), \ac{PCEN} yields a performance more resilient to random initialisation (smaller variance), and show significantly improved performance. This will be studied in greater details in the next section.
\item Comparing sparrow and ResNet given \ac{PCEN} normalised spectrograms, sparrows gives a more stable higher performance. One possible explanation for this is the total number of weights of the architectures. Sparrow has around 300k trainable parameters, and the ResNet-18 has 11M. With a relatively small datasets like this one, smaller models might decrease the risk of overfitting.
\end{itemize}

\begin{figure}
    \centering
    \includegraphics[width=\linewidth]{fig/best_trains.pdf}
    \caption{Distribution of performances after 5 runs on the best scoring hyper-parameters of each architecture. Best scoring hyper-parameters were tuned systematically using the \ac{ASHA} algorithm for 100 trials on each architecture independently.}
    \label{fig:perf_best}
\end{figure}


\subsubsection{\ac{PCEN} beneficial behaviour} \label{chap:pcen}

The \ac{PCEN} range compression procedure appeared to be beneficial with some but not all datasets. For the orca call detection task, it appeared to be beneficial (Fig.~\ref{fig:pcenvslogmel}). To verify the significance of the impact of \ac{PCEN}, a statistical analysis was run to compare the two distribution of scores. To discard low performing models that were early stopped by the search algorithm, only the top 50\% of the scores were kept for each distribution. 

The two distribution were significantly different (Kruskal-Wallis H test, p-value < 0.001) and the gain in performance was higher for the test set from the different antenna (median gain of 0.03 and 0.08 of \ac{mAP} for the same antenna and the different antenna test sets respectively).

\begin{figure}
    \centering
    \includegraphics[width=.6\linewidth]{fig/pcenvslogmel.pdf}
    \caption{Distribution of performances on the orca call detection task depending on spectrogram range compression. Scores are taken from trials of the systematic \ac{ASHA} hyper-parameter search (all architectures are grouped together). For reach frontend, only the top 50\% scores are reported.}
    \label{fig:pcenvslogmel}
\end{figure}

The trainable parameters ($s$, $\delta$, $\alpha$ and $r$) remained stable around their initialisation value for a large majority of the training runs. This was not the case during experiments with other datasets such as the Antarctic blue and fin whale vocalisations, where the \ac{PCEN} parameters appeared to diverge towards irrelevant values (see section \ref{chap:pcen2}). On this orca call detection dataset however, \ac{PCEN} significantly improves generalisation, especially facing domain shift (foreign test set). This result is consistent with the study by \cite{allen2021convolutional} on humpback whale vocalisation detection.


\subsection{Experiments on a large public dataset (Antarctic mysticetes)}
\textit{This work has been subject to a workshop intervention \cite{bestdclde}.}

The Antarctic mysticete dataset (introduced in section \ref{chap:acTrend}) offers two main opportunities: its public aspect allows a common mean of evaluation for detection systems among researchers, and its large size enables this evaluation to be the most relevant. Indeed, as Table \ref{tab:recap_soos} summarises, annotations come in large numbers (close to 80k in total, 2.5k for the least represented class) and are spread across multiple recording locations, devices and years. As discussed earlier in section \ref{chap:valid}, this gives us a chance to learn robust models and measure their generalisation capabilities.

\begin{table}[ht]
    \centering
    \begin{tabular}{c|c|c|c|c}
         \textbf{Spectrogram} & \textbf{Architecture} & \textbf{SpecAugm} & \textbf{Train \ac{mAP}} & \textbf{test \ac{mAP}}\\ \hline
         logarithm & sparrow & no & 0.47 & 0.37 \\
         logarithm & ResNet-18 & no & 0.86 & 0.54 \\
         logarithm & ResNet-50 & no & 0.84 & \textbf{0.66} \\
         \ac{PCEN} & ResNet-50 & no & 0.82 & 0.57 \\
         fixed \ac{PCEN} & ResNet-50 & no & 0.80 & 0.58 \\
         logarithm & ResNet-50 & yes & 0.70 & 0.60
    \end{tabular}
    \caption{Experiments on spectrogram range compression, architecture, and data augmentation for the detection of Antarctic mysticetes calls. \ac{mAP} scores are computed over each class independently before averaging to ignore class imbalance.}
    \label{tab:mysti_bench}
\end{table}

With this dataset at hand, several architectures were first experimented, with trials on different spectrogram range compression and data augmentation. They are summarised in Tab.~\ref{tab:mysti_bench}, and demonstrate several insights:
\begin{itemize} \setlength{\itemsep}{1pt}
    \item Non residual architectures such as sparrow don't have the capacity to learn even the training set,
    \item The larger architecture (ResNet-50) generalises better to the test set,
    \item Spectral data augmentation produces underfitting,
    \item \ac{PCEN} normalisation, whether with trainable or fixed parameters, decreases generalisation.
\end{itemize}

The next section will try and get a sense of the latter insight which goes against the observations on the orca call detection dataset (section \ref{chap:pcen}).


\subsubsection{\ac{PCEN} unfavorable behaviour} \label{chap:pcen2}

A reasonable hypothesis of why \ac{PCEN} appears counter productive is that it filters the long stationary signals of the blue whale (10 to 15 seconds long). In \ac{PCEN}, the $s$ parameter describes the coefficient of the \ac{IIR} filter, which yields the smoothed version of the spectrogram $\mathbf{M}$. $\mathbf{M}$ is then used to withdraw background noise from the input $\mathbf{S}$ (Eq.~\ref{eq:pcen2}).

Accounting for this, we want the \ac{IIR} to have a high enough time constant $\tau = \frac{-1}{\log(1-s)}$. Indeed, the time constant of a filter is the time it needs to reach $1-\frac{1}{e} \approx 0.63$ given an logical gate input \cite{liptak2003instrument} (we could make the analogy with the blue whale calls being logical gates on their frequency bin). Using this relationship, with $s=0.01$, it takes 13 seconds for $\mathbf{M}$ to integrate 63\% of the energy of $\mathbf{S}$. Figure \ref{fig:trained_compression} illustrates this effect of $s$ on \ac{PCEN} normalisation and compares it to the log compression.

This value of $s=0.01$ seemed sufficient to avoid withdrawing too much of the blue whale calls, and was used to train a model with a non-trainable (`fixed') \ac{PCEN}. On the other hand, the intuition is that if a better value exists, the trainable $s$ would converge to it during optimisation.

\begin{figure}
    \centering
    \includegraphics[width=\linewidth]{fig/compressions.pdf}
    \caption{Comparison of the different range compression approaches. All spectrograms come from the same sample containing a Bm-A call. For log compression, $a$ converged to 0.3 during training. For \ac{PCEN}, we show how a too high value for $s$ can lead to the reduction of some target signals. The remaining \ac{PCEN} parameters were left to the default values proposed by \citet{wang2017trainable}.}
    \label{fig:trained_compression}
\end{figure}

Unexpectedly, the trainable \ac{PCEN} $s$ parameter converged towards 0.9, an almost instantaneous smoothing coefficient, high enough to integrate blue whale calls in the smoothed spectrogram $\mathbf{M}$ and subtract them from $\mathbf{S}$. The other trainable parameters $\alpha$, $\delta$, and $r$ converged around 0.94, 1, and 0.94 respectively. Considering these parameters (the smoothed spectrogram $\mathbf{M}$ being approximately equal to $\mathbf{S}$ with $s\approx1$), the \ac{PCEN} equation can be rewritten as Eq.~\ref{eq:pcen2}.

\begin{equation}\label{eq:pcen2}
    \mathrm{PCEN}_{t, f} = \left(\frac{\mathrm{S}_{t,f}}{(\epsilon+\mathrm{M}_{t,f})^\alpha} + \delta \right)^r - \delta^r \approx \mathrm{S}_{t,f}^{0.06}
\end{equation}

As for the fixed version, the smoothing parameter was set to $s=0.01$, corresponding to a 13\,sec time constant. It yielded a significant decrease of performance on the test set (10 points of \ac{mAP}). Trials were conducted with several other values ($[0.001, 0.0025, 0.005, 0.01, 0.025, 0.05, 0.1]$) and the maximum performance was reached with $s=0.025$ (reported in Tab.~\ref{tab:mysti_bench}).

These experiments demonstrate that \ac{PCEN} does not always yield performance gains: it depends on the signals to detect and the noises surrounding them. Also, even if choosing a reasonable parameter $s$ tuned for the target signals, performances might be lowered. This is perhaps explained by the difference in compression compared to the trainable log approach \cite{schluter2017deep}. Experiments should thus be conducted on each task before choosing this spectrogram range compression method.

Another insight on \ac{PCEN} behavior was yielded by late experiments with the classification of humpback whale sounds (they are preliminary results not reported in this thesis). \ac{PCEN} was beneficial to detect humpback whale calls, but appeared detrimental to classify then by call type. Put in perspective with the beneficial impact on orca call detection and the opposite effect on the Antarctic mysticete dataset, an hypothesis could be that \ac{PCEN} hinders performance in mutli-class and multi-label datasets.


\subsubsection{Study of performance metrics}

After the selection of the best performing model (ResNet-50 with logarithmic range compression), the \ac{mAP} remains quite low as compared to the \ac{AUC} (0.11 against 0.99 for \ac{Bm} B calls for instance, see Tab.~\ref{tab:scores_AT}). This is due to the high imbalance of the dataset (ratio close to 50 between amounts of positive and negative samples).

\begin{table}[ht]
    \centering \resizebox{\linewidth}{!}{
    \begin{tabular}{c|c|c|c|c|c|c|c}
     & \textbf{\ac{Bm} A} & \textbf{\ac{Bm} B} & \textbf{\ac{Bm} Z} & \textbf{\ac{Bm} D} & \textbf{\ac{Bp} 20\,Hz} & \textbf{\ac{Bp} 20+} & \textbf{\ac{Bp} DS} \\ \hline
    \textbf{Train \ac{AUC}} & 0.99 & 0.99 & 0.99 & 1.00 & 1.00 & 1.00 & 1.00 \\ 
    \textbf{Train \ac{mAP}} & 0.92 & 0.74 & 0.75 & 0.98 & 0.95 & 0.96 & 0.93 \\ 
    \textbf{Test \ac{AUC}} & 0.97 & 0.91 & 0.96 & 0.97 & 1.00 & 1.00 & 0.99 \\     
    \textbf{Test \ac{mAP}} & 0.73 & 0.11 & 0.55 & 0.83 & 0.94 & 0.61 & 0.86 \\ 
    \end{tabular}}
    \caption{Detection performance of the top performing model on the Acoustic Trends dataset (calls from \textit{\acl{Bm}} and \textit{\acl{Bp}}). The model is a Resnet-50 with logarithmic spectrogram range compression trained without SpecAugment.}
    \label{tab:scores_AT}
\end{table}

Indeed, the \ac{mAP} uses the precision, which normalises true positives by positive predictions, whereas the \ac{AUC} uses the specificity, which normalises true negatives by negative samples. For a dataset with mostly silent sections like the Acoustic Trends dataset, the \ac{AUC} will thus be over-optimistic, and the \ac{mAP} will be over-pessimistic. This motivated to experiment on a different, more informative metric: the number of false positives per hour, previously used by \citet{shiu2020deep} on automatic cetacean \ac{PAM} systems.

Figure \ref{fig:recall_fphour} summarises the number of false positives per hour against the recall for each class and data source. It shows how for some calls, the performance is significantly impacted by the data source. This can be explained by a difference in background noise, average \ac{SNR} of the annoted calls, or both. Moreover, the curve for \ac{Bm} B calls in the Kerguelen 2005 data confirms the low score reported in Table \ref{tab:scores_AT}, probably due to the presence of hard samples in the dataset (events that trigger false positive even at high thresholds).

Table \ref{tab:recall_20fphr} summarises these curves once more by reporting the recall at which there are 20 false positives per hour. Indeed, \citet{shiu2020deep} argues that this threshold is the maximum acceptable for quality control processes.

\begin{figure}
    \centering
    \includegraphics[width=\linewidth]{fig/all_recall_vs_fphour.pdf}
    \caption{Number of false positives per hour as a function of recall. Curves are given for each class and each data source. The dotted horizontal line denotes the 20 false positives per hour threshold.}
    \label{fig:recall_fphour}
\end{figure}


\begin{table}
    \centering\resizebox{\linewidth}{!}{
    \begin{tabular}{|c|c|c|c|c|c|c|c|} \hline
        \textbf{Data Source} & \textbf{\ac{Bm} A} & \textbf{\ac{Bm} B} & \textbf{\ac{Bm} Z} & \textbf{\ac{Bm} D} & \textbf{\ac{Bp} 20\,Hz} & \textbf{\ac{Bp} 20+} & \textbf{\ac{Bp} DS} \\ \hline
        Balleny Islands 2015 &  1.00 &  1.00 &  0.98 &  1.00 &  1.00 &  1.00 &  1.00  \\ \hline 
        Elephant Island 2013 &  0.99 &  0.99 &  0.99 &  1.00 &  1.00 &  1.00 &  1.00  \\ \hline 
        Elephant Island 2014 &  0.96 &  0.97 &  0.95 &  0.98 &  0.99 &  0.99 &  0.99  \\ \hline 
        Greenwich 64S 2015 &  0.97 &  0.89 &  0.90 &  0.91 &  &  &  0.98  \\ \hline 
        MaudRise 2014 &  0.98 &  0.82 &  0.75 &  0.98 &  0.92 &  &   \\ \hline 
        Ross Sea 2014 &  1.00 &  &  &  &  &  &   \\ \hline 
        Casey 2014 &  0.98 &  0.92 &  0.96 &  0.99 &  0.95 &  &   \\ \hline 
        Casey 2017 &  1.00 &  1.00 &  1.00 &  1.00 &  1.00 &  1.00 &   \\ \hline 
        Kerguelen 1 2005 &  0.93 &  0.79 &  0.89 &  0.93 &  1.00 &  1.00 &  0.98  \\ \hline 
        Kerguelen 2 2014 &  0.98 &  0.94 &  0.94 &  0.98 &  1.00 &  1.00 &  1.00  \\ \hline 
        Kerguelen 2 2015 &  0.99 &  0.97 &  0.98 &  1.00 &  1.00 &  1.00 &  0.92  \\ \hline 
        \textbf{All} & 0.98 &  0.97 &  0.98 &  0.99 &  1.00 &  1.00 &  1.00  \\ \hline
    \end{tabular}}
    \caption{Recall at 20 FP/hr for each class and data source. Cells with less than 20 samples are not reported.}
    \label{tab:recall_20fphr}
\end{table}

These results emphasis the importance of the choice of performance metric. It needs to account for the datasets' class imbalance, and for the subsequent application needs. In the absence of the latter, the recall at 20 false positives per hour seems a good generic, for its stability facing class imbalance and its high interpretability for production use (other thresholds than 20 can be chosen, depending on project needs).


\section{Resulting detectors performance}

After exploring several \ac{ANN} architectures on datasets of different characteristics (target signals, amount of annotation, diversity of recording systems), this section intends to get an overview of the resulting detection systems.

Best configurations were kept for each task to report performances. When multiple runs were operated (20\,Hz fin whale pulses and sperm whale clicks) the median values are reported. As for the fin whale 20\,Hz pulses, since 3 test folds were studied, the median gathers the 5 runs of the 3 folds.

\begin{table}[ht]
    \centering
    \begin{tabular}{c|c|c|c|c}
         \textbf{Target signal} & \textbf{Archi} & \textbf{\ac{AUC}} & \textbf{\ac{mAP}} & \textbf{Rec(20FP/hr)} \\ \hline
         Fin whale 20\,Hz pulses & 3 depth-wise & 0.99 & 0.84 & 0.94 \\
         Sperm whale clicks & 3 depth-wise & 0.93 & 0.85 & 0.65 \\
         Dolphin whistles & sparrow & 0.98 & 0.86 & 0.61  \\
         Humpback whale calls & sparrow & 0.99 & 0.99 & 0.97 \\
         Orca calls & sparrow & 0.99 & 0.98 & 0.87 \\
        Antarctic mysticetes & ResNet-50 & 0.97 & 0.66 & 0.93 \\
    \end{tabular}
    \caption{Summary of performances for all trained detection systems on their test set (see section \ref{chap:splits}). Reported metrics are, from left to right, area under the receiving operating characteristics curve, area under the precision recall curve, and recall at 20 false positives per hour.}
    \label{tab:recap_perf}
\end{table}

Low complexity architectures such as 3 depth-wise convolution layers suffice in learning to detect low variability signals such as fin 20\,Hz pulses and sperm whale clicks. To increase the precision, several detections can be integrated in larger temporal windows, either with handcrafted rules (discussed with the detection of fin whale songs in section \ref{chap:fin_song}) or with learnt sequential models as proposed by \citet{madhusudhana2021improve}.

The sparrow architecture allows to learn more variable signals, as it was originally designed for bird classification \cite{grill2017two}. It is able to yield satisfactory performances despite a reduced amount of labels.

Then, when larger amounts of annotations are available, the ResNet-50 architecture originally designed for image classification can be used to detect multiple calls (Antarctic mysticetes), sharing the same model embedding before discrimination. Neither sparrow nor ResNet-18 architectures had the capacity to solve this task as the ResNet-50 did.

This thesis' work in annotation and training binary classifiers thus resulted in successful detection systems for 13 different target signals (the Antarctic mysticetes model gathering 7 different signals). The satisfactory performances, especially on test sets that were designed to reflect generalisation capabilities, allow to consider using these trained models in production. Indeed, as we will see in the next chapters, the models served to analyse databases of several thousands recorded hours



\section{Contrastive learning for orca call classification} \label{chap:orca_classif}

A second axis of work conducted on training procedures was applied to a classification task for orca call types. Indeed, call types have been attributed discrete classes, and have served in behavioural and social structure studies \cite{ford1987catalogue, ford1989acoustic}. These studies were done by manually annotating calls, a time consuming task that we try to automate here.

This task implied to use other losses than the \ac{BCE} (the only loss function used so far). Motivated by the lack of annotations, experiments with unsupervised algorithms were first conducted, and are reported in the first part of this section. Then, as annotations were progressively gathered, performances of a semi-supervised learning algorithm are compared to a traditional supervised learning procedure.


\subsection{Trials with deep representation learning algorithms}

Given the large amount of detections of orca calls (section \ref{chap:orca_detec}) and the lack of call type annotation, unsupervised learning approaches have been experimented with. This call type classification task comes down to classifying similar pitch patterns together, which fits with the contrastive learning paradigm. Indeed, learning a representation that ignores small distortions of shapes (time and frequency shifts and dilations) seems appropriate: these distortions are found among instances of a call type. Once such a representation has been learnt, supervised learning can be operated using a small amount of labels to optimise discrimination boundaries (fine tuning).

As mentioned in section \ref{chap:triplet}, numerous algorithms have been proposed in the literature to learn from sparsely annotated datasets using contrastive learning. They mainly differ by the distance metric they use in their embedding space. In search for the right one, papers were in part selected for their top position on the CIFAR-10 with 1000 labels benchmark \cite{cifar10}, as it contains a number of classes and labels that is similar to the dataset at hand. Experiments were thus conducted with SimCLR \cite{chen2020simple}, \acs{UDA} \cite{xie2019unsupervised}, Barlow twins \cite{zbontar2021barlow}), and \acs{IIC}, \cite{sohn2020fixmatch}.

\begin{figure}
    \centering
    \includegraphics[width=.7\linewidth]{fig/boxplot_selfreprlr.png}
    \caption{Distribution of \acp{NMI} between clusters found using k-means on learnt embeddings and labels (5 training initialisation for each algorithm). It needs to be taken into account that annotations were made by filtering some clusters proposed by the \ac{AE} and t-SNE. Here, for a comparison of deep metric learning, \ac{UDA} was trained only with its unsupervised loss.}
    \label{fig:ssl_res}
\end{figure}

In a way reflecting the caveat of modern days deep learning research, a plethora of algorithms were implemented, with limited understanding of their underlying behaviours. Moreover, in addition to their proposed main algorithm, each paper comes with a handful of training tricks which are also responsible for the reported performances. This makes a fair comparison between techniques difficult.

Figure \ref{fig:ssl_res} reports on learnt representation quality for each of the algorithms implemented. The metric used is the \ac{NMI} between clustering of the embeddings and their associated label. Barlow twins and \ac{UDA}, for some initialisation, show a slightly higher \ac{NMI} than the representation used to annotate (t-SNE projected \ac{AE} embeddings).

Despite the invested efforts, none of the implemented algorithms (SimCLR, \acs{UDA}, Barlow twins, and \acs{IIC}) showed a relevant gain in performance after fine tuning for the classification task (as compared to a random initialisation of weights).


\subsection{FixMatch versus supervised learning}
\textit{This work has been subject to a workshop intervention \cite{bestvihar}.}

With the progressive collection of labels, semi-supervised learning approaches became more and more relevant. Again, several algorithms were experimented with: Meta Pseudo Labels \cite{pham2021meta}, \ac{UDA} \cite{xie2019unsupervised}, mixMatch \cite{berthelot2019mixmatch} and fixMatch \cite{sohn2020fixmatch}. However, selected for its good loss convergence, reasonable performances, and very few training tricks needed, fixMatch was retained for further comparison with the regular supervised approach.

The fixMatch algorithm combines a supervised loss, pseudo labelling, and consistency training in one framework. Pseudo labelling consists in applying a supervised loss on samples without annotation by using the highest prediction of a model (the  `pseudo label'). This allows to make use of unlabelled data, especially on easy samples (pseudo labels can be retained only if the confidence is above a predefined threshold, see Fig.~\ref{fig:fixMatch}). This can be beneficial because it broadens the diversity of data seen by the model without demanding more annotation.

On the other hand, consistency training is the concept of learning a projection that ignores (is `consistent' against) variations in the data. It is very close to the concept of contrastive learning aforementioned. FixMatch makes use of it by applying different levels of data augmentation to its inputs.

Fig.~\ref{fig:fixMatch} shows how data augmentation and pseudo-labelling were combined for the orca call classification task, following the fixMatch approach. $H(p,q)$ denotes the cross-entropy between the pseudo label and the prediction after strong augmentation. It represents the unsupervised loss that will be added to the regular supervised cross-entropy loss before the backward propagation.

\begin{figure}
    \centering
    \includegraphics[width=\linewidth]{fig/fixMatch.png}
    \caption{The fixMatch algorithm \cite{sohn2020fixmatch}, a combination of pseudo-labelling and consistency training. The figure was taken from the original paper, and adapted for the orca call classification task.}
    \label{fig:fixMatch}
\end{figure}

The main difference with this thesis' implementation is the chosen data augmentation policy. Here, SpecAugment \cite{park2019specaugment} was used (instead of RandAugment \cite{cubuk2020randaugment}). It was applied on \ac{PCEN} normalised Mel-spectrograms of 2 seconds excerpts, with 128 frequency bins and 128 frequency bins and 262 temporal bins ($f_s=22,050$, $NFFT=1024$, $hop=165$). Strong augmentations allowed until 20\% of dilation (time and frequency wise), dropping bands of maximum 20 frequency and temporal bins, and gaussian blurring, whereas weak augmentations capped dilations to 5\%, and dropped bands up to 5 bins, without gaussian blurring.

As for the remaining hyper-parameters, (learning rate, cosine scheduling, batch sizes, pseudo-labelling threshold, and loss weighting) they were left as proposed by the paper \cite{sohn2020fixmatch}.

\begin{table}[ht]
    \centering
    \begin{tabular}{c|cc|cc}
    & \multicolumn{2}{c|}{\textbf{90/10 train/test split}} & \multicolumn{2}{c}{\textbf{50/50 train/test split}} \\ \hline
          & \textbf{F1 score} & \textbf{Accuracy} & \textbf{F1 score} & \textbf{Accuracy} \\ \hline
         Supervised & 0.95 & 0.95 & 0.94 & 0.94 \\ 
         FixMatch & 0.92 & 0.94 & 0.84 & 0.89\\ \hline
    \end{tabular}
    \caption{Comparison of performances for regular supervised learning and semi-supervised learning approaches. Results are given for a regular train/test split, and with a reduced training set (200 samples per class in average).}
    \label{tab:orca_clf_perf}
\end{table}

The resulting performances of semi-supervised and supervised training are compared in Tab.~\ref{tab:orca_clf_perf}, with the accuracy computed across all samples and the F1-score being computed for each class independently before averaging. Both were trained with a ResNet-50 and cosine learning rate scheduling.

The results demonstrate a counter productive effect of the unsupervised loss, even when reducing the number of annotations by half (approximately 200 samples per class in average). This might be explained by a too strong augmentation policy which might mask out complete calls in some cases (some calls lie in less than 20 frequency bins for instance). Further work should focus on researching augmentations that are more adapted to the variations found among calls of the same types but without risking to change 
\chapter{Application to species conservation}
\label{chap:conservation}
\minitoc


\section{Context and objective}

Given previously trained detection and classification systems, this section describes how they can be put to production and serve species conservation purposes. Focusing on the sperm whales and fin whales of the Mediterranean sea, a first axis of conservation is the reduction of ship strikes, a significant cause of death for these species evolving in the Pelagos marine mammal sanctuary \cite{panigada2006mediterranean}. Then, the detection mechanisms is run upon the Bombyx long term survey. This yields insights on sperm whale behaviour in relation to anthropic pressure, helping to implement relevant conservation measures in the long term.


\section{Alert system for collision risk mitigation}
\label{chap:GIAS}

\subsection{Context and objective}

\begin{figure}[!htb]
   \begin{minipage}{0.8\textwidth}
     \centering
     \includegraphics[width=\linewidth]{fig/bombyx4.png}
   \end{minipage}\hfill
   \begin{minipage}{0.15\textwidth}
     \centering
     \includegraphics[width=\linewidth]{fig/bombyx1.png}
   \end{minipage}\hfill
   \caption{Technical plans of the Bombyx 2 system, taken from OSEAN SAS manufacturing report. (left) Mooring system. (right) Pentaphonic acoustic recorder and floatability variation system (total height of 3 meters).}
\end{figure}

As part of the GIAS project aiming at reducing navigation risks in the Mediterranean sea, the Bombyx 2 buoy was designed, in a collaboration between DYNI and OSEAN SAS. Preliminary work on this project was subject to a conference publication \cite{best2020stereo}

This buoy is equipped with 5 hydrophones, a floatability variation system, and embedded algorithms for the detection of sperm whale clicks and 20\,Hz fin whale pulses. To mitigate surface noise and exposure to strong weather conditions, the buoy parks at 25 meters depth to record and acoustically detect its target species (sperm whales and fin whales). In the event of a detection, the buoy reaches the surface to transmit the alert with supporting data via the mobile network. The alerts then allow ferries of the zone to make decisions to mitigate their risk of collision with nearby whales (reducing speed or changing route for instance).


\subsection{CNN deployment to an embedded MCU}

Section \ref{chap:lightweight} introduced low complexity \acp{CNN}, especially designed to answer the needs of this alert system. These models, after being trained on \acp{GPU} using the Pytorch package \cite{NEURIPS2019_9015}, were implemented on the embedded system, namely the Microchip PIC32 \ac{MCU} (integrated on the High-Blue sound card \cite{barchasz2020novel}).

This demanded to build a custom interface to export and load architectures and weights via text files. The exports are done in Python, and imported in C (required programming language for the \ac{MCU}). Design choices were made for the C implementation, for a compromise between flexibility and reduced development effort:
\begin{itemize} \setlength{\itemsep}{1pt}
    \item The model input consists in a Mel-spectrogram,
    \item Signal length, sampling frequency, window length, hop size, number of Mel-bands, and Mel-frequency boundaries are parametric,
    \item The architecture consists of successive depth-wise separable convolution layers intertwined with batch normalisation and leaky \ac{ReLU},
    \item The number of layers, and the number of features, kernel sizes and strides for each layer are parametric,
    \item The last layer is pooled by maximum to yield a global prediction of the signal.
\end{itemize}


\subsection{Computation times}

Specifications of the input parameters and processing time for the two target signals are given in Tab.~\ref{tab:pic_implem}.
%
\begin{table}[ht]
    \centering
    \begin{tabular}{l|c|c}
    Target signal & Sperm whale clicks & 20\,Hz fin whale pulse \\ \hline
    Signal length (sec) & 10 & 60 \\
    Sampling frequency (Hz) & 64,000 & 4,000 \\
    \ac{FFT} window length & 512 & 4096 \\
    \ac{FFT} hop size & 256 & 256 \\
    Mel bands & 64 & 64 \\
    Mel start (Hz) & 2,000 & 0 \\
    Mel end (Hz) & 25,000 & 100 \\
    Signal loading (sec) & 1 & 5 \\
    Spectrogram computation (sec) & 12 & 26 \\
    \ac{CNN} inference (sec) & 4 & 4 \\
    \end{tabular}
    \caption{Specifications and corresponding processing times on the PIC32 \ac{MCU}, for the detection mechanisms of sperm whale clicks and 20\,Hz fin whale pulses.}
    \label{tab:pic_implem}
\end{table}
%
The longest step is by far the spectrogram computation compared to \ac{CNN} inference. This comforts the choice of the Fourier transform which offers a fast \ac{FFT} implementation, rather than others such as the wavelet transforms.

% \subsection{Power budgeting}
% duty cycles
% \cite{rand2022effects, riera2013effects}


\subsection{Detection report}

In the event of detections triggered by the \acp{CNN}, the buoy is ordered to lift towards the surface to transmit a report supporting the alert. It includes multi-channel chunks of signals (cut surrounding detection peaks), prediction sequences for the two species, and buoy orientation (compass, and magnetometer). These extracts of signals allow experts to confirm the veracity of the alert and to take decisions accordingly. Moreover, the reported extracts being multi-channel (5 hydrophones), triangulation via cross-correlation is possible, increasing the spatial precision of the alert.

The prediction sequences can serve a quick discrimination between false positives, by examining distribution among successive files (Fig.~\ref{fig:compare_cacha}).

\begin{figure}
    \centering
    \includegraphics[width=\linewidth]{fig/falsedetec_vs_passage.pdf}
    \caption{Comparison of the distribution of model predictions for a day with a sperm whale (July 7th 2015) and a day with false detections (September 8th 2017).}
    \label{fig:compare_cacha}
\end{figure}

% TODO conclude : insight

% \section{Interpretation of model predictions}
% Check pred distribution
% Dolphin / humpbacks CARIMAM
% Carimap


\section{Long term presence monitoring}

In addition to its production use in the context of ship collision mitigation, the sperm whale click detection \ac{CNN} has been forwarded on the whole Bombyx dataset (3,532 recorded hours from May 2015 to December 2018) for a long term study of sperm whale presence. This work resulted in a journal publication \cite{bombyx}, from which some of the results are reported here.


\subsection{Sperm whale acoustic presence} \label{chap:cacha_presence}

A first analysis focused solely on reporting the presence of sperm whales through the recorder years. Files (1min long) with more than 40 \ac{CNN} predictions above 0.95 were manually verified using the interface described in section \ref{chap:bombyx_annot}. Like so, automatic detections were validated and number of individuals were estimated (inferred from simultaneous click trains and \ac{TDOA} tracks). This process yielded 57 new sperm whales passages (missed during the annotation procedure described in section \ref{chap:bombyx_annot}), and 25 false positives (including 15 triggered by sound card malfunctions). The notion of passage was used to account for sperm whale presence, considering that clicks belong to the same passage if separated by less than 1h.

In total, 226 sperm whale passages have been recovered, with a total of 347 individuals. Fig.~\ref{fig:res_nbr_indiv_calendar} presents the number of detected individuals each day during the 4 years of recording. Sperm whales were found all year round, with no statistically significant seasonal pattern. The number of animals per passage varied from 1 to 9 individuals, with a mean duration of 4 hours.

To evaluate dial patterns, the probability of presence was computed for each hour of the day. Grouping probabilities into four periods (Night, Morning, Afternoon, and Evening) demonstrates a statistically significant differences among periods of the day : sperm whales are more present during morning or afternoons than in the evening (Fig.~\ref{fig:res_nbr_indiv_calendar}, Kruskal-Wallis test : p-value < 0.01).

\begin{figure}[ht]
\centering
\includegraphics[width=\linewidth]{fig/bombyx_results_calendar.pdf}
\caption{ Left (a): The Number of detected sperm whales per day during the 4 years of recordings (white region: \textit{no d = no data}). Right (b): Distribution of hourly probabilities of presence for each period of the day.}
\label{fig:res_nbr_indiv_calendar}
\end{figure}


\subsection{Presence in relation to anthropogenic noise pressure}

To assess the performance of the detection system as well as to measure the impact of noise on the presence of sperm whales, the amplitudes of different octave bands were computed and analysed. The distribution of the background noise (octave 800\,Hz) through the day is shown in Fig.~\ref{fig:db_cach}. All octaves' dial distributions have the same shape as the 12,800\,Hz octave, with the energy peaking around 4am and 9pm.

\begin{figure}[ht]
    \centering
    \includegraphics[width=\linewidth]{fig/bombyx_results_noise.pdf}
    \caption{(left) Distributions of 12,800\,Hz amplitudes during and outside sperm whale passages. (right) Superposition of dial pattern of amplitudes for the octave 12,800\,Hz and probability of presence of sperm whales.}
\label{fig:db_cach}
\end{figure}

Ferries cross the study area daily, connecting Toulon or Marseille to Corsica, with scheduled times between 3am - 6am and from 8pm - 9pm. The closest ferry route is approximately 3km away from the antenna. For all octaves, dB amplitudes are significantly higher during ferry schedules (Mann–Whitney test, \textit{p-value} $<$ 0.05), with an average gain of approximately 3\,dB.

Moreover, as Fig.~\ref{fig:db_cach} illustrates, the data shows a significantly lower noise during the sperm whales' presence (Mann-Whitney U=14.44, sample size=300, \textit{p-value} < 0.01) for all octaves except 6,400\,Hz and 12,800\,Hz. This is further demonstrated in Fig.~\ref{fig:db_cach}, where, during 4 AM and 9 PM (noise peaks), the presence of sperm whales is lowest. This last figure also shows that the reduced sperm whale presence is not due to an increased background noise, since sperm whale probability drops before the background noise rises.


\section{Conclusion}

These studies are a first demonstration of the versatility of the detection systems designed through this thesis. Indeed, they can be applied to a real-time alert system to mitigate collision risks, but also in long-term surveys, revealing presence patterns that are crucial in the implementation of relevant conservation measures.
\chapter{Application to communication modelling}

\minitoc

\section{Context and objective}

The previous chapter demonstrated how robust detection systems can be used for species conservation purposes. A second axis of use can also be the study of animal communication systems. In the past, \ac{PAM} has put forward several examples of song and social communication systems in cetaceans. This allows comparative studies to reflect on the natural evolution of music and language \cite{fitch2006biology}. For that same purpose, robust detection and classification mechanisms are able to analyse large datasets and yield new insights. This is demonstrated in the following chapter with the long term evolution of the Mediterranean fin whale song and communication patterns of the \ac{NRKW}.


\section{Fin whale song structure and temporal trends}

\subsection{Context and objective}

In parallel to the sperm whale study with the Bombyx dataset, a similar one was conducted on fin whales of the Ligurian sea, again making use of the detection system designed for the GIAS buoy. The trained \ac{CNN} described in section \ref{chap:lightweight} was run over three available datasets : Boussole, Bombyx, and KM3Net (see section \ref{chap:data_Toulon}). This time, instead of presence monitoring, the study focused on fin whale song patterns, yet poorly documented in the Mediterranean sea. It is also subject to a journal publication \cite{finsong}, which results are reported here.

As other cetacean species, fin whales show geographical acoustic differentiation \cite{helble2020fin, morano2012seasonal, castellote2012fin}, hypothesised to be cultural in some cases \cite{weirathmueller2017spatial}. The divergence of mysticetes songs in different populations is presumably a result of drifts emerging from the conformity and creativity constraints of song production \cite{payne2000progressively}. Moreover, the character displacement theory with songs serving as a discrimination marker for allopatric populations has been hypothised for fin whales of the Northern Atlantic \cite{delarue2009geographic}. As for the Mediterranean population, it has been shown to be resident and genetically dissociated from the North Atlantic population \cite{berube1998population}, and their songs (especially the \ac{IPI}) were shown to allow for their identification \cite{castellote2012fin, pereira2020fin}. The Mediterranean fin whales do not follow strict migration patterns or reproduction periods unlike their oceanic conspecifics \cite{Notarbartolo}, so their song can be heard all year round.

The base unit of the songs, the 20\,Hz pulse, is shared by all fin whales. These pulses occur in sequences that typically last several hours \cite{Watkins_Tyack_Moore_Bird_1987}, with highly regular pulse intervals between 10 and 40 seconds. The main differentiation of songs lies in the \ac{IPI} and pulse spectra \cite{Thompson, hatch2004acoustic}. Alike fin whales of the Pacific \cite{weirathmueller2017spatial, helble2020fin}, Mediterranean 20\,Hz pulses fall into 2 distinct types, one with a slightly higher frequency content than the other \cite{ClarkBorsani2002, sciacca2015annual} (see Fig. \ref{fig:spectro}). These two categories are sometimes labelled 20\,Hz pulse and back-beat, they will be referred to as type A and B for short, with A being the higher pitched pulse. Fin whales of the Pacific and Atlantic often exhibit sequences that alternate between A and B pulses. These are called doublet patterns, as opposed to singlets where only one of the pulse types occur. In doublets there is a strong relationship between \ac{IPI} and pulse type: two different \acp{IPI} are found, one from A to B, and another one from B to A \cite{Oleson, constaratas2021fin, Furumaki_Tsujii_Mitani_2021, morano2012seasonal, helble2020fin}. On the other hand, singlets also follow their own stereotypical \ac{IPI}.

Mediterranean fin whale songs present more diversity in the consecution of pulse types than simple singlets and doublets (Fig. \ref{fig:spectro}). Nonetheless, two studies present stereotypical \acp{IPI}. Based on recordings from 1999, \citet{ClarkBorsani2002} observe a link between pulse type and \ac{IPI} in the Mediterranean sea for two bouts (about 100 pulses). About ten years later, \citet{castellote2012fin} observe a common \ac{IPI} around 14.9\,sec for that same population, but do not mention its relationship with pulse types.

Besides geographical variations, fin whale song structures also exhibit temporal variations, such as seasonal \ac{IPI} increases \cite{Oleson, morano2012seasonal}, and inter-annual variations of \ac{IPI} and peak frequency \cite{weirathmueller2017spatial}. \ac{PAM} stations combined with automated analysis (template matching approach) have played a key role in revealing these long-term trends.

Until now, no large scale analysis has been conducted on Mediterranean fin whale songs that could reveal the long-term evolution of their vocal behaviour, which motivates the following study.


\subsection{Method}

\subsubsection{Model inference}

While the model was trained to detect pulse presence in 5-second segments, the convolutional stack is designed to maintain the temporal resolution of the predictions throughout the network. Discarding the max pooling layer at the end of the CNN, pulse times were retained as the highest predictions above a given threshold within sliding 4 second windows. These timings are approximate up to the size of the receptive field of the network (0.8 seconds).

Thresholds were set at the balance point of the \ac{ROC} curves (equal sensitivity and specificity). This setting lead to sensitivities and specificities of 0.96 and 0.97 for the Bombyx and Boussole data respectively. For the KM3Net data, since the \ac{ROC} curve is unknown (no annotation are available), a threshold of 0.12 was chosen so that there is approximately the same proportion of detections as in Bombyx and Boussole ($\approx$ 0.5\%) .

Tab. \ref{tab:rorqual_recap} summarises the resulting detections, along with a calendar Fig. \ref{fig:calendar}. Following \citet{Watkins_Tyack_Moore_Bird_1987}, pulses at a distance of less than 45 seconds were considered as being part of the same sequence, and sequences less than 2 hours apart were considered as being part of the same bout.

\begin{table}[ht]
\centering\resizebox{\linewidth}{!}{
\begin{tabular}{|l|c|c|c|c|}
\hline
 \textbf{Data source} & \textbf{Boussole} \cite{parsuivi} & \textbf{Bombyx} \cite{vamos2017} & \textbf{KM3Net} \cite{aiello2021km3net} & \textbf{Total} \\ \hline
\textbf{Location} & South of Sanremo & Port-Cros Island & Cap Sicié & \\ \hline
\textbf{Recording year} & 2008-2009 & 2015-2018 & 2020-2021 & \\ \hline
\textbf{Recorded time (hours)} & 1,860 & 3,291 & 1,124 & 6,275 \\ \hline
\textbf{Detection threshold} &  0.15 & 0.68 & 0.12 & \\ \hline
\textbf{Pre-filtering detections} & 52,863 & 83,583 & 9,684 & 146,130 \\ \hline
\textbf{Detected pulses} & 1,647 & 2,827 & 657 & 5,131 \\ \hline
\textbf{Detected A pulses} & 1,411 & 2,554 & 322 & 4,287 \\ \hline
\textbf{Detected B pulses} & 236 & 273 & 335 & 844 \\ \hline
\textbf{Detected sequences} & 246 & 615 & 58 & 919 \\ \hline
\textbf{Detected bouts} & 51 & 214 & 11 & 276 \\ \hline
\end{tabular}}
\caption{\label{tab:rorqual_recap}Summary of recording characteristics and automatic detections for each source of data.}
\end{table}

\begin{figure}
    \centering
    \includegraphics[width=.75\linewidth]{fig/rorqual_calendar.pdf}
    \caption{Number of detected sequences for each day with recordings, normalized by the amount of recorded hours. Grey cells denote days with recordings but no detection.}
    \label{fig:calendar}
\end{figure}


\subsubsection{Spectro-temporal pulse analysis}

Following the detection process, a signal processing analysis was conducted to precisely describe each pulse (exact time position, center frequency, bandwidth and \ac{SNR}). This yields the necessary data to search for song patterns, as shown in Figure \ref{fig:spectro}.

\begin{figure}[ht]
 \centering
 \includegraphics[width=1\linewidth]{fig/spectrogram.pdf}
 \caption{Spectrogram of a fin whale pulse sequence recorded by the Bombyx buoy in October 2018 ($f_s=250$, $NFFT=1024$, $hop=8$, $padding=75\%$). Dots show the center frequencies of the detected pulses, with white dashed lines showing \acp{IPI}. The grey dashed line denotes the discrimination threshold between type A and B pulses, at 19.88\,Hz.}\label{fig:spectro}
\end{figure}

For this analysis, an 8\,sec window surrounding the prediction peak is selected (\(T=[0,8]\)), band-pass filtered (Butterworth of order 3 between 10\,Hz and 30\,Hz), and resampled at 250\,Hz. The \ac{STFT} is then applied to the resulting signal (Hann window of 1024 including 75\% of zero padding and 97\% overlap) resulting in spectral and temporal resolutions of 0.24\,Hz and 0.03\,sec respectively.

From this spectrogram, the precise time position of the pulse \(\hat{t}\) is first estimated by selecting the column of the maximum value in the 18-22\,Hz frequency band (Eq. \ref{eq:time}). This value will be kept for \ac{IPI} measurements. 

\begin{equation}
\hat{t} = \argmax_{t \in T}\left( \max_{f \in [18,22]}\left( \mathbf{S}(f, t)\right)\right)
\label{eq:time}
\end{equation}

To measure the spectral envelope of the pulse, a 1.2\,sec window around $\hat{t}$ is max-pooled time wise. Background components are withdrawn (to focus on the pulse spectra only) by subtracting an estimate of the background spectrum: the median of each frequency bin within the window $T$ (Eq. \ref{eq:spec}). Doing so, effects such as the impact of \ac{SNR} on peak frequency and bandwidth (observed by \citet{helble2020fin}) are mitigated.

\begin{equation}
E(f) = \max_{t \in [\hat{t}-0.6, \hat{t}+0.6]} \left(\mathbf{S}(f, t)\ \right) - \   \underset{t \in T}{\mathrm{median}}\ \left( \mathbf{S}(f, t)\right)
\label{eq:spec}
\end{equation}

The resulting pulse envelope is used to compute the left and right boundaries of the pulse spectrum, with $\frac{\max E(f)}{4}$ as a threshold (equivalent to -6\,dB). Left and right intersection frequencies are linearly interpolated to increase the precision of the estimate. This process yields the 6\,dB bandwidth (width between the boundaries), and the center frequency (mid-point between the boundaries) of the analysed pulse.

\begin{figure}[ht]
     \centering
     \includegraphics[width=.75\linewidth]{fig/centroid.pdf}
     \caption{Histogram of the center frequencies of the detected pulses. Black lines denote the fitted \ac{GMM}.}\label{fig:centroid}
\end{figure}
 

For later filtering by pulse quality, the \ac{SNR} is also computed following Eq. \ref{eq:bckgnd} (pulse energy as the maximum of its envelope and background energy as the median of the spectrogram surrounding the pulse).

\begin{equation}
\begin{split}
E_{Background} & = \underset{T \setminus [\hat{t}-1,\hat{t}+3]}{\underset{\ f \in [15, 25]}{\mathrm{median}}}\ \mathbf{S}(f, t) \\
E_{Pulse} & = \max_{f}E(f) \\
SNR & = 10\log_{10}\left(\frac{E_{Pulse}}{E_{Background}}\right)
\label{eq:bckgnd}
\end{split}
\end{equation}

The pulse spectral characteristics of mysticetes are often described using the frequency of maximum energy (peak frequency) or the spectrum weighted mean (centroid frequency) \cite{weirathmueller2017spatial, malige2020inter}. Here, the center frequency was chosen, as it appeared to be better suited for the discrimination between the two pulse types. In fact, when modeling the distribution of peak frequencies using a Gaussian mixture model, the two components (emerging from the two types of pulses) overlap more than when using center frequencies (the Kullback-Leibler divergence between the Gaussian components in center frequency is significantly higher than that of peak frequencies, with 113 nats and 30 nats respectively).


\subsubsection{Pre-analysis filtering}

To filter out false positives, only pulses with a bandwidth below 10Hz and a center frequency within \([18.5,22.5]\) were retained. Besides, only sequences with a mean \ac{SNR} of at least 8\,dB, and with at least 3 pulses were kept for the following analysis. Sequences containing \acp{IPI} below 10sec or above 45sec were discarded as well. The resulting number of registered pulses and sequences are shown in a calendar Fig. \ref{fig:calendar} and in Tab. \ref{tab:rorqual_recap}.

To classify between A and B types, a two component \ac{GMM} was fitted on the center frequency data (Fig. \ref{fig:centroid}) using the \ac{EM} algorithm. This lead to a threshold of 19.88 Hz to discriminate between the two types. Even though the center frequency is found to evolve over time, the change is sufficiently small to not interfere with the categorisation (see Fig. \ref{fig:centScat}).


\subsection{Results}\label{chap:rorq_results}

\subsubsection{Stereotypical \ac{IPI}}

The time between a pair of consecutive pulses in a sequence (the \ac{IPI}) appears to be strongly determined by their type (see Fig. \ref{fig:histIPI}). The typical interval for an `AB' bi-gram is 2sec longer than that of `AA' or `BA'. On the other hand, the 'BB' pairs (less frequent but still commonly found) are 11sec longer on average, but present larger variability than the others.

\begin{figure}[ht]
     \centering
     \includegraphics[width=.75\linewidth]{fig/IPI_hist.pdf}
     \caption{Histogram of the \ac{IPI} for each type sequence (bi-gram).}\label{fig:histIPI}
\end{figure}

Figure \ref{fig:scattIPI} shows how these intervals have changed over the course of two decades, following an approach similar to \citet{weirathmueller2017spatial}. For each month and pulse type pair, points denote the most frequent \ac{IPI} (quantized with a resolution of 0.1sec). For months containing more than 100 bi-gram occurrences, the most frequent \ac{IPI} was retained only if representing at least 5\% of it. Points measured in previous studies of the same population were also added : in 1999 by \citet{ClarkBorsani2002} (the only study to our knowledge that references \ac{IPI} depending on type sequence in the Mediterranean sea), and in 2008 by \citet{castellote2012fin} (assuming it describes the most common pair `AA', as it was not specified). The `BB' sequence did not provide enough occurrences for the statistical tests to be relevant. 

\begin{figure}[ht]
     \centering
     \includegraphics[width=.75\linewidth]{fig/IPI_scatter.pdf}
     \caption{Scatter plot of the most frequent \ac{IPI} per month for each type sequence. Fitted linear models are shown as grey dashed lines. Points extracted from \citet{ClarkBorsani2002} and \citet{castellote2012fin} appear as crosses.}\label{fig:scattIPI}
\end{figure}

For sequences `AA', `AB', and `BA', fitted linear models are plotted (their coefficients of determination are 0.83, 0.89, and 0.91 respectively). The p-values for the null-hypothesis that the slopes are not significantly different from 0 are all inferior to 0.01. The estimated slopes for the `AA', `AB', and `BA' bi-grams are 84, 83, and 88 respectively (in milliseconds/year). 
 
 
\subsubsection{Center frequency}\label{seq:freq}

In a similar fashion, temporal trends of pulses' spectral characteristics were analysed. This revealed an intra-annual decrease in pulse center frequency between the months of August and May (Fig. \ref{fig:centScat}). On the other hand, no inter-annual shift was observed (Pearson analysis yields a correlation coefficient of -0.06 between pulse absolute dates and their center frequency).

\begin{figure}[ht]
    \centering
    \includegraphics[width=.75\linewidth]{fig/centroid_hist2d.pdf}
     \caption{Bi-histogram of the center frequencies against months of the year. The horizontal line shows the separation between type A and type B pulses. The fitted linear model is shown as a black dashed line.}\label{fig:centScat}
\end{figure}

For this statistical analysis, center frequencies were quantized to a resolution of 0.1Hz and grouped by months. Center frequencies with the most occurrences were kept, if among groups (months) of at least 50 pulses. Fitting a linear model on the retained points yields a coefficient of determination of 0.73, with an estimated slope of -0.08 Hz/month) (for the null-hypothesis that the slope is not significantly different from 0, the p-value is below 0.01).

For comparison with other previous studies, the same analysis was ran using peak and centroid frequencies. The slope of the observed intra-annual trends are similar for all metrics (-0.09Hz/month, -0.08Hz/month, and -0.11Hz/month for peak, center, and centroid frequencies respectively) and p-values for the null-hypothesis that the slope is not different from 0 are all below 0.01.


\subsubsection{Correlation between center frequency and \ac{IPI}}

With the observation of synchronous inter-annual shifts of both \ac{IPI} and center frequencies in Pacific fin whales, the hypothesis of a link between the two arose. \citet{weirathmueller2017spatial} states that the augmentation of the \ac{IPI} through the years could be explained by the simultaneous decrease in pulse peak frequencies (lower frequency pulses presumably requiring a bigger effort to produce, a bigger gap between them could be needed). The observed stereotypical \acp{IPI} of Mediterranean fin whales also support this idea (sequences towards A pulses show lower \acp{IPI}). This hypothesis was thus further tested by analysing the correlation between \ac{IPI} and center frequency (for pulses with \acp{IPI} between 14 and 20 seconds).

To dissociate this analysis from the link between pulse types and \ac{IPI}, 3 component Gaussian mixture model was fitted on the bi-dimensional representation of pulses (center frequency versus time until the next pulse). This enabled to group the different pulse bi-grams (`AA', `AB', and `BA'), and conduct a correlation analysis on each group independently. Figure \ref{fig:centroidIPI} shows the scatter plot of the pulses with their assignation to each mixture component. For each of the latter, the Pearson correlation coefficient was computed, yielding -0.37, -0.22, and -0.35 for `BA', `AB', and `AA' respectively (all p-values are below 0.01).

\begin{figure}[ht]
    \centering
     \includegraphics[width=.75\linewidth]{fig/centroid_vs_IPI.pdf}
     \caption{Scatter plot of pulses center frequency against the time until the next pulse (\ac{IPI}). Colors denote the \ac{GMM} assignations, whose means are marked with crosses.}\label{fig:centroidIPI}
\end{figure}


\subsection{Discussion}

\subsubsection{Mediterranean sea stereotypical \acp{IPI}}

The present study led to the confirmation of the local stereotypical \acp{IPI} being determined by pulse bi-grams. These results were previously shown on relatively a small corpora of around 100 pulses \cite{ClarkBorsani2002}, they are hereby confirmed with a corpus larger by an order of magnitude, and over a span of 10 years.

Moreover, two temporal trends were observed. They are put in relation to other fin whale song studies in Tab. \ref{tab:summTrends} and discussed in the following sections.

\begin{table}[ht]
\centering\resizebox{\linewidth}{!}{
    \begin{tabular}{lc|cc|cc}
    & & \multicolumn{2}{c|}{\textbf{Inter-annual}} & \multicolumn{2}{c}{\textbf{Intra-annual}}  \\
     \textbf{Study} & \textbf{Location} & \textbf{Frequency} & \textbf{IPI} & \textbf{Frequency} & \textbf{IPI} \\ \hline
    \citet{weirathmueller2017spatial} & N.E. Pacific & -0.17\,Hz/yr& 0.5-0.9\,sec/yr & - & - \\
    \citet{Oleson} & N. Pacific & - & - & - & +7.5\,sec \\
    \citet{leroy2018long} & Indian & -0.21\,Hz/yr & - & $\sim$ -0.1\,Hz/mth & - \\
    \citet{helble2020fin} & N. Pacific & - & 0.6-1.3\,sec/yr & - & - \\ 
    \citet{morano2012seasonal} & N.W. Atlantic & - & * 0.5\,sec/yr & - & +5.5\,sec \\
    \citet{Watkins_Tyack_Moore_Bird_1987} & N.W. Atlantic & - & - & - & +6\,sec \\
    \citet{vsirovic2017fin} & Gulf of California & - & $\sim$ 1\,sec/yr & - & $\sim$ +8\,sec  \\
    \citet{furumaki2021fin} & Chukchi sea & - & $\sim$ 0.5\,sec/yr & - & $\sim$ +1\,sec \\
    \citet{wood2022characterization} & W. Antarctic & -0.2\,Hz/yr & 0.1\,sec/yr & - & - \\ \hline
    \textbf{self} & W. Mediterranean & - & 0.1\,sec/yr & -0.1\,Hz/mth & - \\
    \end{tabular}}
    \caption{Summary of song pattern trend studies. For intra-annual \ac{IPI} shifts, since trends are not linear, we report the difference between low \ac{IPI} season and high \ac{IPI} season (summer vs winter). The inter-annual \ac{IPI} shift for \citet{morano2012seasonal} (see `*') is reported between two consecutive years only.}
    \label{tab:summTrends}
\end{table}


\subsubsection{\ac{IPI} trends} % IPI inter year

Mediterranean fin whale stereotypical \acp{IPI} are shown to evolve over the years, following a linear growth of approximately 0.1 sec/year over the past 20 years. Such trends had already been observed in the songs of North-East Pacific \cite{weirathmueller2017spatial} and Central-North Pacific \cite{helble2020fin} fin whales.

Inter-annual shifts in \ac{IPI} are rather recent and poorly documented. \citet{weirathmueller2017spatial} state that the increasing \ac{IPI} might be linked to the downward frequency shift, lower frequency pulses potentially being more demanding in energy. As for the present data, a low correlation coefficient was measured between the two variables, and no evidence of any inter-annual center frequency decrease was found. These observations thus go against this hypothesis, but more data is required to draw firm conclusions.

As for the \ac{IPI} shift slopes, it seems plausible that the differences between Pacific and Mediterranean populations arise culturally. Whether they are originally caused by the same factors or not, the singing patterns drift independently, with song conformity only taking place within a given population. If environmental or physiological factors alone were responsible for such patterns, they would have to be present both in the Pacific and in the Mediterranean sea, but operating at different rates. The hypothesis of the post-whaling population recovery (increasing density and animal sizes) explaining those trends suits the latter conditions, as recovery rates could differ between Mediterranean and Pacific waters. On the other hand, cultural features such as contact rate between individuals could explain slope differences as well, regardless of the root cause of the shift.

% IPI intra year
On the other hand, studies of Atlantic and Pacific fin whales \cite{morano2012seasonal,Watkins_Tyack_Moore_Bird_1987, Oleson, weirathmueller2017spatial} point to \ac{IPI} increases during winter, before dropping back to autumn values. These trends are hypothesised to be directly linked to the reproductive season \cite{Oleson} (due to hormonal activity or progressive dilution of the competition for instance). No such trend was observed in the present data, but the irregular data sampling through seasons might create an observational bias in that sense.


\subsubsection{Pulse frequency trends} % inter annual

Inter-annual shifts in vocalization frequencies were already documented in blue whales \cite{mcdonald2009worldwide, malige2020inter, rice2022update}, and bowhead whales \cite{thode2017decadal}. Fin whales also showed similar trends in the Pacific \cite{weirathmueller2017spatial} (for 20\,Hz pulses, -0.17\,Hz/year) and in the Indian Ocean \cite{leroy2018long} (for 99Hz pulses, -0.21 Hz/year). Numerous hypotheses have been formulated for the cause of this phenomenon, such as the increase in population density or body sizes (following the cessation of commercial whaling), the increase in calling depth \cite{gavrilov2012steady}, the augmentation of noise from melting icebergs \cite{leroy2018long}, or the acidification of the oceans affecting sound propagation \cite{hester2008unanticipated} (among others).

% centroid intra year
No inter-annual frequency shift was found in the analysed data. Mediterranean fin whales could thus be an exception to this widespread trend. Nonetheless, an intra-annual decrease in center frequencies was observed (-0.08 Hz/month). Such phenomenon was previously observed in large mysticetes of the Indian Ocean including fin whales \cite{leroy2018long}. The latter study hypothesised pulse frequencies to follow seasonal ambient noise level variations (notably due to melting ice). Such phenomenon does not apply to the Mediterranean sea.


\section{Orca call sequences}
\subsection{Context and objective}

As previously mentioned, part of mysticete communication systems have been characterised as songs for being associated with courtship. No such phenomenon has been observed in odontocetes. Nonetheless, toothed whale communication has been studied extensively, especially with bottlenose dolphins and orcas. Their vocal displays (whistles and pulsed calls) have been suggested to serve social signaling / bonding \cite{janik2014cetacean}. Bottlenose dolphins use individual specific signature whistles \cite{tyack2012review}, orcas use clan / pod specific pulsed calls \cite{ford1987catalogue}.

Given the available 5 years of consecutive recordings from the OrcaLab observatory, the following study will focus on the \ac{NRKW} population of British Columbia. First, by running it on the summers from 2015 to 2020 (season of presence of the \ac{NRKW}), more than 300 thousands calls were automatically detected by the \ac{ANN} presented in section \ref{chap:orca_detec}. The \ac{ANN} presented in section \ref{chap:orca_classif} then allowed to recognise 7 predominant call types, and to tell when unknown calls are encountered.

The observation of stereotyped calls in relation to behavioral states has suggested no strict relationship between the two, but rather a group identification function. \citet{ford1989acoustic} has manually analysed 20 thousands calls from 43 days of boat observation from 1978 to 1983, and reported call type bi-gram distributions (Fig. \ref{fig:transition_fordvself}). Some call type distributions significantly differed across activities, especially when involving multiple pods. \citet{filatova2013dependence} have manually analysed 32 hours of recordings for calls to be assigned among 4 categories. Activity did not affect proportions of occurrence, but multi-pod interactions did.

The intent of this study is to look for patterns in the sequences of types, trying to make the most out of the large but blind corpus at hand (no information activity is available).


\begin{figure}
    \centering
    \includegraphics[width=\linewidth]{fig/transitions_.pdf}
    \caption{Comparison between the transition matrix from \citet{ford1989acoustic} (left) and the present study (right).}
    \label{fig:transition_fordvself}
\end{figure}



\iffalse
\begin{table}[ht]
    \centering
    \begin{tabular}{|c|c|c|c|c|c|c|c|c|c|} \hline
 & N1 & N2 & N23 & N3 & N4 & N5 & N9 & other & Total \\ \hline 
 N1 &0.21 & 0.03 & 0.06 & 0.01 & 0.24 & 0.1 & 0.12 & 0.09 & 15,551 \\ \hline 
 N2 &0.07 & 0.15 & 0.06 & 0.01 & 0.23 & 0.1 & 0.11 & 0.1 & 6,446 \\ \hline 
 N23 &0.07 & 0.03 & 0.23 & 0.01 & 0.23 & 0.05 & 0.1 & 0.1 & 15,752 \\ \hline 
 N3 &0.04 & 0.02 & 0.04 & 0.39 & 0.17 & 0.05 & 0.09 & 0.05 & 5,187 \\ \hline 
 N4 &0.05 & 0.02 & 0.04 & 0.01 & 0.54 & 0.06 & 0.11 & 0.06 & 82,625 \\ \hline 
 N5 &0.07 & 0.03 & 0.04 & 0.01 & 0.27 & 0.22 & 0.14 & 0.08 & 17,722 \\ \hline 
 N9 &0.06 & 0.02 & 0.05 & 0.01 & 0.31 & 0.08 & 0.27 & 0.06 & 29,808 \\ \hline 
 other &0.05 & 0.02 & 0.05 & 0.01 & 0.18 & 0.05 & 0.07 & 0.1 & 27,081 \\ \hline
    \end{tabular}
    \caption{Caption}
    \label{tab:my_label}
\end{table}
\fi
\chapter{Conclusion and perspectives}

\section{Thesis contributions}

This thesis demonstrates several \ac{PAM} use cases, revolving about the use of \acp{ANN} to accelerate data analysis. It lies between a tutorial on how to use \ac{ANN}s for \ac{PAM}, an empirical study of what works and what doesn't, and the demonstration of the wide potential ahead of this approach. It is motivated by the following problematic: how to best use \acp{ANN} for cetacean vocalisation detection ? 
This thesis answers the latter in 3 folds : data annotation, architecture design and training regularisation, and detection exploitation for biological insights.

\paragraph{Methods in annotation}
Robust detection systems are needed to save analysis time on long term \ac{PAM} recordings. \acp{ANN} offer an opportunity for this, but demand annotations to be trained and evaluated on. In the absence of already available robust analysis systems (detection or classification) and annotated databases, I proposed several procedures to enhance annotation efficiency, making the most out of recording characteristics and prior knowledge on target signals.

The proposed procedures where illustrated with several use cases starting from raw recordings, yielding 6 annotated databases (5 for detection and 1 for classification).


\paragraph{Training procedures}
Given annotated databases, training \ac{ANN} allowed to solve the detection tasks for 12 target signals (5 from custom annotated databases, and 7 from the Antarctic mysticetes database). For signals with a limited variability such as sperm whale clicks and fin whale 20\,Hz pulses, relatively small (three depth-wise convolution) networks yield satisfactory performances, improved compared to previous handcraft algorithms.

As for detecting the more variable orca calls, systematic searches and heavier models also yield satisfactory performances. Several insights arise from the exploration of network frontends, architectures and hyper-parameters, but they might be task specific.

On the other hand, heavier models can also serve the detection of several target signals with a shared set of weights, as shown with Antarctic mysticete calls. In this context, performance metrics are discussed and an interpretable metric for \ac{PAM} uses is proposed.

Eventually despite efforts in using unlabeled data for self supervised representation learning and semi-supervised learning, the regular supervised approach appeared to be the most efficient for the orca call type classification task.


\paragraph{Applications}
Perhaps the most ambitious objective of this thesis was to bridge the gap between training deep learning algorithms and their application to long term bioacoustic surveys. This was conducted for the study of 3 species: sperm whales, fin whale and orcas. For each of them, different orientations were taken for the analysis. Sperm whale presence was studied in relation to anthropogenic noise, the fin whale song structure was described by long-term trends, and sequences of orca call types were analysed in search of specific patterns and dependencies.


\section{Future work}

\paragraph{Frontend experiments}
\ac{PCEN} is a promising frontend but does not lead to a systematic performance gain. Work should be oriented towards understanding better why it might be detrimental, especially when fixing its smoothing and compression parameters.

In addition, to advance on embedded capacities for real time alert systems, analog feature extraction (stack of band-pass filters) should be experimented with. This would be relevant to tackle the main computational bottleneck of embedded bioacoustic analysis: the \ac{STFT}.

\paragraph{Integration of spatial information} \label{chap:km3net_triang}
The data available at DYNI has the potential to numerous other studies than the ones conducted so far. Work on the spatialisation of acoustic sources could be conducted on the data from KM3Net and OrcaLab data. This would allow to add a new dimension of analysis when processing vocalisation sequences.

\paragraph{Intra call modulations}
The analysis of orca call sequences presented in this work was subject to the prior discretisation by types. Some information is presumably lost in this process, such as within call variations. \citet{li2020learning} propose a deep learning based whistle contour extraction procedure, which seems robust to low \ac{SNR} and overlapping calls. Experiments with this approach would be relevant to the analysis of orca call sequences.

\paragraph{Using ANNs for sequence modelling}
Modern day language modelling is often conducted with \ac{ANN} based methods, especially with the recent boom of Transformer architectures \cite{devlin2018bert}. These models could be trained on orca call sequences and yield a notion predictability and / or perplexity more reliable than with n-gram models.
\chapter{Perspectives}

\section{Frontends}
Multi-scale
work on PCEN
Analog

\section{Triangulation on KM3Net} \label{chap:km3net_triang}
+ triangulation sur OL

\section{Application to machine assisted online monitoring (SPHYRNA)}

\section{intra call modulations}
annotation de pitch, deep salience representation

\section{Using ANNs for sequence modelling}

% Starts lettered appendices, adds a heading in table of contents, and adds a
%    page that just says "Appendices" to signal the end of your main text.
\startappendices
% Add or remove any appendices you'd like here:
%\input{text/appendix/list_of_appendices}


%----------------------------------------------------------------------------------------
%	REFERENCES
%----------------------------------------------------------------------------------------

\bibliographystyle{plainnat}
\bibliography{references.bib}



\end{document}
